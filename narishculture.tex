\documentclass[a4paper,11pt,twoside,openright]{memoir}
\usepackage{multicol, multirow, array}
\usepackage{fontspec}
\usepackage{anyfontsize}
\usepackage[pagecolor=none,dvipsnames]{xcolor}
\usepackage{ragged2e}
\usepackage{amsmath}
\usepackage{amssymb}
\usepackage[hidelinks]{hyperref}
\usepackage{url}
\usepackage[margin=0.8in]{geometry}
\usepackage{float, hhline}
\usepackage{booktabs}
\usepackage{textcomp}
\usepackage{expex}
\usepackage{enumitem}
\usepackage[calc,english]{datetime2}
\usepackage{suffix}

%-----CONFIGURATION------
%------------------------

\setmainfont{Charis SIL}[CharacterVariant=43:1]
\restylefloat{table}

\setsecnumdepth{subsubsection}
\settocdepth{subsubsection}

\lingset{glstyle=nlevel,numoffset=3em,textoffset=1.5em,exskip=.75ex,belowglpreambleskip=.25ex,aboveglftskip=.25ex}

\DTMnewdatestyle{eurodate}{%
    \renewcommand{\DTMdisplaydate}[4]{%
        \number##3.\nobreakspace%           day
        \DTMmonthname{##2}\nobreakspace%    month
        \number##1%                         year
    }%
    \renewcommand{\DTMDisplaydate}{\DTMdisplaydate}%
}

\DTMsetdatestyle{eurodate}

\renewcommand{\arraystretch}{1.4}

%-------COMMANDS---------
%------------------------

\newcommand{\lang}{ɁaɁa-\textsc{f}a\textsc{m}}
\newcommand{\longv}{ː}
\newcommand{\sqbrack}[1]{$\langle$#1$\rangle$}
\newcommand{\phipa}[1]{/#1/}
\newcommand{\bripa}[1]{[#1]}
\newcommand{\ttilde}{\raise.17ex\hbox{$\scriptstyle\sim$}}
\newcommand{\rootpart}{$\Theta$}
\newcommand{\glotstop}{ʔ}
\newcommand{\bigglot}{Ɂ}
\newcommand{\lilglot}{ɂ}
\newcommand{\nm}{\symbol{"2205}}
\newcommand{\tiebar}{͡}
\newcommand\famword[1]{{\addfontfeatures{Letters=UppercaseSmallCaps}#1}}

%------TITLE PAGE--------
%------------------------

\title{{\fontsize{100}{100}\selectfont Nareland} \\ \Huge \sffamily A look at the history, legacy, and culture of the Narish people}
%Alternate subtitle: A glorified Wikipedia article formatted as a book
\author{Bethany E. Toma, Knut F. K. Ulstrup}
\date{\today}

%-------MAIN DOC---------
%------------------------

\begin{document}

\maketitle

\part{History of Nareland}

Nareland is an island, roughly equivalent in size to Sardinia, located 200km west of the coast of Norway and 50km north of the Orkney Islands of Scotland.

\chapter{Pre-history}

At the end of the last great ice age, the powerful wash of the glacial runoffs exiting the Scandinavian peninsula carried soil out into the North Sea, where the soil accumulated up against hilly formations north of Scotland to form a bank.

Archeological findings suggest that the island has been inhabited since the mesolithic period, having been settled upon as early as 8,000 years ago by a culture predating even the Narish people. As the ice sheet covering the north sea retracted, the prehistoric population of the British isles crept north in tow, while the rising sea levels eliminated any land bridges to Nareland behind them. The fate of these early indigenous peoples is uncertain, as remnants of their culture are near non-existent.

The Narish people originally inhabited the western coast of the Scandinavian peninsula well before the arrival of Indo-European settlers. 

\chapter{Early history}



\chapter{Scandinavian immigrant waves}

During the 8th and 9th century, the island saw numerous visits from Norsemen, who attempted to colonize it. Their invasion of the island was largely motivated by the overpopulation of Scandinavia, as well as opponents of Harald I Fairhair and his reign over a now-united kingdom of Norway. These attempts were initially quelled by the Narish, and written accords from the Norsemen show these struggles pervaded through much of the 9th century. However, tides turned when Harald Fairhair, attracted by the vast arable plains of Nareland, amassed his navy and sailed for the island in the 870's. Although unsuccessful in pushing through the Narish army and seizing the entire island, he was able to establish a fortified colony on the southeastern coast named Haraldsborg.

With a rise in population, the colony was prone to overflow. New settlers would seek lands outside of the fortifications on which to build their farmsteads, even neighboring more amiable Narish inhabitants in certain cases. While initial relations were hostile, over time the Scandinavian immigrants were seen in more and more positive light, and eventually the two cultures would intermingle. Norse women married into Narish families, their children were brought up inheriting both Narish and Norse culture, and the Narish culture inevitably adopted on a large scale these customs and traditions.

The Black Death hit the island in 1350, as a result of Norwegian trade ships sailing between Haraldsborg and mainland Norway. The native population saw a severe decline, with estimated remaining headcounts as low as 7,500. In the following centuries, the island saw several smaller outbreaks; none as severe as the first, but nonetheless effective at culling the population. At the start of the 18th century, Narelanders were closer to 30,000 strong.
This increase still didn't prove a match against the expanding kingdom of Denmark-Norway, who were quick to colonize the rest of the island. Haraldsborg, now renamed to Frederiksborg after King Frederick IV, was made the capital city.

With the total colonization of Nareland came Reformation as well, and Narelanders were heavy-handedly converted to Lutheran practices. 

\part{Culture of the Narish People}

\chapter{Food}

grain cultivation
livestock consists of cattle, sheep, goose, duck, pig, dog

breakfast:
hearty breakfast, very communal
bread is staple food, flatbread made from barley seasoned with caraway seeds
sausages, cured meats, and cured mackerel. 
apple and pear slices
some kinda hard cheese

lunch:
light meal, even in modern times, not necessarily communal
more buns

dinner:
largest meal, hugely communal and important

night food:
snacks, whoever's up is whoever's company

\chapter{Names}

Narish names are of widespread origins, stemming from traditional naming conventions, names of Scandinavian and British immigrants, as well as combinations thereof.

\section{Given Names}

\subsection{Native Narish}

In native Narish culture, children are generally given informal `milk names' as infants, which tend to be simple cutesy adjectives like \famword{CHaNKa}, \famword{LJaDa}, etc. based on quirks and personality traits exhibited during babyhood. At age 6, a child is given their `proper name'. This goes on to be the name used in all official capacities and by most other people throughout the child's life, though some family and very close friends from early childhood may use the milk name as a sort of affectionate nickname even later in life.

In form, Narish `proper' given names are generally formed by choosing two `roots' (as in sets of consonants) and putting in vowels wherever one thinks they sound good. The roots chosen tend to be ones with positive associations, and the vowels chosen tend to deliberately avoid any existing morphological patterns (though in some cases they are chosen to evoke particular words).

There are some cultural restrictions on proper names. It is considered improper to give someone the exact same name as anyone you know, particularly a relative, because it's believed that having the same name as someone else could confuse the spirits in charge of bringing your soul to the afterlife. Within families, further restrictions apply: neither of the word roots present in your name can be the same as one present in that of your \textit{SanaW} (a word describing any member of your close family who is two generations from you---e.g., grandparents, grandchildren, close great-aunts/uncles) or your \textit{Bana\bigglot} (a word describing close family members of your same generation---e.g., siblings and first cousins). This is also traditionally explained as being due to the risk of mix-ups when it comes to taking your soul to the afterlife, as the spirits aren't great with names and might mix you up with these family members if your names are even a bit similar.

\subsection{Scandinavian}

\subsection{Other}

\section{Surnames}

As with given names, surnames in Nareland differ based on the origin of the family in question. However, since acquiring home rule, the native Narish convention of surnames preceding given names has been legally codified even for names of non-Narish origin. Traditional Narish surnames are pure matronyms, applying suffix \textit{-aki} to one's mother's given name (e.g., the child of \textit{\textsc{Kiwanumi}} would have the surname \textsc{\textit{Kiwanumiaki}}). In some circumstances a child is given a native Narish name not based on their birth mother (for instance, when they or one of their siblings is adopted, as parents often want their children to all share a surname, or when the mother has been disgraced or refuses to claim the child)---in these cases, the surname is instead form by attaching the possessive particle \textit{wan} as a suffix to either the mother's or father's name (e.g., \textsc{\textit{Kiwanumi}} and \textsc{\textit{Ruthaj}} could choose to give their children \textit{\textsc{Kiwanumiwan}} or \textit{\textsc{Ruthajwan}} as a surname). The ability to use either a traditional Narish matronym (or patronym) or a Norse patronym as one's legal surname is protected by law, and the majority of the native Narish population use traditional Narish matronyms. The usage among those of Scandinavian origin is somewhat mixed---about 60\% use patronyms of the Norse form (e.g., the child of \textit{Kristian Pedersen} would have the surname of either \textit{Kristiansen} or \textit{Kristiansdotter}, depending on their gender), while about 30\% treat them like more modern patrilineal family names, passed down through the generations (e.g., the child of \textit{Kristian Pedersen} would have the surname \textit{Pedersen}). Generally, the more recently a family immigrated to Nareland, the more likely they are to be using family names passed down through the generations rather than patronyms or matronyms. Mixed families, of course, tend to make a variety of choices depending on their own personal circumstances. A given group of random Narelanders will tend to have surnames formed by a variety of different strategies.

Changing one's surname upon marriage is not traditionally done in Narish culture. There are laws in place allowing for surname changes, including after marriage, but it is far less common than in the surrounding countries. 

\chapter{Birthdays \& Age Reckoning}

Traditional Narish age reckoning is based on how many winters a child has survived. This means that a child born in the fall is considered to be one year old upon the beginning of the spring planting season, despite having been born only a few months prior. However, a child born \emph{during} the winter is will not have their age increase at the end of the winter and will not have their age increase until the beginning of the spring planting season the following year, as someone born during the winter is not considered to have truly lived through it.

In the modern day, 
%a couple options:
% - either age was codified in a more typical European way but Narelanders use traditional ages in more informal contexts
% - or Narish age is still used even on a legal front and has just become more exact now that the beginning and end of winter are more officially codified.

\chapter{Religion}

\chapter{Disposal of deceased people}

Because of the island's large swaths of bogland, Narelanders performed the practice of bog burials. The slow decomposition as a result of the anaerobic ecosystem was taken as a sign that it was "clean", a trait important to Narelandish culture. This practice has been commonplace since prehistoric times, and the findings of several mummified corpses that are genetically distinct from native Narelanders suggests that the practice was inherited from an unknown pre-Narish culture.

\end{document}