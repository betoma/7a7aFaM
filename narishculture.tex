\documentclass[a4paper,11pt,twoside,openright]{memoir}
\usepackage{multicol, multirow, array}
\usepackage{fontspec}
\usepackage{anyfontsize}
\usepackage[pagecolor=none,dvipsnames]{xcolor}
\usepackage{ragged2e}
\usepackage{amsmath}
\usepackage{amssymb}
\usepackage[hidelinks]{hyperref}
\usepackage{url}
\usepackage[margin=0.8in]{geometry}
\usepackage{float, hhline}
\usepackage{booktabs}
\usepackage{textcomp}
\usepackage{expex}
\usepackage{enumitem}
\usepackage[calc,english]{datetime2}
\usepackage{suffix}

%-----CONFIGURATION------
%------------------------

\setmainfont{Charis SIL}[CharacterVariant=43:1]
\restylefloat{table}

\setsecnumdepth{subsubsection}
\settocdepth{subsubsection}

\lingset{glstyle=nlevel,numoffset=3em,textoffset=1.5em,exskip=.75ex,belowglpreambleskip=.25ex,aboveglftskip=.25ex}

\DTMnewdatestyle{eurodate}{%
    \renewcommand{\DTMdisplaydate}[4]{%
        \number##3.\nobreakspace%           day
        \DTMmonthname{##2}\nobreakspace%    month
        \number##1%                         year
    }%
    \renewcommand{\DTMDisplaydate}{\DTMdisplaydate}%
}

\DTMsetdatestyle{eurodate}

\renewcommand{\arraystretch}{1.4}

%-------COMMANDS---------
%------------------------

\newcommand{\lang}{ɁaɁa-\textsc{f}a\textsc{m}}
\newcommand{\longv}{ː}
\newcommand{\sqbrack}[1]{$\langle$#1$\rangle$}
\newcommand{\phipa}[1]{/#1/}
\newcommand{\bripa}[1]{[#1]}
\newcommand{\ttilde}{\raise.17ex\hbox{$\scriptstyle\sim$}}
\newcommand{\rootpart}{$\Theta$}
\newcommand{\glotstop}{ʔ}
\newcommand{\bigglot}{Ɂ}
\newcommand{\lilglot}{ɂ}
\newcommand{\nm}{\symbol{"2205}}
\newcommand{\tiebar}{͡}
\newcommand{\famword}[5]{#1\textsc{#2}#3\textsc{#4}#5}

%------TITLE PAGE--------
%------------------------

\title{{\fontsize{100}{100}\selectfont Nareland} \\ \Huge \sffamily A look at the history, legacy, and culture of the Narish people}
%Alternate subtitle: A glorified Wikipedia article formatted as a book
\author{Bethany E. Toma, Knut F. K. Ulstrup}
\date{\today}

%-------MAIN DOC---------
%------------------------

\begin{document}

\maketitle

\part{History of Nareland}

Nareland is an island, roughly equivalent in size to Sardinia, located 200km west of the coast of Norway and 50km north of the Orkney Islands of Scotland.

\chapter{Pre-history}

At the end of the last great iceage, the powerful wash of the glacial runoffs exiting the Scandinavian peninsula carried soil out into the North Sea, where the soil accumulated up against hilly formations north of Scotland to form a bank.

Archeological findings suggest that the island has been inhabited since the mesolithic period, having been settled upon as early as 8,000 years ago by a culture predating even the Narish people. As the ice sheet covering the north sea retracted, the prehistoric population of the British isles crept north in tow, while the rising sea levels eliminated any land bridges to Nareland behind them. The fate of these early indigenous peoples is uncertain, as remnants of their culture are near non-existent.

The Narish people originally inhabited the western coast of the Scandinavian peninsula well before the arrival of Indo-European settlers. 

\chapter{Early history}



\chapter{Scandinavian immigrant waves}

During the 8th and 9th century, the island saw numerous visits from Norsemen, who attempted to colonize it. Their invasion of the island was largely motivated by the overpopulation of Scandinavia, as well as opponents of Harald I Hairfair and his reign over a now-united kingdom of Norway. These attempts were initially quelled by the Narish, and written accords from the Norsemen show these struggles pervaded through much of the 9th century. However, tides turned when Harald Hairfair, attracted by the vast arable plains of Nareland, amassed his navy and sailed for the island in the 870's. Although unsuccessful in pushing through the Narish army and seizing the entire island, he was able to establish a fortified colony on the southeastern coast named Haraldsborg.

With a rise in population, the colony was prone to overflow. New settlers would seek lands outside of the fortifications on which to build their farmsteads, even neighboring more amiable Narish inhabitants in certain cases. While initial relations were hostile, over time the Scandinavian immigrants were seen in more and more positive light, and eventually the two cultures would intermingle. Norse women married into Narish families, their children were brought up inheriting both Narish and Norse culture, and the Narish culture inevitably adopted on a large scale these customs and traditions.

The Black Death hit the island in 1350, as a result of Norwegian trade ships sailing between Haraldsborg and mainland Norway. The native population saw a severe decline, with estimated remaining headcounts as low as 7,500. In the following centuries, the island saw several smaller outbreaks; none as severe as the first, but nonetheless effective at culling the population. At the start of the 18th century, Narelanders were closer to 30,000 strong.
This increase still didn't prove a match against the expanding kingdom of Denmark-Norway, who were quick to colonize the rest of the island. Haraldsborg, now renamed to Frederiksborg after King Frederick IV, was made the capital city.

With the total colonization of Nareland came Reformation as well, and Narelanders were heavy-handedly converted to Lutheran practices. 

\part{Culture of the Narish People}

\chapter{Food}

grain cultivation
livestock consists of cattle, sheep, goose, duck, pig, dog

breakfast:
hearty breakfast, very communal
bread is staple food, flatbread made from barley seasoned with caraway seeds
sausages, cured meats, and cured mackerel. 
apple and pear slices
some kinda hard cheese

lunch:
light meal, even in modern times, not necessarily communal
more buns

dinner:
largest meal, hugely communal and important

night food:
snacks, whoever's up is whoever's company

\chapter{Names}

Narish names are of widespread origins, stemming from traditional naming conventions, names of Scandinavian and British immigrants, as well as combinations thereof.

\chapter{Religion}



\end{document}