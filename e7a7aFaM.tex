\documentclass[a4paper,10pt,twoside,openright]{memoir}
\usepackage{multicol, multirow, array}
\usepackage{fontspec}
\usepackage{anyfontsize}
\usepackage[pagecolor=none,dvipsnames]{xcolor}
\usepackage{ragged2e}
\usepackage{amsmath}
\usepackage{amssymb}
\usepackage[hidelinks]{hyperref}
\usepackage{url}
\usepackage[margin=0.8in]{geometry}
\usepackage{float, hhline}
\usepackage{booktabs}
\usepackage{textcomp}
\usepackage{expex}
\usepackage{enumitem}
\usepackage[calc,english]{datetime2}

%-----CONFIGURATION------
%------------------------

\setmainfont{Charis SIL}[CharacterVariant=43:1]
\restylefloat{table}

\setsecnumdepth{subsubsection}
\settocdepth{subsubsection}

\lingset{glstyle=nlevel,numoffset=3em,textoffset=1.5em,exskip=.75ex,belowglpreambleskip=.25ex,aboveglftskip=.25ex}

\DTMnewdatestyle{eurodate}{%
    \renewcommand{\DTMdisplaydate}[4]{%
        \number##3.\nobreakspace%           day
        \DTMmonthname{##2}\nobreakspace%    month
        \number##1%                         year
    }%
    \renewcommand{\DTMDisplaydate}{\DTMdisplaydate}%
}

\DTMsetdatestyle{eurodate}

\renewcommand{\arraystretch}{1.4}

%-------COMMANDS---------
%------------------------

\newcommand{\lang}{ɁaɁa\textsc{f}a\textsc{m}}
\newcommand{\longv}{ː}
\newcommand{\sqbrack}[1]{$\langle$#1$\rangle$}
\newcommand{\phipa}[1]{/#1/}
\newcommand{\bripa}[1]{[#1]}
\newcommand{\ttilde}{\raise.17ex\hbox{$\scriptstyle\sim$}}
\newcommand{\rootpart}{$\Theta$}
\newcommand{\glotstop}{ʔ}
\newcommand{\bigglot}{Ɂ}
\newcommand{\lilglot}{ɂ}
\newcommand{\nm}{\symbol{"2205}}
\newcommand{\tiebar}{͡}
\newcommand{\famword}[5]{#1\textsc{#2}#3\textsc{#4}#5}

%-------TITLE PAGE-------
%------------------------

\title{{\fontsize{80}{80}\selectfont \lang} \\ \Huge \sffamily A Reference Grammar}
\author{Bethany E. Toma, Knut F. K. Ulstrup}
\date{\today}

%--------MAIN DOC--------
%------------------------

\begin{document}

\pagecolor{Melon}
\maketitle
\pagecolor{white}

\frontmatter

\chapter{Foreword}

\lang{} is a constructed language.

\newpage

\tableofcontents

\mainmatter

\part{Grammar}

\chapter{Phonology}
\section{Consonants}

\begin{table}[ht]
    \centering
    \begin{tabular}{rcccccc}
    \toprule
            & Labial & Alveolar & Palatal & Velar & Uvular & Glottal \\
    \midrule
    Fortis & pʰ \ttilde{} p\tiebar ɸ & tʰ \ttilde{} t\tiebar s &
    % \multirow{2}{*}{c \ttilde{} c\tiebar ç} & \multirow{2}{*}{k \ttilde{} k\tiebar x} & \multirow{2}{*}{q \ttilde{} q\tiebar χ} & \multirow{2}{*}{ʔ} \\
    c \ttilde{} c\tiebar ç & k \ttilde{} k\tiebar x & q \ttilde{} q\tiebar χ & \multirow{2}{*}{ʔ} \\
     Lenis & p \ttilde{} b & t \ttilde{} d & & & & \\
    Fricative & ɸ & s & \multicolumn{4}{c}{ç \enspace \ttilde{} \enspace x \enspace \ttilde{} \enspace χ \enspace \ttilde{} \enspace ħ \enspace \ttilde{} \enspace h} \\
    Approximant & & l & j & w & & \\
    Nasal & m & n & & & & \\
    Rhotic & & \multicolumn{4}{c}{ɾ \enspace \ttilde{} \enspace r \enspace \ttilde{} \enspace ɹ \enspace \ttilde{} \enspace ɽ \enspace \ttilde{} \enspace ɻ \enspace \ttilde{} \enspace ʀ \enspace \ttilde{} \enspace ʁ } & \\
    \bottomrule
    \end{tabular}
    \caption{Phonemic Consonant Inventory}
    \label{tab:consinv}
\end{table}

\section{Vowels}

\begin{table}[ht]
    \centering
    \begin{tabular}{rccc}
    \toprule
          & Front & Central & Back \\
    \midrule
    Close & i (i\longv{}) & & u (u\longv) \\
    Mid   & & ə & \\
    Open  & & a (a\longv) & \\
    \bottomrule
    \end{tabular}
    \caption{Phonemic Vowel Inventory}
    \label{tab:vowelinv}
\end{table}

\subsection{Epenthetic schwa}

\section{Prosody}

Stress, in the form of elevated pitch and volume, is placed on the first non-schwa vowel of the word, after the first root radical, on a long vowel immediately preceding the first radical, or on certain morphemes that carry stress.

\begin{table}[ht]
    \centering
    \begin{tabular}{lll}
        nemiwi & [nəˈmiwi] & first non-schwa vowel of word\\
        parse & [ˈparsə] & first non-schwa vowel of word\\
        \famword{i}{f}{aa}{m}{} & [iˈfa{\longv}m] & vowel after first radical\\
        \famword{}{f}{ana}{s}{} & [ˈfanas] & vowel after first radical\\
        \famword{aa}{n}{i}{w}{} & [ˈa{\longv}niw] & long vowel preceding radical\\
        \famword{i}{l}{aa}{s}{}ak & [iˌla{\longv}ˈsak] & presence of stress-carrying morpheme (imperative affix \emph{-ak})
    \end{tabular}
\end{table}


\section{Morphophonemics}

\section{Orthography}

\lang{} has two recognized orthographic conventions, both based on the Latin alphabet. Both conventions use marked letterforms to indicate which part of a word are part of the underlying root and which are grammatical markers. The precise manner in which they're marked is the major point of difference between the two orthographic styles.

By and large, both orthographic conventions attempt to use the most intuitive representation of a given phoneme. There are very few differences between the conventions. Fortis and lenis stops are written using the typical voiceless and voiced symbols, respectively, in both systems. The labial fricative is written as \sqbrack{f} and the dorsal fricative as \sqbrack{h}. The palatal approximant is written using \sqbrack{j}, and the rhotic is, of course, written as \sqbrack{r}. The other phonemes are written with their usual IPA characters in both conventions, except for \phipa{\glotstop}, which is dealt with differently depending on which convention one is using.

\subsection{Formal writing style}

The formal writing conventions make use of small-caps letterforms to highlight roots. In addition, it uses the glottal stop character to indicate the glottal stop phoneme, using the capital glottal stop character \sqbrack{\bigglot} when the glottal stop is part of a root radical (for instance, in the word \textit{\bigglot a\bigglot a}) and the lowercase glottal stop character \sqbrack{\lilglot} otherwise (such as in the suffix \textit{-(e)\lilglot}).

\subsection{Informal writing style}

The informal writing conventions, also known as ``texting script", is the orthography used in the majority of day-to-day communication. Rather than using small-caps letterforms, it uses true capital letters for roots. It also uses \sqbrack{7} for the glottal stop, with no difference between capital and lowercase. While these differences could be considered less aesthetically pleasing, they result in an ASCII-compatible script, which makes this writing style far easier to use in most messaging apps and computer interfaces. Texting-style \lang{} also allows for several shorthand abbreviations that tend not to be used in more formal style.

\chapter{Morphology}
\section{Underlying roots}
\section{Derivational morphology}

\lang{} allows for words to be altered syntactically and semantically using a rich set of morphological operations, divided into two categories based on their concatenation. 

\subsection{Primary derivation}

Primary derivation refers to the non-concatenative morphology of stems. These operations are for the most part not productive, and not all roots have a corresponding stem with each of these patterns. They may not stack, i.e. a stem may only be inflected by one pattern at a time.

\begin{table}[ht]
    \centering
    \begin{tabular}{llll}
    \textit{Pattern} & \textit{Meaning} & \textit{Example} & \\
    \multirow{2}{*}{{\rootpart}a{\rootpart}} & \multirow{2}{*}{Abstract noun}& \famword{}{s}{a}{j}{} & sleep \emph{(cf. \famword{i}{s}{aa}{j}{} `to sleep')}\\
    & & \famword{}{k}{a}{l}{} & rainfall \emph{(cf. \famword{}{k}{ur}{l}{i} `raindrop')}\\
    {\rootpart}ana{\rootpart} & Person of X, Agentive noun & \famword{}{k}{ana}{j}{} & author \emph{(cf. \textsc{k}ii\textsc{j} `to write X')}\\
    {\rootpart}ar{\rootpart}i & Liquid noun & \famword{}{q}{ar}{f}{i} & coffee \emph{(cf. i\textsc{q}aa\textsc{f} `to drink coffee')} \\
    {\rootpart}ur{\rootpart}i & Object noun & \famword{}{n}{ur}{m}{i} & food \emph{(cf. \textsc{n}ii\textsc{m} `to eat X')}\\
    {\rootpart}idi{\rootpart} & Loose granular mass & \famword{}{w}{idi}{w}{} & sugar \emph{(cf. \textsc{w}a\textsc{w}a `sweet')} \\
    {\rootpart}asi{\rootpart} & Long slender object & \famword{}{b}{asi}{t}{} & hair \emph{(cf. \textsc{b}uli\textsc{t} `head')} \\
    {\rootpart}uli{\rootpart} & Associated body part & \famword{}{b}{uli}{t}{} & head \emph{(cf. i\textsc{b}aa\textsc{t} `to understand')}\\
    m{\rootpart}i{\rootpart} & Instrument, tool & \famword{m}{r}{i}{w}{} & weapon \emph{(cf. \textsc{r}a\textsc{q} `pain')} \\
    i{\rootpart}u{\rootpart}a & Place of X/with X attribute & \famword{i}{h}{u}{j}{a} & night \emph{(cf. \textsc{h}a\textsc{t}a `dark')} \\
    {\rootpart}ii{\rootpart} & Transitive verb & \famword{}{f}{ii}{s}{} & to give birth to \emph{(cf. \textsc{f}ana\textsc{s} `person')} \\
    i{\rootpart}aa{\rootpart} & Intransitive verb & \famword{i}{\bigglot}{aa}{\bigglot}{} & to act stupidly \emph{(cf. e\bigglot a\bigglot a `dumb')} \\
    aa{\rootpart}i{\rootpart} & Causative verb & TBD & TBD \\
    {\rootpart}a{\rootpart}a & Primary attribute & \famword{}{s}{a}{fr}{a} & hot \emph{(cf. \textsc{s}a\textsc{f}e\textsc{r} `heat')} \\
    {\rootpart}u{\rootpart}u & Animal & \famword{}{b}{u}{rk}{u} & dog \emph{(cf. \textsc{b}a\textsc{rk} `bark')} \\
    {\rootpart}uu{\rootpart} & Country & \famword{}{f}{uu}{ns}{} & France \emph{(cf. \textsc{f}u\textsc{ns}u `frog')} \\
    {\rootpart}aju{\rootpart}a & Flat plane, surface & \famword{}{k}{aju}{l}{a} & Water surface \emph{(cf. \textsc{k}ar\textsc{l}i `water')}
    \end{tabular}
    \caption{Primary derivation patterns}
    \label{tab:primedevs}
\end{table}

\subsection{Secondary derivation}

Secondary derivation refers to the exclusively suffixing operations that may be applied to stems in addition to primary derivation. Unlike primary derivation, these suffixes may be stacked freely. 

\subsubsection{\emph{-uru} - `to be'}

\subsubsection{\emph{-ila} - `to have'}

\subsubsection{\emph{-ara} - wishes and greetings}

\subsubsection{\emph{-iri} - `to make'}

\subsubsection{\emph{-ana} - person}

\subsubsection{\emph{-ari} - `to become', `to cause to be'}

\subsection{Gender}

Certain lexical items may be inflected to convey the gender of its referent. On certain words, namely \emph{-ara} greetings, gender marking is obligatory.

\begin{table}[ht]
    \centering
    \begin{tabular}{>{\em}ll}
    -un & Feminine gender \\
    -aj & Masculine gender \\
    -uj & Explicitly non-binary \\
    -an & Gender-neutral, agender \\
    \end{tabular}
\end{table}



\section{Inflectional morphology}

\subsection{Verb finals}

\subsection{Evidential modality}

\section{Pronouns and determiners}

\begin{table}[ht]
    \centering
    \begin{tabular}{rll}
        & \textit{Nonplural} & \textit{Plural} \\
    \textit{Speaker-only} & nas & naswi \\
    \textit{Addressee-only} & mi & miwi \\
    \textit{Inclusive} & nemi & nemiwi \\
    \end{tabular}
    \caption{Discourse participant pronouns}
    \label{tab:firstandsecond}
\end{table}

\begin{table}[ht]
    \centering
    \begin{tabular}{>{\em}rll}
        & \textit{Determiner} & \textit{Pronoun}  \\
    Proximal & wa & wase \\
    Medial & par & parse \\
    Distal & bu & buse \\
    Interrogative & li & lise \\
    Relative & kun & kunse 
    \end{tabular}
    \caption{Determiners and demonstrative pronouns}
    \label{tab:determiners}
\end{table}

\chapter{Syntax}

\section{Verb stacking}

\section{Subordinate clauses}

Full verb phrases may be nominalized and act as an argument of another predicate.

\subsection{Relative clauses}

Relative clauses are a type of subordinate clauses that describes a referent's states or actions. They are internally headed, always verb-final, and the relative determiner \emph{kun} is used to mark the head of the clause, i.e. the thing that is being described.

\ex
\begingl
\famword{}{f}{ana}{s}{}[person]
\famword{i}{l}{aa}{s}{}[walk]
-tu[\textsc{rel}]
\famword{}{s}{a}{j}{a}uru[sleepy:\textsc{cop}]
\glft `The person who walked home was sleepy.'
\endgl
\xe

Clauses with a single argument do not require that the head is marked, as the argument is assumed to be the head by default. Still, the verb itself can be marked to describe the realization or performance of the action.

\ex
\begingl
\famword{in}{f}{i}{m}{}[children]
kun[\textsc{rel}]
\famword{i}{m}{aa}{w}{}[play]
-tu[\textsc{rel}]
naswi[\textsc{1p.ex}]
\famword{}{d}{ii}{l}{}[look]
\glft `We watched the playtime that the children were having'
\endgl
\xe

In high-valency clauses, \emph{kun} becomes more pertinent. The most agentive argument (subject) is considered to be the head of the phrase, but may still be marked for emphasis.

\pex[interpartskip=3ex]
\a
\begingl
(kun)[\textsc{rel}]
\famword{}{f}{ana}{s}{}[person]
\famword{i}{f}{u}{s}{a}[house]
daw[to]
fit[in]
\famword{i}{l}{aa}{s}{}tu[walk\textsc{:rel}]
nas[\textsc{1s}]
\famword{}{f}{ii}{l}{}[see]
\glft `I saw the person who walked into the house.'
\endgl
\a
\begingl
\famword{}{f}{ana}{s}{}[person]
kun[\textsc{rel}]
\famword{i}{f}{u}{s}{a}[house]
daw[to]
fit[in]
\famword{i}{l}{aa}{s}{}tu[walk\textsc{:rel}]
nas[\textsc{1s}]
\famword{}{f}{ii}{l}{}[see]
\glft `I saw the house that the person walked into.'
\endgl
\a
\begingl
\famword{}{f}{ana}{s}{}[person]
\famword{i}{f}{u}{s}{a}[house]
daw[to]
fit[in]
kun[\textsc{rel}]
\famword{i}{l}{aa}{s}{}tu[walk\textsc{:rel}]
nas[\textsc{1s}]
\famword{}{f}{ii}{l}{}[see]
\glft `I saw how the person walked into the house.'
\endgl
\xe

An alternative to using a determiner is simply to topicalize a given constituent. Only noun phrases may be relativized through topicalization; the relative verb may not be periphrastically topicalized (i.e. left-dislocated), as this introduces major syntactical ambiguities.

\section{Comparative constructions}

from-comparative, marks standard (to which is compared)

\pex[interpartskip=3ex]
\a
\begingl
\famword{}{p}{u}{m}{u}[rabbit]
\famword{}{f}{ana}{s}{}[person]
fun[from]
\famword{}{m}{a}{nt}{a}[big]
-uru[\textsc{cop}]
\glft `The rabbit was bigger than a person.'
\endgl
\a
\begingl
\famword{}{t}{a}{n}{}[\textsc{top}/time]
nemi[\textsc{qual}/\textsc{du.in}]
buse[\textsc{std}/\textsc{dist:pn}]
fun[\textsc{mrk}/from]
\famword{}{j}{a}{l}{}[/many\_things]
-ila[/have]
\glft `We have more time than them.'
\endgl
\xe

\section{Animacy hierarchy}

\begin{table}[ht]
    \centering
    \begin{tabular}{ll}
    0 & Natural Forces \\
    1 & Pronouns (1>2>3) \\
    2 & Speakers of \lang{} \\
    3 & Non-speakers of \lang{} \\
    4 & Higher-order animals (mammals, octopus, intelligent creatures) \\
    5 & \parbox[t]{7cm}{Body parts, tools, any inanimate object used for acting upon something} \\
    6 & Lower-order animals \\
    7 & Plants \\
    8 & Inanimate objects \\
    9 & Abstract concepts 
    \end{tabular}
    \caption{Animacy hierarchy in nominals}
    \label{tab:hierarchy}
\end{table}

\section{Causative constructions}

\lang{} has several different strategies when it comes to causative constructions, depending on the nature of the predicate in question. Some of these are morphological in nature, while others more periphrastic. 

\subsection{\textit{-ari} for nominal and adjectival predicates}

Simple nominal and adjectival predicates are turned into causatives using the translative suffix \textit{-ari}. If the predicate in question would be expressed with \textit{-uru} in its non-causative form, \textit{-ari} is likely appropriate for the causative.

\pex
\a
\begingl
\famword{}{q}{ar}{f}{i}[coffee]
\famword{}{s}{a}{fr}{a}-[hot-]@
uru[\textsc{cop}]
\glft `The coffee is hot.'
\endgl
\a
\begingl
\famword{}{q}{ar}{f}{i}[coffee]
nas[\textsc{1sg}]
\famword{}{s}{a}{fr}{a}-[hot]@
ari[\textsc{transl}]
\glft `I heated up the coffee.'
\endgl
\xe

When used with only one argument, verbs ending in \textit{-ari} are assumed to have a null subject and the argument serving as the unaccusative object. This results in \textit{-ari} also serving as `to become' (the reason for its being glossed as `translative') as well as `to cause to be'.

\ex
\begingl
\famword{}{q}{ar}{f}{i}[coffee]
\famword{}{s}{a}{fr}{a}-[hot]@
ari[\textsc{transl}]
\glft `The coffee got hot.'
\endgl
\xe

\subsection{Valency-increasing verb patterns}

Which pattern is used to form the causative of a predicate depends largely on the nature of the intransitive form of that root. There are two different potentially valency-increasing patterns that can be used for verbs: the {\rootpart}ii{\rootpart} and the aa{\rootpart}i{\rootpart}. The exact effect of each of these valency-increasing operations depends on the individual root; their behavior can differ.

For verbs that would be agentive ambitransitives in English, such as `to eat', generally the behavior is rather straightforward: the {\rootpart}ii{\rootpart} form turns the verb into a straightfoward transitive, and the aa{\rootpart}i{\rootpart} form serves as a causative of the intransitive. 

\pex
\a
\begingl
nas[\textsc{1sg}]
\famword{i}{n}{aa}{m}{}[eat\textbackslash\textsc{intr}]
\glft `I was eating.'
\endgl
\a
\begingl
nas[\textsc{1sg}]
\famword{}{k}{ur}{k}{i}[cookie]
\famword{}{n}{ii}{m}{}[eat\textbackslash\textsc{tr}]
\glft `I ate a cookie.'
\endgl
\a
\begingl
nas[\textsc{1sg}]
\famword{in}{m}{i}{m}{}[parent\_child\textbackslash\textsc{dim}]
\famword{aa}{n}{i}{m}{}[eat\textbackslash\textsc{caus}]
\glft `I fed my daughter.'
\endgl
\xe

It's worth noting that object of the transitive verb cannot be included as the object of the causative verb; the causative verb can still only have two arguments.

\ex
\ljudge{*}
\begingl
nas[\textsc{1sg}]
\famword{in}{m}{i}{m}{}[parent\_child\textbackslash\textsc{dim}]
\famword{}{k}{ur}{k}{i}[cookie]
\famword{aa}{n}{i}{m}{}[eat\textbackslash\textsc{caus}]
\endgl
\xe

\noindent To express this notion, a periphrastic causative would be required.

Other types of verbal paradigms make this causative relationship less obvious and use these roots in other ways. For instance, for some roots the intransitive form is unaccusative or passive in nature. In these cases, the transitive form behaves as a causative:

\pex
\a
\begingl
nas[\textsc{1sg}]
wan[\textsc{poss}]
\famword{}{m}{ana}{m}{}[parent\_child]
\famword{i}{n}{aa}{w}{}[death\textbackslash\textsc{intr}]
\glft `My mother died.'
\endgl
\a
\begingl
nas[\textsc{1sg}]
\famword{}{m}{ana}{m}{}[parent\_child]
\famword{}{n}{ii}{w}{}[death\textbackslash\textsc{tr}]
\glft `I killed my mother.'
\endgl
\xe

For these roots, the aa{\rootpart}i{\rootpart} form means the same thing as the {\rootpart}ii{\rootpart} form, but while the {\rootpart}ii{\rootpart} form implies a successfully completed action, the same implication is not present for the causative form.

\ex
\begingl
nas[\textsc{1sg}]
\famword{}{m}{ana}{m}{}[parent\_child]
\famword{aa}{n}{i}{w}{}[death\textbackslash\textsc{caus}]
\glft `I tried to kill my mother' (and she may or may not have died).
\endgl
\xe

\noindent For many of these roots, the intransitive is identical in meaning to a `passive' use of the transitive with an omitted subject; whether there is any noticeable difference between these depends on the verb.

\ex
\begingl
nas[\textsc{1sg}]
wan[\textsc{poss}]
\famword{}{m}{ana}{m}{}[parent\_child]
\famword{}{n}{ii}{w}{}[death\textbackslash\textsc{tr}]
\glft `My mother was killed.'
\endgl
\xe

% to-do
Unergative verbs

\subsection{Periphrastic causatives}

In addition to the morphological causatives above and their aforementioned limitations, \lang{} has a periphrastic causative that can scope over a wider variety of predicates. This periphrasis is expressed through a serial construction using the verb \textit{\textsc{w}\famword{e}{s}{ii}{n}{}} `to effect, to cause' followed by the description of the caused predicate. 

\ex
\begingl
nas[\textsc{1sg}]
\textsc{w}\famword{e}{s}{ii}{n}{}[bring\_about]@
,[]
\famword{}{q}{ar}{f}{i}[coffee]
mi[\textsc{2sg}]
\famword{}{k}{ii}{l}{}[drink]
\glft `I caused you to drink coffee.' (lit., `I brought it about, you drank coffee.')
\endgl
\xe

Insert stuff about causatives and directness here.

\chapter{Semantics and pragmatics}
\section{Phatic expressions}

Phatic expressions in \lang{} are all in some way related to the nouns they are derived from, suggesting an emphasis on acknowledging the addressee's current or upcoming actions. The addressee may respond with the same expression back, even if it does not apply to the original speaker in any way, or respond in kind with a more suitable expression.

The obligatory gender marking is a means of expressing your gender identity in an unintrusive manner.\footnote{The real reason is that as Beth once ended a conversation with "sayonara", Knut noticed some coincidental similarities with the word \famword{}{s}{a}{j}{} `sleep' and the affix -un to indicate feminine gender, with the -ara reanalyzed as a phatic/optative marker of sorts.}

\subparagraph{\famword{}{f}{a}{s}{anara}} \textit{(from \famword{}{f}{a}{s}{} `life')} is a catch-all greeting, suitable for any time of day. 

\subparagraph{\famword{}{s}{a}{j}{anara}} \textit{(from \famword{}{s}{a}{j}{} `sleep')} is similar in use to "good night", but is only used if the person is going to bed, not just leaving for the night.

\section{Idiomatic expressions}

\famword{}{c}{u}{mp}{u} \famword{}{c}{u}{mp}{uuru} = no shit, preaching to the choir

\part{Dictionary}

\newcommand{\newentry}[2]{\item[#1] $\bullet$ \textit{#2}\hfill}

\setsecnumdepth{part}
\settocdepth{section}

\chapter{Roots and Derived Words}
\begin{multicols*}{2}
\section{\textsc{b---t}}
\begin{description}[leftmargin=*]
    \newentry{\textsc{b}ii\textsc{t}}{v.tr.}
    \begin{enumerate}
        \item to know \textit{smth.}, to understand \textit{smth.}
        \begin{quote}
            mi i\textsc{b}aa\textsc{rb}e\lilglot{} kajuc nas \textsc{b}ii\textsc{t}\\
            \textit{`I know that you want to leave.'}
        \end{quote}
        \item to love \textit{sme.} like a brother, to have a close platonic bond with \textit{sme.}, to be best friends with \textit{sme.}
        \begin{quote}
            nas \textsc{J}ana\textsc{B} \textsc{b}ii\textsc{t}ibi\\
            \textit{`I love my friends.'}
        \end{quote}
        \textit{(NB: the subject is reversed from its use as `to understand': \emph{mi nas \textsc{b}ii\textsc{t}ibi} means `you understand me' but `I love you'.)}
    \end{enumerate}
    \newentry{\famword{i}{b}{aa}{t}{}}{v.intr.}
    \begin{enumerate}
        \item to know, to understand, to be in a state of knowing or understanding what is going on
        \item \textit{(when used reciprocally)} to love each other, to have a close platonic bond, to be the best of friends
        \begin{quote}
            nemi i\textsc{b}aa\textsc{t}ami\\
            \textit{`The two of us are thick as thieves.'}
        \end{quote}
    \end{enumerate}
\end{description}

\section{\textsc{f---m}}
\begin{description}[leftmargin=*]
    \newentry{\famword{aa}{f}{i}{m}{}}{v.}
    \begin{description}[labelwidth=*]
        \item[] to quote \textit{sme.}
    \end{description}
    \newentry{\famword{a}{f}{i}{m}{u}}{n.}
    \begin{enumerate}
        \item book
        \item alphabet soup
    \end{enumerate}
    \newentry{\famword{}{f}{a}{m}{}}{n.}
    \begin{description}[labelwidth=*]
        \item[] language, speech, way of speaking
    \end{description}
    \newentry{\famword{}{f}{ii}{m}{}}{v.tr.}
    \begin{description}[labelwidth=*]
        \item[] to say \textit{smth.}, to speak \textit{smth.}, to tell \textit{smth.}
    \end{description}
    \newentry{\famword{i}{f}{aa}{m}{}}{v.intr.}
    \begin{description}[labelwidth=*]
        \item[] to talk, to speak, to chatter
    \end{description}
    \newentry{\famword{in}{f}{i}{m}{}}{n.}
    \begin{description}[labelwidth=*]
        \item[] word
    \end{description}
    \newentry{\famword{in}{f}{i}{m}{ini}}{n.}
    \begin{description}[labelwidth=*]
        \item[] letter, character, symbol
    \end{description}
    \newentry{\lang{}}{n.}
    \begin{description}[labelwidth=*]
        \item[] this language, \lang{}
    \end{description}
\end{description}

\section{\textsc{k---l}}
\begin{description}[leftmargin=*]
    \newentry{\famword{a}{k}{i}{l}{u}}{n.}
    \begin{description}[labelwidth=*]
        \item[] container of water to be drunk from, glass, cup, water bottle
        \begin{quote}
            \textsc{mark} se kaj men \famword{a}{k}{i}{l}{uila}\\
            \textit{`Mark owned five water bottles.'}
        \end{quote}
    \end{description}
    \newentry{\famword{i}{k}{aa}{l}{}}{v.intr.}
    \begin{enumerate}
        \item \textit{(impersonal)} to be a rainy day
        \begin{quote}
            \famword{waj}{h}{u}{j}{a} \famword{i}{k}{aa}{l}{}\\
            \textit{`Today's a rainy day.'}
        \end{quote}
        \item \textit{(impersonal)} to be raining
        \begin{quote}
            \famword{i}{m}{u}{nt}{a} daw nas \famword{i}{j}{aa}{t}{e\lilglot} kaj da buse fit \famword{i}{k}{aa}{l}{}\\
            \textit{`I wanted to go to the mountains, but it's raining there.'}
        \end{quote}
    \end{enumerate}
    \newentry{\famword{}{k}{aju}{l}{a}}{n.}
    \begin{enumerate}
        \item the surface of a body of water
        \begin{quote}
            naswi \famword{}{k}{aju}{l}{a} tui \famword{}{f}{ii}{l}{ami\lilglot} dak\\
            \textit{`We could see ourselves on the water's surface.'}
        \end{quote}
        \item puddle
        \begin{quote}
            nas \famword{}{k}{aju}{l}{a} daw tui \famword{i}{l}{aa}{s}{}\\
            \textit{`I stepped in a puddle.'}
        \end{quote}
        \item map
            \begin{quote}
                par \famword{}{k}{aju}{l}{a} \famword{}{sw}{uu}{t}{ilali}?\\
                \textit{`Is Switzerland on that map?'}
            \end{quote}
    \end{enumerate}
    \newentry{\famword{}{k}{a}{l}{}}{n.}
    \begin{description}[labelwidth=*]
        \item[] humidity, wetness, dampness
        \begin{quote}
            \famword{i}{f}{u}{s}{a} \famword{}{j}{a}{b}{auru}, da \famword{}{l}{aju}{s}{a} \famword{}{k}{a}{l}{ila}\\
            \textit{`The house is lovely, but the floors are damp.'}
        \end{quote}
    \end{description}
    \newentry{\famword{}{k}{a}{l}{a}}{adj.} 
    \begin{enumerate}
        \item covered in water, saturated with water, wet, soaked
        \begin{quote}
            \famword{}{k}{a}{l}{a} \famword{m}{l}{i}{s}{} \famword{}{s}{a}{fr}{aariak}\\
            \textit{`Warm up your wet shoes.'}
        \end{quote}
        \item fluid, liquid, melted
        \begin{quote}
            \famword{}{k}{asi}{l}{} \famword{}{k}{a}{l}{a} \famword{}{c}{ur}{kl}{iila}\\
            \textit{`The river was made of melted chocolate'}
            %KNUT PLEASE WEIGH IN ON THIS
            %I LIKE IT THIS IS GOOD -K
        \end{quote}
    \end{enumerate}
    \newentry{\famword{}{k}{ar}{l}{i}}{n.}
    \begin{description}[labelwidth=*]
        \item[] liquid water, fresh water, water not part of a body of water or stream, water served as a beverage
        \begin{quote}
            mi \famword{}{n}{ar}{k}{a} \famword{}{k}{ar}{l}{iilali}?\\
            \textit{`Do you have any cold water?'}
        \end{quote}
    \end{description}
    \newentry{\famword{}{k}{asi}{l}{}}{n.}
        \begin{enumerate}
            \item river, stream
            \item stream or sprinkle of water, as from a faucet or tap
            \begin{quote}
                \famword{i}{s}{u}{n}{a} \famword{}{b}{a}{b}{a} \famword{}{k}{asi}{l}{ila}\\
                \textit{`The shower is low-flow'}\\
                (lit., \textit{`The shower has a mild stream'})
            \end{quote}
        \end{enumerate}
    \newentry{\famword{}{k}{u}{l}{i}}{adj., n.}
        \begin{description}[labelwidth=*]
            \item[] blue, the color blue
        \end{description}
    \newentry{\famword{}{k}{u}{l}{u}}{ n.}
        \begin{description}[labelwidth=*]
            \item[] fish, fish-adjacent aquatic animal
        \end{description}
    \newentry{\famword{}{k}{ur}{l}{i}}{n.}
        \begin{description}[labelwidth=*]
            \item[] the Earth, the globe
        \end{description}
\end{description}

\section{\textsc{pl---s}}
\begin{description}[leftmargin=*]
    \newentry{\famword{aa}{pl}{i}{s}{}}{v.}
        \begin{description}[labelwidth=*]
            \item[] to move \textit{smth.} to a place
        \end{description}
    \newentry{\famword{a}{pl}{i}{s}{u}}{n.}
        \begin{description}[labelwidth=*]
            \item[] generic container, bin, box
        \end{description}
    \newentry{\famword{i}{pl}{aa}{s}{}}{v.}
        \begin{description}[labelwidth=*]
            \item[] to be in a place
        \end{description}
    \newentry{\famword{}{pl}{ii}{s}{}}{v.}
        \begin{description}[labelwidth=*]
            \item[] to move oneself to a place
        \end{description}
\end{description}

\section{\textsc{q---h}}
\begin{description}[leftmargin=*]
    \newentry{\famword{}{q}{asi}{h}{}}{n.}
        \begin{description}[labelwidth=*]
            \item[] snake
        \end{description}
    \newentry{\famword{}{q}{u}{h}{u}}{n.}
        \begin{description}[labelwidth=*]
            \item[] lizard, reptile
        \end{description}
\end{description}

\section{\textsc{s---j}}
\begin{description}[leftmargin=*]
    \newentry{\famword{i}{s}{aa}{j}{}}{v.}
        \begin{description}[labelwidth=*]
            \item[] to sleep
        \end{description}
    \newentry{\famword{}{s}{a}{j}{}}{n.}
        \begin{description}[labelwidth=*]
            \item[] sleep, the state of being asleep
        \end{description}
    \newentry{\famword{}{s}{a}{j}{a}}{adj.}
        \begin{description}[labelwidth=*]
            \item[] sleepy, tired, exhausted
        \end{description}
    \newentry{\famword{}{s}{a}{j}{}anara}{int.}
        \begin{description}[labelwidth=*]
            \item may you sleep well, have a good night\\
            \textit{the gender affix -an may be substituted with a more apt affix.}
        \end{description}
    \newentry{\famword{}{s}{uli}{j}{}}{n.}
        \begin{description}[labelwidth=*]
            \item[] the human back
        \end{description}
\end{description}

\section{\textsc{sl---s}}
\begin{description}[leftmargin=*]
    \newentry{\famword{aa}{sl}{i}{s}{}}{v.}
        \begin{description}[labelwidth=*]
            \item[] to shut \textit{smth.} inside a container or behind a door
        \end{description}
    \newentry{\famword{i}{sl}{aa}{s}{}}{v.}
        \begin{description}[labelwidth=*]
            \item[] to be closed, to be locked
        \end{description}
    \newentry{\famword{}{sl}{ii}{s}{}}{v.}
        \begin{description}[labelwidth=*]
            \item[] to close, to shut, to lock
        \end{description}
\end{description}

\end{multicols*}

\chapter{Rootless Words}
\begin{multicols*}{2}
\section{Auxiliary Verbs}
\begin{description}[leftmargin=*]
    \newentry{dak}{aux.}
    \begin{description}[labelwidth=*]
        \item[] can, to be able to, to be allowed to
    \end{description}
    \newentry{hwii}{aux.}
    \begin{description}[labelwidth=*]
        \item[] not, no, don't, never
    \end{description}
    \newentry{jaa}{aux.}
    \begin{description}[labelwidth=*]
        \item[] already, previously, by now
    \end{description}
    \newentry{kaj}{aux.}
    \begin{description}[labelwidth=*]
        \item[] to want to, to be going to 
    \end{description}
    \newentry{lit}{aux.}
    \begin{description}[labelwidth=*]
        \item[] to very
    \end{description}
    \newentry{naw}{aux.}
    \begin{description}[labelwidth=*]
        \item[] to stop
    \end{description}
    \newentry{tuuq}{aux.}
    \begin{description}[labelwidth=*]
        \item[] must, to have to
    \end{description}
\end{description}




\section{Postpositions}

\section{Pronouns}

\section{Numbers}

\section{Attitudinals}

\end{multicols*}

\part{Example Texts \& Translations}

\setsecnumdepth{subsubsection}
\settocdepth{subsubsection}

\end{document}