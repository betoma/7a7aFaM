\documentclass[a4paper,10pt,twoside,openright]{memoir}
\usepackage{multicol, multirow, array}
\usepackage{fontspec}
\usepackage{anyfontsize}
\usepackage[pagecolor=none,dvipsnames]{xcolor}
\usepackage{ragged2e}
\usepackage{amsmath}
\usepackage{amssymb}
\usepackage[hidelinks]{hyperref}
\usepackage{url}
\usepackage[margin=0.8in]{geometry}
\usepackage{float, hhline}
\usepackage{booktabs}
\usepackage{textcomp}
\usepackage{expex}
\usepackage{enumitem}
\usepackage[calc,english]{datetime2}

%-----CONFIGURATION------
%------------------------

\setmainfont{Charis SIL}[CharacterVariant=43:1]
\restylefloat{table}

\setsecnumdepth{subsubsection}
\settocdepth{subsubsection}

\lingset{glstyle=nlevel,numoffset=3em,textoffset=1.5em,exskip=.75ex,belowglpreambleskip=.25ex,aboveglftskip=.25ex}

\DTMnewdatestyle{eurodate}{%
    \renewcommand{\DTMdisplaydate}[4]{%
        \number##3.\nobreakspace%           day
        \DTMmonthname{##2}\nobreakspace%    month
        \number##1%                         year
    }%
    \renewcommand{\DTMDisplaydate}{\DTMdisplaydate}%
}

\DTMsetdatestyle{eurodate}

\renewcommand{\arraystretch}{1.4}

%-------COMMANDS---------
%------------------------

\newcommand{\lang}{ɁaɁa\textsc{f}a\textsc{m}}
\newcommand{\longv}{ː}
\newcommand{\sqbrack}[1]{$\langle$#1$\rangle$}
\newcommand{\phipa}[1]{/#1/}
\newcommand{\bripa}[1]{[#1]}
\newcommand{\ttilde}{\raise.17ex\hbox{$\scriptstyle\sim$}}
\newcommand{\rootpart}{$\Theta$}
\newcommand{\glotstop}{ʔ}
\newcommand{\bigglot}{Ɂ}
\newcommand{\lilglot}{ɂ}
\newcommand{\nm}{\symbol{"2205}}
\newcommand{\tiebar}{͡}
\newcommand{\famword}[5]{#1\textsc{#2}#3\textsc{#4}#5}

%-------TITLE PAGE-------
%------------------------

\title{{\fontsize{80}{80}\selectfont \lang} \\ \Huge \sffamily A Reference Grammar}
\author{Bethany E. Toma, Knut F. K. Ulstrup}
\date{\today}

%--------MAIN DOC--------
%------------------------

\begin{document}

\pagecolor{Melon}
\maketitle
\pagecolor{white}

\frontmatter

\chapter{Foreword}

\lang{} is a constructed language.

\newpage

\tableofcontents

\mainmatter

\part{Grammar}

\chapter{Phonology}
\section{Consonants}

\begin{table}[ht]
    \centering
    \begin{tabular}{rcccccc}
    \toprule
            & Labial & Alveolar & Palatal & Velar & Uvular & Glottal \\
    \midrule
    Fortis & pʰ \ttilde{} p\tiebar ɸ & tʰ \ttilde{} t\tiebar s &
    % \multirow{2}{*}{c \ttilde{} c\tiebar ç} & \multirow{2}{*}{k \ttilde{} k\tiebar x} & \multirow{2}{*}{q \ttilde{} q\tiebar χ} & \multirow{2}{*}{ʔ} \\
    c \ttilde{} c\tiebar ç & k \ttilde{} k\tiebar x & q \ttilde{} q\tiebar χ & \multirow{2}{*}{ʔ} \\
     Lenis & p \ttilde{} b & t \ttilde{} d & & & & \\
    Fricative & ɸ & s & \multicolumn{4}{c}{ç \enspace \ttilde{} \enspace x \enspace \ttilde{} \enspace χ \enspace \ttilde{} \enspace ħ \enspace \ttilde{} \enspace h} \\
    Approximant & & l & j & w & & \\
    Nasal & m & n & & & & \\
    Rhotic & & \multicolumn{4}{c}{ɾ \enspace \ttilde{} \enspace r \enspace \ttilde{} \enspace ɹ \enspace \ttilde{} \enspace ɽ \enspace \ttilde{} \enspace ɻ \enspace \ttilde{} \enspace ʀ \enspace \ttilde{} \enspace ʁ } & \\
    \bottomrule
    \end{tabular}
    \caption{Phonemic Consonant Inventory}
    \label{tab:consinv}
\end{table}

\section{Vowels}

\begin{table}[ht]
    \centering
    \begin{tabular}{rccc}
    \toprule
          & Front & Central & Back \\
    \midrule
    Close & i (i\longv{}) & & u (u\longv) \\
    Mid   & & ə & \\
    Open  & & a (a\longv) & \\
    \bottomrule
    \end{tabular}
    \caption{Phonemic Vowel Inventory}
    \label{tab:vowelinv}
\end{table}

\subsection{Epenthetic schwa}

\section{Morphophonemics}

\section{Orthography}

\lang{} has two recognized orthographic conventions, both based on the Latin alphabet. Both conventions use marked letterforms to indicate which part of a word are part of the underlying root and which are grammatical markers. The precise manner in which they're marked is the major point of difference between the two orthographic styles.

By and large, both orthographic conventions attempt to use the most intuitive representation of a given phoneme. There are very few differences between the conventions. Fortis and lenis stops are written using the typical voiceless and voiced symbols, respectively, in both systems. The labial fricative is written as \sqbrack{f} and the dorsal fricative as \sqbrack{h}. The palatal approximant is written using \sqbrack{j}, and the rhotic is, of course, written as \sqbrack{r}. The other phonemes are written with their usual IPA characters in both conventions, except for \phipa{\glotstop}, which is dealt with differently depending on which convention one is using.

\subsection{Formal writing style}

The formal writing conventions make use of small-caps letterforms to highlight roots. In addition, it uses the glottal stop character to indicate the glottal stop phoneme, using the capital glottal stop character \sqbrack{\bigglot} when the glottal stop is part of a root radical (for instance, in the word \textit{\bigglot a\bigglot a}) and the lowercase glottal stop character \sqbrack{\lilglot} otherwise (such as in the suffix \textit{-(e)\lilglot}).

\subsection{Informal writing style}

The informal writing conventions, also known as ``texting script", is the orthography used in the majority of day-to-day communication. Rather than using small-caps letterforms, it uses true capital letters for roots. It also uses \sqbrack{7} for the glottal stop, with no difference between capital and lowercase. While these differences could be considered less aesthetically pleasing, they result in an ASCII-compatible script, which makes this writing style far easier to use in most messaging apps and computer interfaces. Texting-style \lang{} also allows for several shorthand abbreviations that tend not to be used in more formal style.

\chapter{Morphology}
\section{Underlying roots}
\section{Derivational morphology}

\lang{} allows for words to be altered syntactically and semantically using a rich set of morphological operations, divided into two categories based on their concatenation. 

\subsection{Primary derivation}

Primary derivation refers to the non-concatenative morphology of stems. These operations are for the most part not productive, and not all roots have a corresponding stem with each of these patterns. They may not stack, i.e. a stem may only be inflected by one pattern at a time.

\begin{table}[ht]
    \centering
    \begin{tabular}{llll}
    \textit{Pattern} & \textit{Meaning} & \textit{Example} & \\
    \multirow{2}{*}{{\rootpart}a{\rootpart}} & \multirow{2}{*}{Abstract noun}& \famword{}{s}{a}{j}{} & sleep \emph{(cf. \famword{i}{s}{aa}{j}{} `to sleep')}\\
    & & \famword{}{k}{a}{l}{} & rainfall \emph{(cf. \famword{}{k}{ur}{l}{i} `raindrop')}\\
    {\rootpart}ana{\rootpart} & Person of X, Agentive noun & \famword{}{k}{ana}{j}{} & author \emph{(cf. \textsc{k}ii\textsc{j} `to write X')}\\
    {\rootpart}ar{\rootpart}i & Liquid noun & \famword{}{q}{ar}{f}{i} & coffee \emph{(cf. i\textsc{q}aa\textsc{f} `to drink coffee')} \\
    {\rootpart}ur{\rootpart}i & Object noun & \famword{}{n}{ur}{m}{i} & food \emph{(cf. \textsc{n}ii\textsc{m} `to eat X')}\\
    {\rootpart}idi{\rootpart} & Loose granular mass & \famword{}{w}{idi}{w}{} & sugar \emph{(cf. \textsc{w}a\textsc{w}a `sweet')} \\
    {\rootpart}asi{\rootpart} & Long slender object & \famword{}{b}{asi}{t}{} & hair \emph{(cf. \textsc{b}uli\textsc{t} `head')} \\
    {\rootpart}uli{\rootpart} & Associated body part & \famword{}{b}{uli}{t}{} & head \emph{(cf. i\textsc{b}aa\textsc{t} `to understand')}\\
    m{\rootpart}i{\rootpart} & Instrument, tool & \famword{m}{r}{i}{w}{} & weapon \emph{(cf. \textsc{r}a\textsc{q} `pain')} \\
    i{\rootpart}u{\rootpart}a & Place of X/with X attribute & \famword{i}{h}{u}{j}{a} & night \emph{(cf. \textsc{h}a\textsc{t}a `dark')} \\
    {\rootpart}ii{\rootpart} & Transitive verb & \famword{}{f}{ii}{s}{} & to give birth to \emph{(cf. \textsc{f}ana\textsc{s} `person')} \\
    i{\rootpart}aa{\rootpart} & Intransitive verb & \famword{i}{\bigglot}{aa}{\bigglot}{} & to act stupidly \emph{(cf. e\bigglot a\bigglot a `dumb')} \\
    {\rootpart}a{\rootpart}a & Primary attribute & \famword{}{s}{a}{fr}{a} & hot \emph{(cf. \textsc{s}a\textsc{f}e\textsc{r} `heat')} \\
    {\rootpart}u{\rootpart}u & Animal & \famword{}{b}{u}{rk}{u} & dog \emph{(cf. \textsc{b}a\textsc{rk} `bark')} \\
    {\rootpart}uu{\rootpart} & Country & \famword{}{f}{uu}{ns}{} & France \emph{(cf. \textsc{f}u\textsc{ns}u `frog')} \\
    {\rootpart}aju{\rootpart}a & Flat plane, surface & \famword{}{k}{aju}{l}{a} & Water surface \emph{(cf. \textsc{k}ar\textsc{l}i `water')}
    \end{tabular}
    \caption{Primary derivation patterns}
    \label{tab:primedevs}
\end{table}

\subsection{Secondary derivation}

Secondary derivation refers to the exclusively suffixing operations that may be applied to stems in addition to primary derivation. Unlike primary derivation, these suffixes may be stacked freely. 

\subsection{Gender}

Certain lexical items may be inflected to convey the gender of its referent. On certain words, namely \emph{-ara} greetings, gender marking is obligatory.

\begin{table}[ht]
    \centering
    \begin{tabular}{>{\em}ll}
    -un & Feminine gender \\
    -aj & Masculine gender \\
    -uj & Explicitly non-binary \\
    -an & Gender-neutral, agender \\
    \end{tabular}
\end{table}



\section{Inflectional morphology}

\subsection{Verb finals}

\subsection{Evidential modality}

\section{Pronouns and determiners}

\begin{table}[ht]
    \centering
    \begin{tabular}{rll}
        & \textit{Nonplural} & \textit{Plural} \\
    \textit{Speaker-only} & nas & naswi \\
    \textit{Addressee-only} & mi & miwi \\
    \textit{Inclusive} & nemi & nemiwi \\
    \end{tabular}
    \caption{Discourse participant pronouns}
    \label{tab:firstandsecond}
\end{table}

\begin{table}[ht]
    \centering
    \begin{tabular}{>{\em}rll}
        & \textit{Determiner} & \textit{Pronoun}  \\
    Proximal & wa & wase \\
    Medial & par & parse \\
    Distal & bu & buse \\
    Interrogative & li & lise \\
    Relative & kun & kunse 
    \end{tabular}
    \caption{Determiners and demonstrative pronouns}
    \label{tab:determiners}
\end{table}

\chapter{Syntax}

\section{Verb stacking}

\section{Subordinate clauses}

Full verb phrases may be nominalized and act as an argument of another predicate.

\subsection{Relative clauses}

Relative clauses are a type of subordinate clauses that describes a referent's states or actions. They are internally headed, always verb-final, and the relative determiner \emph{kun} is used to mark the head of the clause, i.e. the thing that is being described.

\ex
\begingl
\famword{}{f}{ana}{s}{}[person]
\famword{i}{l}{aa}{s}{}[walk]
-tu[\textsc{rel}]
\famword{}{s}{a}{j}{a}uru[sleepy:\textsc{cop}]
\glft `The person who walked home was sleepy.'
\endgl
\xe

Clauses with a single argument do not require that the head is marked, as the argument is assumed to be the head by default. Still, the verb itself can be marked

\ex
\begingl
\famword{in}{f}{i}{m}{}[children]
kun[\textsc{rel}]
\famword{i}{m}{aa}{w}{}[play]
-tu[\textsc{rel}]
naswi[\textsc{1p.ex}]
\famword{}{d}{ii}{l}{}[look]
\glft `We watched the playtime that the children were having'
\endgl
\xe

In high-valency clauses, \emph{kun} becomes more pertinent. 

\pex[interpartskip=3ex]
\a
\begingl
kun[\textsc{rel}]
\famword{}{f}{ana}{s}{}[person]
\famword{i}{f}{u}{s}{a}[house]
daw[to]
fit[in]
\famword{i}{l}{aa}{s}{}tu[walk\textsc{:rel}]
nas[\textsc{1s}]
\famword{}{f}{ii}{l}{}[see]
\glft `I saw the person who walked into the house.'
\endgl
\a
\begingl
\famword{}{f}{ana}{s}{}[person]
kun[\textsc{rel}]
\famword{i}{f}{u}{s}{a}[house]
daw[to]
fit[in]
\famword{i}{l}{aa}{s}{}tu[walk\textsc{:rel}]
nas[\textsc{1s}]
\famword{}{f}{ii}{l}{}[see]
\glft `I saw the house that the person walked into.'
\endgl
\a
\begingl
\famword{}{f}{ana}{s}{}[person]
\famword{i}{f}{u}{s}{a}[house]
daw[to]
fit[in]
kun[\textsc{rel}]
\famword{i}{l}{aa}{s}{}tu[walk\textsc{:rel}]
nas[\textsc{1s}]
\famword{}{f}{ii}{l}{}[see]
\glft `I saw how the person walked into the house.'
\endgl
\xe


\section{Comparative constructions}

from-comparative, marks standard (to which is compared)

\pex[interpartskip=3ex]
\a
\begingl
\famword{}{p}{u}{m}{u}[rabbit]
\famword{}{f}{ana}{s}{}[person]
fun[from]
\famword{}{m}{a}{nt}{a}[big]
-uru[\textsc{cop}]
\glft `The rabbit was bigger than a person.'
\endgl
\a
\begingl
\famword{}{t}{a}{n}{}[\textsc{top}/time]
nemi[\textsc{qual}/\textsc{du.in}]
buse[\textsc{std}/\textsc{dist:pn}]
fun[\textsc{mrk}/from]
\famword{}{j}{a}{l}{}[/many\_things]
-ila[/have]
\glft `We have more time than them.'
\endgl
\xe

\section{Animacy hierarchy}

\begin{table}[ht]
    \centering
    \begin{tabular}{ll}
    0 & Natural Forces \\
    1 & Pronouns (1>2>3) \\
    2 & Speakers of \lang{} \\
    3 & Non-speakers of \lang{} \\
    4 & Higher-order animals (mammals, octopus, intelligent creatures) \\
    5 & \parbox[t]{7cm}{Body parts, tools, any inanimate object used for acting upon something} \\
    6 & Lower-order animals \\
    7 & Plants \\
    8 & Inanimate objects \\
    9 & Abstract concepts 
    \end{tabular}
    \caption{Animacy hierarchy in nominals}
    \label{tab:hierarchy}
\end{table}

\chapter{Semantics and pragmatics}
\section{Phatic expressions}

Phatic expressions in \lang{} are all in some way related to the nouns they are derived from, suggesting an emphasis on acknowledging the addressee's current or upcoming actions. The addressee may respond with the same expression back, even if it does not apply to the original speaker in any way, or respond in kind with a more suitable expression.

The obligatory gender marking is a means of expressing your gender identity in an unintrusive manner.\footnote{The real reason is that as Beth once ended a conversation with "sayonara", Knut noticed some coincidental similarities with the word \famword{}{s}{a}{j}{} `sleep' and the affix -un to indicate feminine gender, with the -ara reanalyzed as a phatic/optative marker of sorts.}

\subparagraph{\famword{}{f}{a}{s}{anara}} \textit{(from \famword{}{f}{a}{s}{} `life')} is a catch-all greeting, suitable for any time of day. 

\subparagraph{\famword{}{s}{a}{j}{anara}} \textit{(from \famword{}{s}{a}{j}{} `sleep')} is similar in use to "good night", but is only used if the person is going to bed, not just leaving for the night.

\section{Idiomatic expressions}

\famword{}{c}{u}{mp}{u} \famword{}{c}{u}{mp}{uuru} = no shit, preaching to the choir

\part{Dictionary}

\newcommand{\newentry}[2]{\item[#1] $\bullet$ \textit{#2}\hfill}

\setsecnumdepth{part}
\settocdepth{section}

\chapter{Roots and Derived Words}
\begin{multicols*}{2}
\section{\textsc{b---t}}
\begin{description}[leftmargin=*]
    \newentry{\textsc{b}ii\textsc{t}}{v.tr.}
    \begin{enumerate}
        \item to know \textit{smth.}, to understand \textit{smth.}
        \begin{quote}
            mi i\textsc{b}aa\textsc{rb}e\lilglot{} kajuc nas \textsc{b}ii\textsc{t}\\
            \textit{`I know that you want to leave.'}
        \end{quote}
        \item to love \textit{sme.} like a brother, to have a close platonic bond with \textit{sme.}, to be best friends with \textit{sme.}
        \begin{quote}
            nas wan \textsc{J}ana\textsc{B} \textsc{b}ii\textsc{t}ibi\\
            \textit{`I love my friends.'}
        \end{quote}
        \textit{(NB: the subject is reversed from its use as `to understand': \emph{mi nas \textsc{b}ii\textsc{t}ibi} means `you understand me' but `I love you'.)}
    \end{enumerate}
    \newentry{\famword{i}{b}{aa}{t}{}}{v.intr.}
    \begin{enumerate}
        \item to know, to understand, to be in a state of knowing or understanding what is going on
        \item \textit{(when used reciprocally)} to love each other, to have a close platonic bond, to be the best of friends
        \begin{quote}
            nemi i\textsc{b}aa\textsc{t}ami\\
            \textit{`The two of us are thick as thieves.'}
        \end{quote}
    \end{enumerate}
\end{description}

\section{\textsc{k---l}}
\begin{description}[leftmargin=*]
    \newentry{\famword{a}{k}{i}{l}{u}}{n.}
    \begin{description}[labelwidth=*]
        \item[] container of water to be drunk from, glass, cup, water bottle
        \begin{quote}
            \textsc{mark} se \famword{}{k}{a}{j}{} men \famword{a}{k}{i}{l}{uila}\\
            \textit{`Mark owned five water bottles.'}
        \end{quote}
    \end{description}
    \newentry{\famword{i}{k}{aa}{l}{}}{v.intr.}
    \begin{enumerate}
        \item \textit{(impersonal)} to be a rainy day
        \begin{quote}
            \famword{waj}{h}{u}{j}{a} \famword{i}{k}{aa}{l}{}\\
            \textit{`Today's a rainy day.'}
        \end{quote}
        \item \textit{(impersonal)} to be raining
        \begin{quote}
            \famword{i}{m}{u}{nt}{a} daw nas \famword{i}{j}{aa}{t}{e\lilglot} kaj da buse fit \famword{i}{k}{aa}{l}{}\\
            \textit{`I wanted to go to the mountains, but it's raining there.'}
        \end{quote}
    \end{enumerate}
    \newentry{\famword{}{k}{aju}{l}{a}}{n.}
    \begin{description}[labelwidth=*]
        \item[] water surface
        \begin{quote}
            naswi \famword{}{k}{aju}{l}{a} tui \famword{}{f}{ii}{l}{ami\lilglot} dak\\
            \textit{`We could see ourselves on the water's surface.'}
        \end{quote}
    \end{description}
    \newentry{\famword{}{k}{a}{l}{}}{n.}
    \begin{description}[labelwidth=*]
        \item[] humidity, wetness, dampness
        \begin{quote}
            nas wan \famword{i}{f}{u}{s}{a} \famword{}{k}{a}{l}{ila}\\
            \textit{`My house is damp.'}
        \end{quote}
    \end{description}
    \newentry{\famword{}{k}{a}{l}{a}}{adj.} 
    \begin{enumerate}
        \item covered in water, saturated with water, wet, soaked
        \item fluid, liquid
    \end{enumerate}
    \newentry{\famword{}{k}{ar}{l}{i}}{n.}
    \begin{description}[labelwidth=*]
        \item[] water served as a beverage, water not part of a body of water or stream
        \begin{quote}
            mi \famword{}{n}{ar}{k}{a} \famword{}{k}{ar}{l}{iilali}?\\
            \textit{`Do you have any cold water?'}
        \end{quote}
    \end{description}
    \newentry{\famword{}{k}{asi}{l}{}}{n.}
        \begin{enumerate}
            \item river, stream
            \item stream or sprinkle of water
        \end{enumerate}
\end{description}
\end{multicols*}

\chapter{Rootless Words}
\begin{multicols*}{2}
\section{Auxiliary Verbs}
\begin{description}[leftmargin=*]
    \newentry{hwii}{aux.}
    \begin{description}[labelwidth=*]
    \item[] not, no, don't, never
    \end{description}
    \newentry{kaj}{aux.}
    \begin{description}[labelwidth=*]
    \item[] to want to, to be going to 
    \end{description}
\end{description}
\end{multicols*}

\part{Example Texts \& Translations}

\setsecnumdepth{subsubsection}
\settocdepth{subsubsection}

\end{document}