\documentclass[a4paper,10pt,twoside,openright]{memoir}
\usepackage{multicol, multirow, array}
\usepackage{fontspec}
\usepackage{anyfontsize}
\usepackage[pagecolor=none,dvipsnames]{xcolor}
\usepackage{ragged2e}
\usepackage{amsmath}
\usepackage{amssymb}
\usepackage[hidelinks]{hyperref}
\usepackage{url}
\usepackage[margin=0.8in]{geometry}
\usepackage{float, hhline}
\usepackage{booktabs}
\usepackage{textcomp}
\usepackage{expex}
\usepackage{enumitem}
\usepackage[calc,english]{datetime2}
\usepackage{suffix}
\usepackage{afterpage}
\usepackage{phonrule}

%-----CONFIGURATION------
%------------------------

\setmainfont{Charis SIL}
\restylefloat{table}

\setsecnumdepth{subsubsection}
\settocdepth{subsubsection}

\lingset{glstyle=nlevel,numoffset=3em,textoffset=1.5em,exskip=.75ex,belowglpreambleskip=.25ex,aboveglftskip=.25ex,everyglft=\it}

\DTMnewdatestyle{eurodate}{%
    \renewcommand{\DTMdisplaydate}[4]{%
        \number##3.\nobreakspace%           day
        \DTMmonthname{##2}\nobreakspace%    month
        \number##1%                         year
    }%
    \renewcommand{\DTMDisplaydate}{\DTMdisplaydate}%
}

\DTMsetdatestyle{eurodate}

\renewcommand{\arraystretch}{1.4}

%-----description command---%
%---------------------------%

\SetLabelAlign{parrightcent}{\strut\smash{\parbox[c]{\labelwidth}{\raggedleft#1}}}
\SetLabelAlign{parright}{\strut\smash{\parbox[t]{\labelwidth}{\raggedleft#1}}}

%-------COMMANDS---------
%------------------------

\newcommand{\lang}{{\bigglot}a{\bigglot}a-\textsc{f}a\textsc{m}}
\newcommand{\longv}{ː}
\newcommand{\sqbrack}[1]{$\langle$#1$\rangle$}
\newcommand{\phipa}[1]{/#1/}
\newcommand{\bripa}[1]{[#1]}
\newcommand{\ttilde}{\raise.17ex\hbox{$\scriptstyle\sim$}}
\newcommand{\rootpart}{$\Theta$}
\newcommand{\glotstop}{ʔ}
\newcommand{\bigglot}{Ɂ}
\newcommand{\lilglot}{ɂ}
\newcommand{\nm}{\symbol{"2205}}
\newcommand{\tiebar}{͡}
\newcommand{\famwordold}[5]{#1\textsc{#2}#3\textsc{#4}#5}
\newcommand\famword[1]{{\addfontfeatures{Letters=UppercaseSmallCaps}#1}}
\newcommand{\famq}[1]{»#1«}

%----QoL COMMAND-------
%----------------------

\newcounter{numbertable}

\newcommand{\wtf}[1]{\thenumbertable. & #1 \\\refstepcounter{numbertable} }

%\newcommand{numbertablerow}[1]{%
%    \refstepcounter{numbertable}
%    \thenumbertable. & #1 }

%-----DICT COMMANDS------
%------------------------

\makeatletter
\@beginparpenalty=10000
\makeatother

\newcounter{dictwordcount}
\newcounter{definition}

\newenvironment{dictroot}[2]%
    {%
    \subsection{\uppercase{#1---#2}}
    \begin{description}[leftmargin=*]
    }{%
    \end{description}
    }%

\newcommand{\dictsubtitle}[1]{%
    \end{description}
    \subsubsection*{#1}
    \begin{description}[leftmargin=*]
}%

\newenvironment{dictentry}[2]%
    {%
    \item[\famword{#1}] $\bullet$ \textit{#2}\hfill
    \setcounter{definition}{0}%
    \refstepcounter{dictwordcount}%
    \begin{description}[align=right,labelwidth=*,font=\normalfont]
    }{%
    \end{description}
    }%

\newcommand{\dictdef}[1]{\refstepcounter{definition}%
\item[\thedefinition.] #1
}%

\WithSuffix\newcommand\dictdef*[1]{%
    \item[] #1
}

\newcommand{\newentry}[2]{%
\item[#1] $\bullet$ \textit{#2}\hfill
}%

%-------TITLE PAGE-------
%------------------------

\title{{\fontsize{100}{100}\selectfont \lang} \\ \Huge \sffamily A Reference Grammar of the Narish Language}
\author{Bethany E. Toma, Knut F. K. Ulstrup}
\date{\today}

%--------MAIN DOC--------
%------------------------

\begin{document}

\pagecolor{Melon}
\maketitle
\pagecolor{white}

\frontmatter

\chapter{Foreword}

\lang{} is a constructed language spoken on the fictitious Nareland island.

\newpage

\tableofcontents

\mainmatter

\part{Grammar}

\chapter{Phonology}
\section{Consonants}

\begin{table}[ht]
    \centering
    \begin{tabular}{rcccccc}
    \toprule
            & Labial & Alveolar & Palatal & Velar & Uvular & Glottal \\
    \midrule
    Fortis & pʰ \ttilde{} p\tiebar ɸ & tʰ \ttilde{} t\tiebar s &
    % \multirow{2}{*}{c \ttilde{} c\tiebar ç} & \multirow{2}{*}{k \ttilde{} k\tiebar x} & \multirow{2}{*}{q \ttilde{} q\tiebar χ} & \multirow{2}{*}{ʔ} \\
    c \ttilde{} c\tiebar ç & k \ttilde{} k\tiebar x & q \ttilde{} q\tiebar χ & \multirow{2}{*}{ʔ} \\
     Lenis & p \ttilde{} b & t \ttilde{} d & & & & \\
    Fricative & f & s & \multicolumn{4}{c}{ç \enspace \ttilde{} \enspace x \enspace \ttilde{} \enspace χ \enspace \ttilde{} \enspace ħ \enspace \ttilde{} \enspace h} \\
    Approximant & & l & j & w & & \\
    Nasal & m & n & & & & \\
    Rhotic & & \multicolumn{2}{c}{ɾ \enspace \ttilde{} \enspace ɹ \enspace \ttilde{} \enspace ɻ } & & & \\
    \bottomrule
    \end{tabular}
    \caption{Phonemic Consonant Inventory}
    \label{tab:consinv}
\end{table}

\section{Vowels}

\begin{table}[ht]
    \centering
    \begin{tabular}{rccc}
    \toprule
          & Front & Central & Back \\
    \midrule
    Close & i (i\longv{}) & & u (u\longv) \\
    Mid   & & ə & \\
    Open  & & a (a\longv) & \\
    \bottomrule
    \end{tabular}
    \caption{Phonemic Vowel Inventory}
    \label{tab:vowelinv}
\end{table}

\section{Phonotactics \& Allophony}

Syllables always contain a vowel nucleus, with rarely more than two onset consonants and two coda consonants. Sonority hierarchy plays a big role in the structure of syllables and their realization.

\subsection{Sonority Hierarchy}



\subsection{High vowel lowering}

The high vowels /i u/ are lowered to [e o] before /j w/ respectively. 

\phonr{i(\longv)}{e(\longv)}{j}

\phonr{u(\longv)}{o(\longv)}{w}

\subsection{Epenthetic schwa}

The epenthetic schwa appears within consonant sequences that, without adjacent vowels, violate syllable structure and may not be realized. It occurs in the earliest non-initial position possible that provides legal consonant sequences. This schwa is never stressed. For example, the word \famword{NurRKi} is phonemically /ˈnurrkʰi/, but \lang{} does not allow duplicate continuant phonemes, leaving the second /r/ out of the first syllable. The following syllable would not be able to accommodate it either, as it would violate the sonority rules by placing a less sonorous phoneme between two more sonorous ones. Of the two positions for the epenthetic schwa within this sequence, [ˈnuɾəɾkʰi] and [ˈnuɾɾəkʰi], only the former solution accommodates both restrictions and yields a legal realization.

\begin{table}[ht]
    \begin{tabular}{lll}
        \textit{Example}    & \textit{Phonemic transcription}   & \textit{Phonetic Realization} \\
        \famword{NurRKi}    & /ˈnurrkʰi/                        & [ˈnuɾəɾkʰi] \\
        \famword{mBiT}      & /ˈmpitʰ/                          & [məˈbitʰ] \\
        \famword{inNiM}     & /inˈnim/                          & [inəˈnim] \\
        \famword{SaFR}      & /ˈsafr/                           & [ˈsafər] 
    \end{tabular}
\end{table}

Word-onset sequences may never be disambiguated by prefixing an epenthetic schwa to the word. Like in medial and coda sequences, the epenthetic schwa must occur interconsonantally, but not across word boundaries. In the sequence |...C$_a$\#C$_x$C$_y$V...|, where C$_a$ is a consonant in the preceding word and C$_x$C$_y$ is an illegal sequence, the epenthetic schwa may not attempt to separate C$_a$ and C$_x$ in any way, and must instead separate C$_x$ and C$_y$. Word boundaries are inherently segmenting and preclude other segmenting elements like the epenthetic schwa. However, compound words are considered to be one phonological word, and in lacking a word boundary must make use of the epenthetic schwa.

\section{Prosody}

Stress, in the form of elevated pitch and volume, is placed on the first non-schwa vowel of the word, after the first root radical, on a long vowel immediately preceding the first radical, or on certain morphemes that carry stress.

\begin{table}[ht]
    %\centering
    \begin{tabular}{lll}
        nemiwi & [nəˈmiwi] & first non-schwa vowel of word\\
        parse & [ˈparsə] & first non-schwa vowel of word\\
        \famword{iFaaM} & [iˈfa{\longv}m] & vowel after first radical\\
        \famword{FanaS} & [ˈfanas] & vowel after first radical\\
        \famword{aaNiW}& [ˈa{\longv}niw] & long vowel preceding radical\\
        \famword{iLaaSak} & [iˌla{\longv}ˈsak] & presence of stress-carrying morpheme (imperative affix \emph{-ak})
    \end{tabular}
\end{table}


\section{Morphophonemics}

\section{Orthography}

\lang{} has two recognized orthographic conventions, both based on the Latin alphabet. Both conventions use marked letterforms to indicate which part of a word are part of the underlying root and which are grammatical markers. The precise manner in which they're marked is the major point of difference between the two orthographic styles.

By and large, both orthographic conventions attempt to use the most intuitive representation of a given phoneme. There are very few differences between the conventions. Fortis and lenis stops are written using the typical voiceless and voiced symbols, respectively, in both systems. The labial fricative is written as \sqbrack{f} and the dorsal fricative as \sqbrack{h}. The rhotic is, of course, written as \sqbrack{r}. The palatal approximant is written as \sqbrack{j}, except when adjacent to an \sqbrack{i} within the same word, in which case it is written as \sqbrack{y}. The other phonemes are written with their usual IPA characters in both conventions, except for \phipa{\glotstop}, which is dealt with differently depending on which convention one is using.

\subsection{Formal writing style}

The formal writing conventions make use of small-caps letterforms to highlight roots. In addition, it uses the glottal stop character to indicate the glottal stop phoneme, using the capital glottal stop character \sqbrack{\bigglot} when the glottal stop is part of a root radical (for instance, in the word \textit{\bigglot a\bigglot a}) and the lowercase glottal stop character \sqbrack{\lilglot} otherwise (such as in the suffix \textit{-(e)}\lilglot).

\subsection{Informal writing style}

The informal writing conventions, also known as ``texting script", is the orthography used in the majority of day-to-day communication. Rather than using small-caps letterforms, it uses true capital letters for roots. It also uses \sqbrack{7} for the glottal stop, with no difference between capital and lowercase. While these differences could be considered less aesthetically pleasing, they result in an ASCII-compatible script, which makes this writing style far easier to use in most messaging apps and computer interfaces. Texting-style \lang{} also allows for several shorthand abbreviations that tend not to be used in more formal style.

\chapter{Morphology}
\section{Underlying roots}

The majority of lexical items are produced by applying morphological operation belonging to their respective morphological categories to abstract roots. These roots take the form of two \emph{radicals,} where each radical constitutes a non-zero number of consonants. 

There are several fossilized derivations within the roots whose meanings have been lost, and as such form discrete roots altogether. Compare \famword{L-S} \emph{`travel by foot'} and \famword{PL-S} \emph{`travel, habitation'}, or \famword{N-W} \emph{`death, murder'} and \famword{N-WK} \emph{`sacrifice, martyrdom'}.

\section{Derivational morphology}

\lang{} allows for words to be altered syntactically and semantically using a rich set of morphological operations, divided into two categories based on their concatenation. 

\subsection{Primary derivation}

Primary derivation refers to the non-concatenative morphology of stems. These operations are for the most part not productive, and not all roots have a corresponding stem with each of these patterns. They may not stack, i.e. a stem may only be inflected by one pattern at a time.

\afterpage{%
\clearpage
\setcounter{numbertable}{0}
\begin{table}[p]
    \centering
    \begin{tabular}{@{}rllll@{}}
    & \textit{Pattern} & \textit{Meaning} & \textit{Example} & \\\refstepcounter{numbertable}
    \wtf{
        \rootpart{a}\rootpart & %
        Abstract noun & %
        \famword{JaB} & %
        good fortune \emph{(cf. \famword{JaBa} `good, fortunate')}
    }
    \wtf{
        \rootpart{ii}\rootpart & %
        Transitive verb & %
        \famword{FiiS} & %
        to give birth to \emph{(cf. \famword{FanaS} `person')}
    }
    \wtf{
        \rootpart{iya}\rootpart & %
        Unaccusative verb & %
        \famword{KiyaL} & %
        to be poured out \emph{(cf. \famword{KarLi} `water')}
    }
    \wtf{
        {i}\rootpart{aa}\rootpart & %
        Unergative verb & %
        \famword{iNaaM} & %
        to eat \emph{(cf. \famword{NiiM} `to eat (smth.)')}
    }
    \wtf{
        {\rootpart$_1$}i{\rootpart$_1$}iya{\rootpart$_2$} & %
        Causative of unaccusative & %
        \famword{KiKiyaL} & %
        to pour (smth.) out \emph{(cf. \famword{KiyaL} `to flow out')}
    }
    \wtf{
        aa\rootpart{i}\rootpart & %
        Causative of unergative & %
        \famword{aaNiM} & %
        to feed \textit{(cf. \famword{iNaaM} `to eat')}
    }
    \wtf{
        {\rootpart}{a}{\rootpart}{a} & %
        Attributive & %
        \famword{SaFRa} & %
        hot \emph{(cf. \famword{SaFR}`heat')}
    }
    \wtf{
        {\rootpart}{ana}{\rootpart}{} & %
        Person of X & %
        \famword{KanaJ}& %
        author \emph{(cf. \famword{KiiY} `to write (smth.)')}
    }
    \wtf{
        {\rootpart}{ur}{\rootpart}{i} & %
        Object & %
        \famword{NurMi}& %
        food \emph{(cf. \famword{iNaaM} `to eat')}
    }
    \wtf{
        {\rootpart}{ar}{\rootpart}{i} & %
        Liquid noun & %
        \famword{QarFi} & %
        coffee \emph{(cf. \famword{iQaaF} `to drink coffee')} 
    }
    \wtf{
        {i}\rootpart{u}\rootpart{a} & %
        Place of X & %
        \famword{iHuTa} & %
        night \emph{(cf. \famword{HaTa} `dark')}
    }
    \wtf{
        {m}\rootpart{i}\rootpart & %
        Tool/instrument & %
        \famword{mRiQ} & %
        weapon \emph{(cf. \famword{RaQ} `pain')}
    }
    \wtf{
        {in}\rootpart{i}\rootpart & %
        Diminutive & %
        \famword{inFiM} & %
        word \textit{(cf. \famword{FaM} `language')}
    }
    \wtf{
        {\rootpart}uli{\rootpart} & %
        Body part & %
        \famword{BuliT}& %
        head \emph{(cf. \famword{iBaaT} `to understand')}
    }
    \wtf{
        \rootpart{u}\rootpart{u} & %
        Animal & %
        \famword{CuMPu} & %
        kangaroo \emph{(cf. \famword{iCaaMP} `to jump')}
    }
    \wtf{
        \rootpart{asi}\rootpart & %
        Long, slender object & %
        \famword{NasiRK} & %
        icicle \emph{(cf. \famword{NuRKi} `snowball')}
    }
    \wtf{
        \rootpart{aju}\rootpart{a} & %
        Flat object or surface & %
        \famword{DajuLa} & %
        mirror \emph{(cf. \famword{DiiL} `to stare at')}
    }
    \wtf{
        \rootpart{idi}\rootpart & %
        Loose, granular mass & %
        \famword{WidiW}& %
        sugar \emph{(cf. \famword{WaWa} `sweet')}
    }
    \wtf{
        {a}\rootpart{i}\rootpart{u} & %
        Closed/natural container & %
        \famword{aBiRDu} & %
        bird's nest \emph{(cf. \famword{BuRDu} `bird')} 
    }
    \wtf{
        \rootpart{imi}\rootpart{u} & %
        Open/unnatural container & %
        \famword{QimiFu} & %
        coffee mug \emph{(cf. \famword{aQiFu} `coffee pot')}
    }
    \wtf{
        \rootpart{u}\rootpart{i} & %
        Color & %
        \famword{KuWi} & %
        green \emph{(cf., \famword{KajuWa} `leaf')}
    }
    \wtf{
        {u}\rootpart{i}\rootpart{i} & %
        Experiential & 
        \famword{uNiMi} & 
        hungry \emph{(cf. \famword{NaMa} `satisfying')}
    }
    \wtf{
        \rootpart{uu}\rootpart & %
        People group, land of X people & %
        \famword{NuuRK} & %
        Nords, Norse, Norway \emph{(cf. \famword{NaRKa} `cold')}
    }
    \wtf{
        \rootpart{a}\rootpart{ia} & %
        Nationstate & %
        \famword{FRaNCia} & %
        France \emph{(cf. \famword{FRuuNC} `Franks')}
    }
    \end{tabular}
    \caption{Primary derivation patterns}
    \label{tab:primedevs}
\end{table}
\clearpage
}

\subsection{Secondary derivation}

Secondary derivation refers to the exclusively suffixing operations that may be applied to stems in addition to primary derivation. Unlike primary derivation, these suffixes may be stacked freely. 

\subsubsection{\emph{-uru} - `to be'}

Rather than using a verbal copula, nominal and adjectival phrases are derived into verbs with the meaning `to be X' or `to have characteristic X' with the \emph{-uru} affix. 

\ex
%\begingl
%\glpreamble
ha \textsc{Knut} \famwordold{in}{f}{i}{s}{uru}.\\
%\endpreamble
%\glft
Knut is a baby.
%\endgl
\xe

Adjectives and determiners may still modify a noun that has been turned modified into part of the verb, resulting in these words serving a semi-adverbial function in these cases. 

\ex
\begingl
\glpreamble
ha \textsc{Knut} li \famwordold{}{b}{a}{c}{a} \famwordold{in}{f}{i}{s}{uruli}?
\endpreamble
ha[\sc ref.m]
\textsc{Knut}[Knut]
li[which]
\famwordold{}{b}{a}{c}{a}[small]
\famwordold{in}{f}{i}{s}{}[baby]@
-uru[\sc -cop]@
-li[\sc -int]
\glft
`Which little baby is Knut?'
\endgl
\xe

Most \lang{} speakers reduce \textit{-auru} and \textit{-iuru} to \textit{-aaru} and \textit{-iiru}, respectively. In western dialects, however, they are instead most often pronounced as \textit{-ore} \bripa{orə} and \textit{-üre} \bripa{ʉrə}. Western dialects also reduce \textit{-uru} to \textit{-ure} \bripa{urə} as well.

\subsubsection{\emph{-ila} - `there is', `to have'}

\subsubsection{\emph{-ara} - wishes and greetings}

\subsubsection{\emph{-iri} - `to make'}

\subsubsection{\emph{-ana} - person}

\subsubsection{\emph{-ini} - diminutive}

\subsubsection{\emph{-ari} - `to become', `to cause to be'}

\subsubsection{\emph{-inala} - `to make X-er, to increase'}

\subsubsection{\emph{-lat} -`measured in', `comprising'}

When describing a quantity of something, often one may desire to use a particular noun as a measure of another. While many languages (English included) turn the measured substance into an adpositional phrase modifying the measure word as the head, in \lang{} one turns the measure word into an adjective modifying the substance being measured using the suffix \emph{-lat}. For instance, \emph{\famwordold{}{k}{imi}{l}{u}} `glass' becomes \emph{\famwordold{}{k}{imi}{l}{ulat}} `a glass of' in \emph{\famwordold{}{k}{imi}{l}{ulat} \famwordold{}{m}{ar}{h}{i}} `glass of milk'

This suffix is not used with numerals whose prupose is to convey cardinal numbers, however, which can be prepended onto noun phrases like determiners without being adjectivalized.

\ex
\begingl
\glpreamble
pars bar \famwordold{a}{k}{i}{l}{ulat} \famwordold{}{k}{ar}{l}{i} \famwordold{}{k}{ii}{l}{}.
\endpreamble
pars[\sc 3.med]
bar[three]
\famwordold{a}{k}{i}{l}{u}[bottle]@
-lat[\sc -meas]
\famwordold{}{k}{ar}{l}{i}[water]
\famwordold{}{k}{ii}{l}{}[drink]
\glft `He drank three bottles of water.'
\endgl
\xe

Numerals modified by -lat convey "X at a time" or "in groups of X".

\ex
\begingl
\glpreamble
miwi naswi sarlat \famword{FiiMe\lilglot} dak.
\endpreamble
miwi[\sc 2.pl]
naswi[\sc 1.ex]
sar[one]@
-lat[\sc -meas]
\famword{FiiMe\lilglot}
dak[want]
\glft `We'd like to speak with you one at a time.'
\endgl
\xe

\ex
\begingl
\glpreamble
paslat \famword{iDaaMak!}
\endpreamble
pas[two]@
-lat[\sc -meas]
\famword{iDaaMak}[gather:\textsc{imp}]
\glft `team up in pairs!'
\endgl
\xe

\subsubsection{\emph{-aki} - `made/comprised of'}

When one thing is constructed from/comprised of a particular material, one can indicate this by taking the noun of the material in question and turning it into an adjective using the suffix \emph{-aki}---for instance, \textit{\famwordold{}{r}{u}{rkd}{i}} `wood' becomes \textit{\famwordold{}{r}{u}{rkd}{iaki}} `wooden'. 

\ex
\begingl
\glpreamble%
nas \famwordold{}{m}{aju}{w}{aaki} \famwordold{i}{f}{u}{s}{airi}%
\endpreamble%
nas[\sc 1sg]
\famwordold{}{m}{aju}{w}{a}[playing\_card]@
-aki[-made\_of]
\famwordold{i}{f}{u}{s}{a}[house]@
-iri[-make]
\glft `I'm building a house of cards.'
\endgl
\xe

\emph{-aki} is also used for things that come out of or are produced by a living thing. This contrasts with \emph{wan}, which relates more to ownership or association. Thus, if Magnus purchases a chicken egg from the grocery store, the egg is \textit{M\famword{AGNUS wan HuNuaki FurSi}}. This is also used to describe who birthed a human person, and thus is the source of \textit{-aki}'s use as a matronymic suffix---if Magnus's mother is named Lisa, he is \textit{\textsc{Lisaaki Magnus}}. Note that this does not extend to fatherhood, since a child does not come out of their father in the same way as their mother. An equivalent patronymic (less common in Narish culture) would thus have to use \textit{wan}.

\subsubsection{\emph{-s(e)} - `the ... one'}

Used to nominalize adjectives in contexts where the head is known or obvious.

\ex
\begingl
\glpreamble
A: mi li \famwordold{}{m}{asi}{h}{} \famwordold{}{n}{ii}{m}{e\lilglot} kajli? \textit{(`Which ice cream bar do you want to eat?')}\\
B: \famwordold{}{c}{a}{kl}{as} nas kaj!%
\endpreamble
\famwordold{}{c}{a}{kl}{a}[chocolate]@
-s[\sc -nmz]
nas[\sc 1sg]
kaj[want]
\glft `I want the chocolate one!'
\endgl
\xe

Unlike equivalent periphrastic constructions from other languages (such as English's `the chocolate one'), this derivation can only be applied to bare adjectives, and thus cannot be applied to relative clauses or strings of adjectives.

\subsection{Compounding}

\subsection{Gender}

Certain lexical items may be inflected to convey the gender of its referent. On certain words, namely \emph{-ara} greetings, gender marking is obligatory.

\begin{table}[ht]
    \centering
    \begin{tabular}{>{\em}ll}
    -un & Feminine gender \\
    -aj & Masculine gender \\
    -uj & Explicitly non-binary \\
    -an & Gender-neutral, agender \\
    \end{tabular}
\end{table}

\section{Inflectional morphology}

\subsection{Verb finals}

Verbs that are either not declarative, or not the head of the matrix clause, must be marked based on their purpose in the sentence. These verbs may appear in subordinate clauses, as converbs, serial verbs, or finite non-declarative head verbs.

\begin{table}[ht]
    \centering
    \begin{tabular}{>{\em}ll}
        -\nm        & Declarative verb \\
        -(e)\lilglot   & Connective \\
        -li         & Interrogative \\
        -ak         & Imperative \\
        -tu         & Relative \\
        -uc         & Subordinate \\
    \end{tabular}
\end{table}

\paragraph{Declarative verbs} are unmarked, finite, and modally neutral.

\paragraph{Connective verbs} may be either finite or non-finite. They work in conjunction with the head verb to describe concurrent or subsequent actions, or to modify the meaning of the verb clause with auxiliary verbs.

\pex[interpartskip=3ex]
\a
\begingl
naswi[\textsc{1ex}]
\famwordold{}{k}{aju}{l}{a}[water\_surface]
tui[on\_surface\_of]
\famwordold{}{f}{ii}{l}{}[notice]@
-ami[\sc -refl]@
-\lilglot[-\textsc{cvb}]
dak[can]
\glft `We could see ourselves in the water surface.'
\endgl
\a
\begingl
nas[\textsc{1s}]
bu[that]
\famwordold{}{n}{u}{w}{u}[possum]
\famwordold{}{r}{ii}{q}{}[hit]
-e\lilglot[\textsc{con}]
\famwordold{}{l}{a}{w}{}[up]
daw[towards]
\famwordold{i}{c}{aa}{n}{}[climb]
\glft `I'm climbing up to hit that possum.'
\endgl
\xe

\subsection{Evidential modality}

\lang{} has a four-way distinction within its evidentials that distinguishes direct witness with reportative, inferential, and internal/assumed speech. These affixes typically appear on the head verb, but may also be used on even non-finite verbs.

\begin{table}[ht]
    \centering
    \begin{tabular}{>{\em}rlll}
        \toprule
                & Function          & Example & Translation \\
        \midrule
        \nm-    & Direct Witness    & \famwordold{i}{n}{aa}{m}{}     & \textit{`they're eating'} \\
        ir\,-     & Reportative       & ir\famwordold{i}{n}{aa}{m}{}   & \textit{`they're eating, they said'} \\
        hwa-    & Inferential       & hwa\famwordold{i}{n}{aa}{m}{}  & \textit{`they're eating, judging by the smell'} \\
        qaa-    & Internal/Assumed  & qaa\famwordold{i}{n}{aa}{m}{}  & \textit{`they're probably eating, it's around dinner time'} \\
        \bottomrule
    \end{tabular}
    \caption{Evidential modality affixes}
    \label{tab:evidentials}
\end{table}

\ex
\begingl
\glpreamble \famwordold{}{f}{ana}{s}{aj} \famwordold{ir}{m}{a}{l}{aurutu} \famwordold{i}{n}{aa}{m}{e\lilglot} jaa.
\endpreamble
\famwordold{}{f}{ana}{s}{aj}[man]
ir-[\textsc{rep-}]@
\famwordold{}{m}{a}{l}{aurutu}[ill:\textsc{cop:rel}]
\famwordold{i}{n}{aa}{m}{e\lilglot}[eat:\textsc{con}]
jaa[indeed]
\glft `The man, who I was told was sick, was eating after all.'
\endgl
\xe

In the above example, the evidential attaches to and scopes over only the relative clause \emph{`\famwordold{}{f}{ana}{s}{aj} \famwordold{}{m}{a}{l}{aurutu}'}, leaving the matrix clause unmodified. 

\subsubsection{\emph{ir-} Reportative speech}

Information that has been obtained through the retelling by a secondary party is marked with \emph{ir-}. The speaker may not have been present to witness the event themselves, and are relying completely on hearsay.

\ex
\begingl
bu[\textsc{dem.dist}]
\famwordold{}{y}{a}{t}{}[shot]
was[\textsc{dem.prox:nmz}]
barari\lilglot[three:become:\textsc{cvb}]
ir-[\textsc{hsy-}]@
\famwordold{}{r}{ii}{t}{}[end]
\glft `He's done that trip three times.' \textit{(speaker heard from someone else)}
\endgl
\xe

\subsubsection{\emph{hwa-} Inferential speech}

If the speaker hasn't observed an event themselves and is interpolating from current circumstances, they may use \emph{hwa-} to mark this. 

\ex
\begingl
\famwordold{}{m}{u}{h}{u}[cow]
wa[\textsc{dem.prox
}]
\famwordold{}{PL}{a}{S}{}[place]
fit[at]
hwa-[\textsc{infer-}]@
\famwordold{}{n}{ii}{w}{}[died]
\glft `The cow was seemingly killed here.' \textit{(speaker noticed signs of struggle)}
\endgl
\xe

\subsubsection{\emph{qaa-} Internal/Assumed speech}

Verbs can also be marked for whether the speaker has no concrete evidence or report of the event, but may still assume that said event happened because of a gut instinct, tendencies, routines, or assumptions about the world.

\ex
\begingl
\famwordold{in}{f}{i}{s}{}[children]
qaa-[\textsc{inter-}]@
\famwordold{i}{s}{aa}{j}{}[sleep]
\glft `The children are probably asleep by now.' \textit{(uttered late at night)}
\endgl
\xe

\newpage
\section{Pronouns and determiners}

\begin{table}[ht]
    \centering
    \begin{tabular}{rll}
        & \textit{Nonplural} & \textit{Plural} \\
    \textit{Speaker-only} & nas & naswi \\
    \textit{Addressee-only} & mi & miwi \\
    \textit{Inclusive} & nemi & nemiwi \\
    \end{tabular}
    \caption{Discourse participant pronouns}
    \label{tab:firstandsecond}
\end{table}

\begin{table}[ht]
    \centering
    \begin{tabular}{>{\em}rll}
        & \textit{Determiner} & \textit{Pronoun}  \\
    Proximal & wa & wase \\
    Medial & par & parse \\
    Distal & bu & buse \\
    Interrogative & li & lise \\
    Relative & kun & kunse 
    \end{tabular}
    \caption{Determiners and demonstrative pronouns}
    \label{tab:determiners}
\end{table}

\chapter{Syntax}

\section{Verb stacking}

Verb phrases can contain several verbs, describing concurrent, subsequent, purposive, or consequential actions or states.

\section{Auxiliary verbs}

The auxiliary verbs are a set of semantically sparse verbs that convey aspectual and modal information. These verbs are either the head verb or a connective (dependent) verb, but always semantically scope over the entire VP.

\subsection{\emph{hwii} - negative}

As the head verb, hwii scopes over the whole VP, negating all of its subordinate verbs. As a connective verb, it scopes leftward, negating any other connective verbs that precede it.

\subsection{\emph{usnak} - hortative}

from WeSiiN → usin + -ak → usnak

encodes a sort of imperative function so doesn't really take -ak suffix

\subsubsection{Exhortative}

let's do X, c'mon

\subsubsection{Subjunctive?}

in subordinate clauses, smth like "would do X"?

\subsubsection{}

\section{Subordinate clauses}

Full verb phrases may be nominalized and act as an argument of another predicate.

\subsection{Relative clauses}

Relative clauses are a type of subordinate clauses that describes a referent's states or actions. They are internally headed, always verb-final, and the relative determiner \emph{kun} is used to mark the head of the clause, i.e. the thing that is being described.

\ex
\begingl
\famwordold{}{f}{ana}{s}{}[person]
\famwordold{i}{l}{aa}{s}{}[walk]@
-tu[\textsc{-rel}]
\famwordold{}{s}{a}{j}{a}uru[sleepy:\textsc{cop}]
\glft `The person who walked home was sleepy.'
\endgl
\xe

Clauses with a single argument do not require that the head is marked, as the argument is assumed to be the head by default. Still, the verb itself can be marked to describe the realization or performance of the action.

\ex
\begingl
\famwordold{in}{f}{i}{m}{}[children]
kun[\textsc{rel}]
\famwordold{i}{m}{aa}{w}{}[play]@
-tu[\textsc{-rel}]
naswi[\textsc{1p.ex}]
\famwordold{}{d}{ii}{l}{}[look]
\glft `We watched the playtime that the children were having'
\endgl
\xe

In high-valency clauses, \emph{kun} becomes more pertinent. The most agentive argument (subject) is considered to be the head of the phrase, but may still be marked for emphasis.

\pex[interpartskip=3ex]
\a
\begingl
(kun)[\textsc{rel}]
\famwordold{}{f}{ana}{s}{}[person]
\famwordold{i}{f}{u}{s}{a}[house]
daw[to]
fit[in]
\famwordold{i}{l}{aa}{s}{}tu[walk\textsc{:rel}]
nas[\textsc{1s}]
\famwordold{}{f}{ii}{l}{}[see]
\glft `I saw the person who walked into the house.'
\endgl
\a
\begingl
\famwordold{}{f}{ana}{s}{}[person]
kun[\textsc{rel}]
\famwordold{i}{f}{u}{s}{a}[house]
daw[to]
fit[in]
\famwordold{i}{l}{aa}{s}{}tu[walk\textsc{:rel}]
nas[\textsc{1s}]
\famwordold{}{f}{ii}{l}{}[see]
\glft `I saw the house that the person walked into.'
\endgl
\a
\begingl
\famwordold{}{f}{ana}{s}{}[person]
\famwordold{i}{f}{u}{s}{a}[house]
daw[to]
fit[in]
kun[\textsc{rel}]
\famwordold{i}{l}{aa}{s}{}tu[walk\textsc{:rel}]
nas[\textsc{1s}]
\famwordold{}{f}{ii}{l}{}[see]
\glft `I saw how the person walked into the house.'
\endgl
\xe

An alternative to using a determiner is simply to topicalize a given constituent. Only noun phrases may be relativized through topicalization; the relative verb may not be periphrastically topicalized (i.e. left-dislocated), as this introduces major syntactical ambiguities.

Due to the syntactic constraints of certain secondary derivations, they cannot inflect relative NPs directly.

\ex
\begingl
\glpreamble \ljudge{*} \famword{CuSu iFaaMtu-uru}
\endpreamble
\famword{CuSu}[cat]
\famword{iFaaMtu}[jump\textsc{:rel}]
kuns[\textsc{rel.pn}]@
-uru[\textsc{-cop}]
\glft `it's a talking cat.'
\endgl
\xe

\section{Comparative constructions}

from-comparative, marks standard (to which is compared)

\pex[interpartskip=3ex]
\a
\begingl
\famwordold{}{p}{u}{m}{u}[rabbit]
\famwordold{}{f}{ana}{s}{}[person]
fun[from]
\famwordold{}{m}{a}{nt}{a}[big]@
-uru[\textsc{-cop}]
\glft `The rabbit was bigger than a person.'
\endgl
\a
\begingl
\famwordold{}{t}{a}{n}{}[\textsc{top}/time]
nemi[\textsc{qual}/\textsc{du.in}]
buse[\textsc{std}/\textsc{dist:pn}]
fun[\textsc{mrk}/from]
\famwordold{}{j}{a}{l}{}[/many\_things]@
-ila[/-have]
\glft `We have more time than them.'
\endgl
\xe

\section{Animacy hierarchy}

\begin{table}[ht]
    \centering
    \begin{tabular}{ll}
    0 & Natural Forces \\
    1 & Pronouns (1>2>3) \\
    2 & Speakers of \lang{} \\
    3 & Non-speakers of \lang{} \\
    4 & Higher-order animals (mammals, octopus, intelligent creatures) \\
    5 & \parbox[t]{7cm}{Body parts, tools, any inanimate object used for acting upon something} \\
    6 & Lower-order animals (insects, mollusks, fish, worms, etc.) \\
    7 & Plants \\
    8 & Inanimate objects \\
    9 & Abstract concepts 
    \end{tabular}
    \caption{Animacy hierarchy in nominals}
    \label{tab:hierarchy}
\end{table}

\section{Causative constructions}

\lang{} has several different strategies when it comes to causative constructions, depending on the nature of the predicate in question. Some of these are morphological in nature, while others more periphrastic. 

\subsection{\textit{-ari} for nominal and adjectival predicates}

Simple nominal and adjectival predicates are turned into causatives using the translative suffix \textit{-ari}. If the predicate in question would be expressed with \textit{-uru} in its non-causative form, \textit{-ari} is likely appropriate for the causative.

\pex
\a
\begingl
\famwordold{}{q}{ar}{f}{i}[coffee]
\famwordold{}{s}{a}{fr}{a}[hot]@
-uru[\textsc{-cop}]
\glft `The coffee is hot.'
\endgl
\a
\begingl
\famwordold{}{q}{ar}{f}{i}[coffee]
nas[\textsc{1sg}]
\famwordold{}{s}{a}{fr}{a}[hot]@
-ari[-\textsc{transl}]
\glft `I heated up the coffee.'
\endgl
\xe

When used with only one argument, verbs ending in \textit{-ari} are assumed to have a null subject and the argument serving as the unaccusative object. This results in \textit{-ari} also serving as `to become' (the reason for its being glossed as `translative') as well as `to cause to be'.

\ex
\begingl
\famwordold{}{q}{ar}{f}{i}[coffee]
\famwordold{}{s}{a}{fr}{a}-[hot]@
ari[\textsc{transl}]
\glft `The coffee got hot.'
\endgl
\xe

\subsection{Valency-increasing verb patterns}

Which pattern is used to form the causative of a predicate depends largely on the nature of the intransitive form of that root. There are two different potentially valency-increasing patterns that can be used for verbs: the {\rootpart}ii{\rootpart} and the aa{\rootpart}i{\rootpart}. The exact effect of each of these valency-increasing operations depends on the individual root; their behavior can differ.

For verbs that would be agentive ambitransitives in English, such as `to eat', generally the behavior is rather straightforward: the {\rootpart}ii{\rootpart} form turns the verb into a straightfoward transitive, and the aa{\rootpart}i{\rootpart} form serves as a causative of the intransitive. 

\pex
\a
\begingl
nas[\textsc{1sg}]
\famwordold{i}{n}{aa}{m}{}[eat\textbackslash\textsc{intr}]
\glft `I was eating.'
\endgl
\a
\begingl
nas[\textsc{1sg}]
\famwordold{}{k}{ur}{k}{i}[cookie]
\famwordold{}{n}{ii}{m}{}[eat\textbackslash\textsc{tr}]
\glft `I ate a cookie.'
\endgl
\a
\begingl
nas[\textsc{1sg}]
\famwordold{in}{m}{i}{m}{}[parent\_child\textbackslash\textsc{dim}]
\famwordold{aa}{n}{i}{m}{}[eat\textbackslash\textsc{caus}]
\glft `I fed my daughter.'
\endgl
\xe

It's worth noting that object of the transitive verb cannot be included as the object of the causative verb; the causative verb can still only have two arguments.

\ex
\ljudge{*}
\begingl
nas[\textsc{1sg}]
\famwordold{in}{m}{i}{m}{}[parent\_child\textbackslash\textsc{dim}]
\famwordold{}{k}{ur}{k}{i}[cookie]
\famwordold{aa}{n}{i}{m}{}[eat\textbackslash\textsc{caus}]
\endgl
\xe

\noindent To express this notion, a periphrastic causative would be required.

Other types of verbal paradigms make this causative relationship less obvious and use these roots in other ways. For instance, for some roots the intransitive form is unaccusative or passive in nature. In these cases, the transitive form behaves as a causative:

\pex
\a
\begingl
nas[\textsc{1sg}]
wan[\textsc{poss}]
\famwordold{}{m}{ana}{m}{}[parent\_child]
\famwordold{i}{n}{aa}{w}{}[death\textbackslash\textsc{intr}]
\glft `My mother died.'
\endgl
\a
\begingl
nas[\textsc{1sg}]
\famwordold{}{m}{ana}{m}{}[parent\_child]
\famwordold{}{n}{ii}{w}{}[death\textbackslash\textsc{tr}]
\glft `I killed my mother.'
\endgl
\xe

For these roots, the aa{\rootpart}i{\rootpart} form means the same thing as the {\rootpart}ii{\rootpart} form, but while the {\rootpart}ii{\rootpart} form implies a successfully completed action, the same implication is not present for the causative form.

\ex
\begingl
nas[\textsc{1sg}]
\famwordold{}{m}{ana}{m}{}[parent\_child]
\famwordold{aa}{n}{i}{w}{}[death\textbackslash\textsc{caus}]
\glft `I tried to kill my mother' (and she may or may not have died).
\endgl
\xe

\noindent For many of these roots, the intransitive is identical in meaning to a `passive' use of the transitive with an omitted subject; whether there is any noticeable difference between these depends on the verb.

\ex
\begingl
nas[\textsc{1sg}]
wan[\textsc{poss}]
\famwordold{}{m}{ana}{m}{}[parent\_child]
\famwordold{}{n}{ii}{w}{}[death\textbackslash\textsc{tr}]
\glft `My mother was killed.'
\endgl
\xe

% to-do
Unergative verbs

\subsection{Periphrastic causatives}

In addition to the morphological causatives above and their aforementioned limitations, \lang{} has a periphrastic causative that can scope over a wider variety of predicates. This periphrasis is expressed through a serial construction using the verb \textit{\textsc{w}\famwordold{e}{s}{ii}{n}{}} `to effect, to cause' followed by the description of the caused predicate. 

\ex
\begingl
nas[\textsc{1sg}]
\textsc{w}\famwordold{e}{s}{ii}{n}{}[bring\_about]
,[]
\famwordold{}{q}{ar}{f}{i}[coffee]
mi[\textsc{2sg}]
\famwordold{}{k}{ii}{l}{}[drink]
\glft `I caused you to drink coffee.' (lit., `I brought it about, you drank coffee.')
\endgl
\xe

Insert stuff about causatives and directness here.

\chapter{Semantics and pragmatics}

\section{Numbers}

Numerals in {\lang} serve several purposes. 

\subsection{Cardinal numbers}

Cardinal numbers make use of bare numerals modifying nouns directly.

\subsection{Ordinal numbers}

Ordinal numbers 

\subsection{Distributive numbers}

Distributive numbers express the quantity of concurrent relevant items, as the answer to "how many at a time?" or "how many per iteration?". They make use of the numerals as modified by \emph{-lat}

\section{Phatic expressions}

Phatic expressions in \lang{} are all in some way related to the nouns they are derived from, suggesting an emphasis on acknowledging the addressee's current or upcoming actions. The addressee may respond with the same expression back, even if it does not apply to the original speaker in any way, or respond in kind with a more suitable expression.

The obligatory gender marking is a means of expressing your gender identity in an unintrusive manner.\footnote{The real reason is that as Beth once ended a conversation with "sayonara", Knut noticed some coincidental similarities with the word \famword{SaJ} `sleep' and the affix -un to indicate feminine gender, with the -ara reanalyzed as a phatic/optative marker of sorts.} One always uses the gender marker that approximates ones gender identity (or at least intended gender presentation at the time), even if responding to someone else's phatic expression with a different gender marker. In contexts where the gender of the speaker is unclear or intentionally left unspecified, \textit{-an-} is used. More recently, \textit{-uj-} has been innovated as an explicitly nonbinary variant, though its use has been pretty much exclusively limited to the LGBTQ+ community.

In informal speech, most speakers shorten \textit{-unara} and \textit{-ajara} to \textit{-nar} and \textit{-jar}, respectively. While the full forms are generally used in formal contexts, the shorter forms are more likely to occur when speaking more casually. There are no generally accepted shortened variants for \textit{-anara}, which tends to not be used outside of formal contexts anyway, or \textit{-ujara}, whose use is limited anyway. 

In the Southernmost regions of Nareland, where contact with the Celts has always been strongest, most dialects have replaced \textit{-ara} with the Scottish Gaelic-inspired \textit{-maa} and \textit{-faa} replacing \textit{-jar} and \textit{-nar}, respectively. 
In urban areas in the East, particularly the capital, Danish influence has led to many speakers using \textit{kud X} (derived from Danish \textit{god}) in colloquial speech. 
In both places, however, the full form is generally still used for formal speech. A speaker of \lang{} from any region is likely to use \textit{\famword{FaSajara/FaSunara}} to greet someone at a job interview. 
With family and friends, a speaker from Masintulwa is likely to use \textit{\famword{kud FaS}}, a speaker from SOUTHERNCITY is likely to use \textit{\famword{FaSmaa/FaSfaa}}, and speakers from most other parts of the country are likely to use \textit{\famword{FaSjar/FaSnar}}.

\subsection*{Examples of Common Phatic Expressions}

\paragraph{\famword{FaSanara}} \textit{(from \famword{FaS} `life')} is a catch-all greeting, suitable for any time of day. It's generally only used in person.

\paragraph{\famword{SaJanara}} \textit{(from \famword{SaJ} `sleep')} is similar in use to "goodbye" or "good night". It is only used if the people in question intend to be separated for a period that includes a night of sleep before seeing each other again.

\paragraph{\famword{YaTanara}} \textit{(from \famword{YaT} `travel not of one's own power or volition')} is used to wish someone a pleasant trip where the person is not directly in control of their means of transportation (e.g. on public transport or as a passenger in a car). In contrast, if the person has direct control over their travel, e.g. by walking or driving a car, one would rather use \textbf{\famword{PLaSanara}} \textit{(from \famword{PLaS} `movement')} or \textbf{\famword{LaSanara}} \textit{(from \famword{LaS} `walking')}.

\section{Name determiners}

In \lang{}, when a name is used referentially (that is, pointing out a particular entity named that), the name must be preceded by a naming particle---\textit{ha} for male names and \textit{fu} for female names (derived from former personal pronouns that have now been replaced by demonstratives in other contexts). More recently, \textit{na} has been innovated as a gender-neutral alternative (this new \textit{na} being unrelated to the former first-person determiner).

These determiners are only required when the name in question is serving a referential function, so they are not necessary when referring to the name itself as a concept (such as in `My name is ...' constructions) or in direct address.

\pex
\a
ha \textsc{Karl}-la fu \textsc{Janne} \famwordold{i}{l}{aa}{f}{}\\
\textit{`Karl \& Janne are in love.'}
\a
pars \textsc{Karl} baj \famwordold{i}{m}{aa}{h}{}\\
\textit{`His name is Karl.'}
\a
\textsc{Karl}, \famwordold{i}{w}{aa}{b}{ak}!\\
\textit{`Karl, come back!'}
\xe


\section{Idiomatic expressions}

\famwordold{}{c}{u}{mp}{u} \famwordold{}{c}{u}{mp}{uuru} = no shit, preaching to the choir

\part{Dictionary}

\setsecnumdepth{part}
\settocdepth{section}

\chapter{Roots and Derived Words}
\begin{multicols*}{2}

%%%%%%%%%%%
%    7
%%%%%%%%%%%
\section*{\bigglot}

\begin{dictroot}{\bigglot}{\bigglot}\label{root:7_7}
    \begin{dictentry}{{\bigglot}a\bigglot}{n.}\label{word:7a7}
        \dictdef*{%
            stupidity
            \begin{quote}
                \famword{iRu\bigglot a fit ReTaSa \bigglot a\bigglot ila}\\
                \textit{`In the legislature, there's only stupidity.'}
            \end{quote}
        }
    \end{dictentry}
    \begin{dictentry}{{\bigglot}ii\bigglot}{v.tr.}\label{word:7ii7}
        \dictdef*{%
            to fail to achieve \textit{smth.}, to miss attaining \textit{smth.}, to fall short of \textit{smth.}
            \begin{quote}
                \famword{iSKuLa wa RaT fit bar SKanaL \bigglot ii\bigglot}\\
                \textit{`Three kids flunked out of school this year.'}
            \end{quote}
        }
    \end{dictentry}
    \begin{dictentry}{{\bigglot}iya{\bigglot}{}}{v.intr.}\label{word:7iya7}
        \dictdef*{%
            \textit{(of an attempt or instrument)} to fail, to be unsuccessful, to not succeed, to be found wanting
            \begin{quote}
                \famword{was wan TaTaBaT \bigglot iya\bigglot e7 lit}\\
                \textit{`His new idea failed spectacularly.'}
            \end{quote}
            }
    \end{dictentry}
    \begin{dictentry}{i{\bigglot}aa\bigglot}{v.intr.}\label{word:i7aa7}
        \dictdef{%
            to behave stupidly, to act stupidly, to mess around
            \begin{quote}
                \famword{mi i\bigglot aa\bigglot e\lilglot tuuquc ManaM FiiM}\\
                \textit{`Mom says that you need to stop messing around.'}
            \end{quote}
        }
        \dictdef{%
            \textit{(of an agent)} to fail, to not achieve anything, to be fall short of the understood standard, to fuck up
            %\begin{quote}
                %pars 
            %\end{quote}
        }
    \end{dictentry}
    \begin{dictentry}{{\bigglot}i{\bigglot}iya\bigglot}{v.tr.}\label{word:7i7iya7}
        \dictdef*{
            to cause \textit{smth.} to fail, to sabotage \textit{smth.}, to fuck with \textit{smth.}, to mess \textit{smth.} up
        }
    \end{dictentry}
    \begin{dictentry}{aa{\bigglot}i\bigglot}{v.tr.}\label{word:aa7i7}
        \dictdef{%
            to cause \textit{sme.} to act like an idiot, to goad \textit{sme.} on, to instigate \textit{sme.}
        }
        \dictdef{%
            to cause \textit{sme.} to fail, to sabotage \textit{sme.}, to trip \textit{sme.} up, to mess with \textit{sme.}
        }
    \end{dictentry}
    \begin{dictentry}{\bigglot{}a\bigglot{}a}{adj.}\label{word:7a7a}
        \dictdef*{
            stupid
        }
    \end{dictentry}
    \begin{dictentry}{\bigglot{}ana\bigglot}{n.}\label{word:7ana7}
        \dictdef{%
            stupid person, fool
        }
        \dictdef{%
            \textit{(archaic)} native Narelander, \lang{} speaker
        }
    \end{dictentry}
    \begin{dictentry}{i\bigglot u\bigglot a}{n.}\label{word:i7u7a}
        \dictdef{%
            Nareland, an island to the northwest of Scotland, home of the Narelanders
        }
        \dictdef{%
            \textit{(meta, joke)} the CDN
        }
    \end{dictentry}
    \begin{dictentry}{\bigglot uu\bigglot}{n.}\label{word:7uu7}
        \dictdef*{\textit{(archaic)} Narelanders, the ethnic group descended from the pre-Indo-European inhabitants of Nareland}
    \end{dictentry}
\end{dictroot}

\begin{dictroot}{\bigglot}{j}\label{root:7_J}
    % aunt/uncle-niece/nephew (close)
    \begin{dictentry}{{\bigglot}anaJ}{n.}\label{word:7anaJ}
        \dictdef{
            parent's sibling, aunt, uncle
        }
        \dictdef{
            sibling's child, niece, nephew
        }
    \end{dictentry}
\end{dictroot}

\begin{dictroot}{\bigglot}{ln}\label{root:7_LN}
    \begin{dictentry}{{\bigglot}aLN}{n.}\label{word:7aLN}
        \dictdef*{
            half a meter, 50cm, one ell \textit{(from Danish `alen')}
        }
    \end{dictentry}
    \begin{dictentry}{in{\bigglot}iLN}{n.}\label{word:in7iLN}
        \dictdef*{
            2cm, one `inch'/thumb
        }
    \end{dictentry}
    \begin{dictentry}{7uliLN}{n.}\label{word:7uliLN}
        \dictdef*{
            forearm
        }
    \end{dictentry}
\end{dictroot}

%%%%%%%%%%%
%    B
%%%%%%%%%%%
\section*{B}

\begin{dictroot}{b}{\bigglot}\label{root:B_7}
    % sibling (close)
    \begin{dictentry}{Bana\bigglot}{n.}\label{word:Bana7}
        \dictdef*{
            sibling, sister, brother, cousin
        }
    \end{dictentry}
\end{dictroot}

\begin{dictroot}{b}{b}\label{root:B_B}
    % so-so, quiet, light, not intense
    \begin{dictentry}{BaB}{n.}\label{word:BaB}
        \dictdef{
            simplicity
        }
        \dictdef{
            softness
        }
        \dictdef{
            silence
        }
        \dictdef{
            unobtrusiveness
        }
    \end{dictentry}
    \begin{dictentry}{BiiB}{v.tr}\label{word:BiiB}
        \dictdef{
            to hug \textit{sme.}
        }
        \dictdef{
            to squeeze \textit{smth.}
        }
    \end{dictentry}
    \begin{dictentry}{BiyaB}{v.intr.}\label{word:BiyaB}
        \dictdef*{
            \textit{of a task} to be performed or finished simply, easily, effortlessly
        }
    \end{dictentry}
    \begin{dictentry}{iBaaB}{v.intr.}\label{word:iBaaB}
        \dictdef{
            to behave oneself, keep quiet
        }
        \dictdef{
            to stay out of the way, not be a nuisance
        }
    \end{dictentry}
    \begin{dictentry}{BiBiyaB}{v.tr.}\label{word:BiBiyaB}
        \dictdef*{
            to perform a task effortlessly, easily
        }
    \end{dictentry}
    \begin{dictentry}{aaBiB}{v.tr.}\label{word:aaBiB}
        \dictdef*{
            to raise, nurture, care for, rear
        }
    \end{dictentry}
    \begin{dictentry}{BaBa}{adj.}\label{word:BaBa}
        \dictdef{
            soft, not hard, not rough
        }
        \dictdef{
            quiet, muffled
        }
        \dictdef{
            inoffensive, plain
        }
        \dictdef{
            easy, trivial
        }
        \dictdef{
            sensitive, gentle, with a light touch
        }
    \end{dictentry}
    \begin{dictentry}{BanaB}{n.}\label{word:BanaB}
        \dictdef{
            affectionate person, caring person
        }
        \dictdef{
            ragdoll
        }
    \end{dictentry}
    \begin{dictentry}{mBiB}{n.}\label{word:mBiB}
        \dictdef{
            fabric softener
        }
        \dictdef{
            exhaust muffler
        }
    \end{dictentry}
    \begin{dictentry}{BuBi}{n., adj.}\label{word:BuBi}
        \dictdef*{
            pink, the color pink
        }
    \end{dictentry}
    \begin{dictentry}{BuliB}{v.tr.}\label{word:BuliB}
        \dictdef*{
            the human breast
        }
    \end{dictentry}
    \begin{dictentry}{BuBu}{n.}\label{word:BuBu}
        \dictdef*{
            stuffed animal, teddy bear
        }
    \end{dictentry}
    \begin{dictentry}{BajuBa}{n.}\label{word:BajuBa}
        \dictdef*{
            a soft blanket, commonly used to 
        }
    \end{dictentry}
    \begin{dictentry}{BimiBu}{n.}\label{word:BimiBu}
        \dictdef*{
            bra, brassiere, undergarment worn to support the breasts 
        }
    \end{dictentry}
\dictsubtitle{Compounds \& Secondary Derivations}
    \begin{dictentry}{BaBainala}{v.tr.}\label{word:BaBainala}
        \dictdef{
            to muffle, shush, quiet
        }
        \dictdef{
            to simplify
        }
    \end{dictentry}
\end{dictroot}

\begin{dictroot}{b}{c}\label{root:B_C}
    % small, few
    \begin{dictentry}{BaCa}{adj.}\label{word:BaCa}
        \dictdef{
            small
        }
        \dictdef{
            few, small number of
        }
    \end{dictentry}
\end{dictroot}

\begin{dictroot}{b}{j}\label{root:B_J}
    \begin{dictentry}{BaJ}{n.}\label{word:BaJ}
        \dictdef*{
            badness, evil, bad luck, bad fortune, bad vibes
        }
    \end{dictentry}
    \begin{dictentry}{BiiY}{v.tr.}\label{word:BiiY}
        \dictdef*{
            to treat \textit{sme.} badly, to screw \textit{sme.} over, to behave badly towards \textit{sme.}
        }
    \end{dictentry}
    \begin{dictentry}{BiyaJ}{v.intr.}\label{word:BiyaJ}
        \dictdef*{
            to have bad luck, to experience poor fortune, to be cursed
        }
    \end{dictentry}
    \begin{dictentry}{iBaaJ}{v.intr.}\label{word:iBaaJ}
        \dictdef*{
            to do evil, to do bad deeds, to behave badly, to misbehave
        }
    \end{dictentry}
    \begin{dictentry}{BiBiyaJ}{v.tr.}\label{word:BiBiyaJ}
        \dictdef*{
            to curse \textit{sme.}, to bestow bad luck upon \textit{sme.}, to give \textit{sme.} a curse
        }
    \end{dictentry}
    \begin{dictentry}{aaBiY}{v.tr.}\label{word:aaBiY}
        \dictdef*{
            to cause \textit{sme.} to behave badly, to instigate \textit{sme.}, to influence \textit{sme.} towards evil, to corrupt \textit{sme.}
        }
    \end{dictentry}
    \begin{dictentry}{BaJa}{adj.}\label{word:BaJa}
        \dictdef*{
            bad, unpleasant, unlucky, unfortunate
        }
    \end{dictentry}
    \begin{dictentry}{uBiJi}{adj.}\label{word:uBiJi}
        \dictdef*{
            unhappy, upset
        }
    \end{dictentry}
    \begin{dictentry}{BanaJ}{n.}\label{word:BanaJ}
        \dictdef{
            evildoer, bad person, villain
        }
        \dictdef{
            enemy, nemesis
        }
    \end{dictentry}
\end{dictroot}

\begin{dictroot}{b}{n}\label{root:B_N}
    \begin{dictentry}{BaN}{n.}\label{word:BaN}
        \dictdef*{
            habit, routine
        }
    \end{dictentry}
    \begin{dictentry}{BiiN}{v.tr.}\label{word:BiiN}
        \dictdef*{
            to keep \textit{smth.} as a habit, to do \textit{smth.} usually, to have \textit{smth.} as a routine
        }
    \end{dictentry}
    \begin{dictentry}{BiyaN}{v.intr.}\label{word:BiyaN}
        \dictdef*{
            to be usual, to be typical practice, to be accepted as normal routine
        }
    \end{dictentry}
    \begin{dictentry}{iBaaN}{v.intr.}\label{word:iBaaN}
        \dictdef*{
            to behave as usual, to go about one's daily business, to do one's usual tasks
            }
    \end{dictentry}
    \begin{dictentry}{BiBiyaN}{v.tr.}\label{word:BiBiyaN}
        \dictdef*{
            to mandate \textit{smth.}, to put \textit{smth.} into practice, to make \textit{smth.} part of the day-to-day routine, to establish \textit{smth.}
        }
    \end{dictentry}
    \begin{dictentry}{aaBiN}{v.tr.}\label{word:aaBiN}
        \dictdef*{
            to train \textit{smth./sme.}, to habituate \textit{sme.} to something, to familiarize \textit{sme.} with something
        }
    \end{dictentry}
    \begin{dictentry}{BaNa}{adj.}\label{word:BaNa}
        \dictdef*{
            usual, common, ordinary, typical, banal, everyday, day-to-day, frequent
        }
    \end{dictentry}
    \begin{dictentry}{uBiNi}{adj.}\label{word:uBiNi}
        \dictdef*{
            bored, dissatisfied, exhausted, sick of the humdrum day-to-day grind
        }
    \end{dictentry}
    \begin{dictentry}{BuNi}{adj., n.}\label{word:BuNi}
        \dictdef*{
            yellow, the color yellow
        }
    \end{dictentry}
    \begin{dictentry}{BasiN}{n.}\label{word:BasiN}
        \dictdef*{
            banana
        }
    \end{dictentry}
\dictsubtitle{Compounds \& Secondary Derivations}
    \begin{dictentry}{BaNiri}{v.intr.}\label{word:BaNiri}
        \dictdef{
            to do routine tasks, to do one's routine
        }
        \dictdef{
            \textit{(euphemistic)} to use the restroom, to take a piss
        }
    \end{dictentry}
    \begin{dictentry}{uBiNiisa}{n.}\label{word:uBiNiisa}
        \dictdef*{
            boredom, dissatisfaction, ennui
        }
    \end{dictentry}
\end{dictroot}

\begin{dictroot}{b}{rb}\label{root:B_RB}
    %to leave (expecting to return), to brb
    \begin{dictentry}{BiyaRB}{v.intr.}\label{word:BiyaRB}
        \dictdef*{
            to be on hiatus, to be gone but expected to return shortly
        }
    \end{dictentry}
    \begin{dictentry}{iBaaRB}{v.intr.}\label{word:iBaaRB}
        \dictdef*{
            to leave expecting to return shortly
        }
    \end{dictentry}
    \dictsubtitle{Compounds \& Secondary Derivations}
    \begin{dictentry}{BaRBanara}{int.}\label{word:BaRBanara}
        \dictdef*{be right back, see you soon, goodbye (when the speaker expects to see the addresse again before either of them has a night's sleep)}
    \end{dictentry}
\end{dictroot}

\begin{dictroot}{b}{rd}\label{root:B_RD}
    %bird
    \begin{dictentry}{BaRD}{n.}\label{word:BaRD}
        \dictdef*{
            flight
        }
    \end{dictentry}
    \begin{dictentry}{BiiRD}{v.tr}\label{word:BiiRD}
        \dictdef{
            to fly to \textit{smwh.}
        }
        \dictdef{
            \textit{(of a projectile like a bullet or arrow)} to hit \textit{smth. or sme.}
        }
    \end{dictentry}
    \begin{dictentry}{BiyaRD}{v.intr.}\label{word:BiyaRD}
        \dictdef{
            to fly, to be a passenger on a plane
        }
        \dictdef{
            to fly due to inertia, to fly through the air not of one's own power
        }
    \end{dictentry}
    \begin{dictentry}{iBaaRD}{v.intr}\label{word:iBaaRD}
        \dictdef*{
            to fly, to fly around
        }
    \end{dictentry}
    \begin{dictentry}{BiBiyaRD}{v.tr.}\label{word:BiBiyaRD}
        \dictdef{
            to throw \textit{smth.}, to launch \textit{smth.}, to shoot \textit{smth.}, to toss into the air
        }
        \dictdef{
            to shoot or launch a projectile \textit{(as from a gun, catapult, or cannon)}
        }
    \end{dictentry}
    \begin{dictentry}{aaBiRD}{v.tr.}\label{word:aaBiRD}
        \dictdef{
            to pilot \textit{smth.}
        }
    \end{dictentry}
    \begin{dictentry}{BaRDa}{adj.}\label{word:BaRDa}
        \dictdef{
            airborne
        }
        \dictdef{
            light, weightless
        }
    \end{dictentry}
    \begin{dictentry}{BanaRD}{n.}\label{word:BanaRD}
        \dictdef{
            acrobat, trapeze artist, tumbler, gymnast
        }
    \end{dictentry}
    \begin{dictentry}{BuRDi}{n.}\label{word:BuRDi}
        \dictdef*{
            bullet, cannonball, projectile
        }
    \end{dictentry}
    \begin{dictentry}{iBuRDa}{n.}\label{word:iBuRDa}
        \dictdef*{
            airport
        }
    \end{dictentry}
    \begin{dictentry}{mBiRD}{n.}\label{word:mBiRD}
        \dictdef{
            bow, catapult, trebuchet, slingshot, gun, any machine designed to launch a projectile
        }
        \dictdef{
            the engine of a plane, that which propels something to fly
        }
    \end{dictentry}
    \begin{dictentry}{BuliRD}{n.}\label{word:BuliRD}
        \dictdef{
            wing, a bird's wing
        }
    \end{dictentry}
    \begin{dictentry}{BuRDu}{n.}\label{word:BuRDu}
        \dictdef{
            bird
        }
        \dictdef{
            plane, airplane
        }
    \end{dictentry}
    \begin{dictentry}{BasiRD}{n.}\label{word:BasiRD}
        \dictdef{
            arrow, bolt, long thin projectile
        }
    \end{dictentry}
    \begin{dictentry}{BajuRDa}{n.}\label{word:BajuRDa}
        \dictdef{
            wing, the wing of a plane or glider
        }
        \dictdef{
            kite
        }
    \end{dictentry}
    \begin{dictentry}{aBiRDu}{n.}\label{word:aBiRDu}
        \dictdef*{
            bird's nest
        }
    \end{dictentry}
    \begin{dictentry}{BimiRDu}{n.}\label{word:BimiRDu}
        \dictdef{
            cage, bird cage
        }
    \end{dictentry}
    \dictsubtitle{Compounds \& Secondary Derivations}
    \begin{dictentry}{aaBiRDana}{n.}\label{word:aaBiRDana}
        \dictdef{
            pilot, someone who flies a plane
        }
    \end{dictentry}
\end{dictroot}

\begin{dictroot}{b}{t}\label{root:B_T}
    \begin{dictentry}{BaT}{n.}\label{word:BaT}
        \dictdef*{
            knowledge, understanding
            \begin{quote}
                \famword{wa aKiJu fit JaLa BaTila.} \\
                \textit{`There's much knowledge in these books.'}
            \end{quote}
        }
    \end{dictentry}
    \begin{dictentry}{BiiT}{v.tr.}\label{word:BiiT}
        \dictdef{%
            to know \textit{smth.}, to understand \textit{smth.}
            \begin{quote}
                \famword{mi iBaaRBe\lilglot{} kajuc nas BiiT}\\
                \textit{`I know that you want to leave.'}
            \end{quote}
            }
        \dictdef{%
        to love \textit{sme.} like a brother, to have a close platonic bond with \textit{sme.}, to be best friends with \textit{sme.}
            \begin{quote}
                \famword{nas JanaB BiiTibi}\\
                \textit{`I love my friends.'}
            \end{quote}
        \textit{NB: the subject is reversed from its use as `to understand': \emph{\famword{mi nas BiiTibi}} means `you understand me' but `I love you' (platonically).}
            }
    \end{dictentry}
    \begin{dictentry}{BiyaT}{v.intr.}\label{word:BiyaT}
        \dictdef*{
            to be known, to be understood, to be a given, to be common sense, to be obvious
        }
    \end{dictentry}
    \begin{dictentry}{iBaaT}{v.intr.}\label{word:iBaaT}
        \dictdef{%
        to know, to understand, to be in a state of knowing or understanding what is going on
        }
        \dictdef{%
        \textit{(when used reciprocally)} to love each other, to have a close platonic bond, to be the best of friends
            \begin{quote}
                nemi i\textsc{b}aa\textsc{t}ami\\
                \textit{`The two of us are thick as thieves.'}
            \end{quote}
        }
    \end{dictentry}
    \begin{dictentry}{BiBiyaT}{v.tr.}\label{word:BiBiyaT}
        \dictdef*{
            to point \textit{smth.} out, to make \textit{smth.} known, to tell \textit{smth.}, to make \textit{smth.} obvious, to show \textit{smth.}
        }
    \end{dictentry}
    \begin{dictentry}{aaBiT}{v.tr.}\label{word:aaBiT}
        \dictdef*{
            to tell \textit{sme.}, to let \textit{sme.} know
            \begin{quote}
                \famword{TaT mi BiiTuc udan (bus fun) nas aaBiTibiak.}\\
                \textit{`If you know the news, tell me (it).'}
            \end{quote}
        }
    \end{dictentry}
    \begin{dictentry}{BaTa}{adj.}\label{word:BaTa}
        \dictdef*{
            knowledgeable, understanding, wise
        }
    \end{dictentry}
    \begin{dictentry}{uBiTi}{adj.}\label{word:uBiTi}
        \dictdef*{
            smug, arrogant, full of oneself, feeling like one knows everything
        }
    \end{dictentry}
    \begin{dictentry}{BanaT}{n.}\label{word:BanaT}
        \dictdef{
            expert, source, knowledgeable person
        }
        \dictdef{
            chief, head, boss
        }
    \end{dictentry}
    \begin{dictentry}{BuliT}{n.}\label{word:BuliT}
        \dictdef*{
            head
        }
    \end{dictentry}
\dictsubtitle{Compounds \& Secondary Derivations}
    \begin{dictentry}{BaTaisa}{n.}\label{word:BaTaisa}
        \dictdef*{
            knowledgeableness, wisdom
        }
    \end{dictentry}
    \begin{dictentry}{uBiTiisa}{n.}\label{word:uBiTiisa}
        \dictdef*{
            smugness, arrogance
        }
    \end{dictentry}
\end{dictroot}

\begin{dictroot}{br}{t}\label{root:BR_T}
    % Wales, Welsh, Great Britain
    \begin{dictentry}{BRuuT}{n.}\label{word:BRuuT}
        \dictdef{
            Wales
        }
        \dictdef{
            the Welsh people
        }
        \dictdef{
            Great Britain
        }
        \dictdef{
            the British people
        }
    \end{dictentry}
\end{dictroot}

\begin{dictroot}{bw}{\bigglot}\label{root:BW_7}
    \begin{dictentry}{BWa\bigglot}{n.}\label{word:BWa7}
        \dictdef*{
            month (of the year)
        }
    \end{dictentry}
    \begin{dictentry}{BWiya\bigglot}{v.intr.}\label{word:BWiya7}
        \dictdef*{
            to recur on a monthly basis, to be monthly
        }
    \end{dictentry}
    \begin{dictentry}{iBWaa\bigglot}{v.intr.}\label{word:iBWaa7}
        \dictdef*{
            to be on one's period, to be menstruating
        }
    \end{dictentry}
    \begin{dictentry}{BWa\bigglot a}{adj.}\label{word:BWa7a}
        \dictdef{
            monthly, per month
        }
        \dictdef{
            menstrual
        }
    \end{dictentry}
    \begin{dictentry}{uBWi7i}{adj.}
        \dictdef{
            moody, crampy, bloated, feeling the symptoms of PMS or period cramps
        }
    \end{dictentry}
    \begin{dictentry}{iBWu\bigglot a}{n.}\label{word:iBWu7a}
        \dictdef*{
            period, menstruation, `that time of the month'
        }
    \end{dictentry}
\dictsubtitle{Compounds \& Secondary Derivations}
    \begin{dictentry}{awkus-BWa\bigglot}{n.}\label{word:awkus-BWa7}
        \dictdef*{
            August, 8th month of the Gregorian calendar
        }
    \end{dictentry}
    \begin{dictentry}{disi-BWa\bigglot}{n.}\label{word:disi-BWa7}
        \dictdef*{
            December, 12th month of the Gregorian calendar
        }
    \end{dictentry}
    \begin{dictentry}{ephii-BWa\bigglot}{n.}\label{word:ephii-BWa7}
        \dictdef*{
            April, 4th month of the Gregorian calendar
        }
    \end{dictentry}
    \begin{dictentry}{fi-BWa\bigglot}{n.}\label{word:fi-BWa7}
        \dictdef*{
            February, 2nd month of the Gregorian calendar
            }
    \end{dictentry}
    \begin{dictentry}{jan-BWa\bigglot}{n.}\label{word:jan-BWa7}
        \dictdef*{
            January, 1st month of the Gregorian calendar
        }
    \end{dictentry}
    \begin{dictentry}{juu\lilglot li-BWa\bigglot}{n.}\label{word:juu7li-BWa7}
        \dictdef*{
            July, 7th month of the Gregorian calendar
        }
    \end{dictentry}
    \begin{dictentry}{juu\lilglot ni-BWa\bigglot}{n.}\label{word:juu7ni-BWa7}
        \dictdef*{
            June, 6th month of the Gregorian calendar
        }
    \end{dictentry}
    \begin{dictentry}{maats-BWa\bigglot}{n.}\label{word:maats-BWa7}
        \dictdef*{
            March, 3rd month of the Gregorian calendar
        }
    \end{dictentry}
    \begin{dictentry}{maj-BWa\bigglot}{n.}\label{word:maj-BWa7}
        \dictdef*{
            May, 5th month of the Gregorian calendar
        }
    \end{dictentry}
    \begin{dictentry}{nufim-BWa\bigglot}{n.}\label{word:nufim-BWa7}
        \dictdef*{
            November, 11th month of the Gregorian calendar
        }
    \end{dictentry}
    \begin{dictentry}{sepcim-BWa\bigglot}{n.}\label{word:sepcim-BWa7}
        \dictdef*{
            September, 9th month of the Gregorian calendar
        }
    \end{dictentry}
    \begin{dictentry}{uktuu-BWa\bigglot}{n.}\label{uktuu-BWa7}
        \dictdef*{
            October, 10th month of the Gregorian calendar
        }
    \end{dictentry}
\end{dictroot}

%%%%%%%%%%%
%    C
%%%%%%%%%%%
\section*{C}

\begin{dictroot}{c}{c}\label{root:C_C}
    % example
    \begin{dictentry}{CaC}{n.}\label{word:CaC}
        \dictdef{
            example
        }
        \dictdef{
            such an idea, an idea like that
        }
    \end{dictentry}
    \begin{dictentry}{CiiC}{v.tr}\label{word:CiiC}
        \dictdef*{
            to serve as an example of \textit{smth.}
            \begin{quote}
                \famword{[SMTH] [ELSE] CiiC.}\\
                \textit{[SMTH] is an example of [ELSE].}
            \end{quote}
        }
    \end{dictentry}
    \begin{dictentry}{CiyaC}{v.intr.}\label{word:CiyaC}
        \dictdef*{
            to be an example, to serve as an example
            \begin{quote}
                \famword{[SENT]. [THING] CiyaC.}\\
                \textit{[SENT]. For example, [THING].}
            \end{quote}
        }
    \end{dictentry}
    \begin{dictentry}{iCaaC}{v.intr.}\label{word:iCaaC}
        \dictdef*{
            to give an example, to provide an example
            \begin{quote}
                \famword{nas iCaaCuc mi kajli?}\\
                \textit{Do you want me to give an example?}
            \end{quote}
        }
    \end{dictentry}
    \begin{dictentry}{CiCiyaC}{v.tr.}\label{word:CiCiyaC}
        \dictdef*{
            to use \textit{smth. or sme.} as an example, to make an example of \textit{smth. or sme.}
            \begin{quote}
                \famword{mi iBaaJe\lilglot{} hakuc udan aaSKiLana mi qaa[PUNISH]ibi7 CiCiyaCibi.}\\
                \textit{If you keep misbehaving, the teacher will punish you to make an example out of you.}
            \end{quote}
        }
    \end{dictentry}
    \begin{dictentry}{aaCiC}{v.tr.}\label{word:aaCiC}
        \dictdef*{
            to ask \textit{sme.} for an example, to make \textit{sme.} give an example
            \begin{quote}
                \famword{pars aaSKiLana aaCiCibi.}\\
                \textit{The teacher asked him for an example.}
            \end{quote}
        }
    \end{dictentry}
    \begin{dictentry}{CurCi}{n.}\label{word:CurCi}
        \dictdef*{
            prototype, example of a physical product
        }
    \end{dictentry}
\end{dictroot}

\begin{dictroot}{c}{f}\label{root:C_F}
    \begin{dictentry}{CaF}{n.}\label{word:CaF}
        \dictdef{
            number
        }
        \dictdef{
            amount
        }
    \end{dictentry}
    \begin{dictentry}{CiiF}{v.tr.}\label{word:CiiF}
        \dictdef*{
            to number \textit{some amount}, to measure out to \textit{some amount}
            \begin{quote}
                \famword{naswi bar CiiF.}\\
                \textit{There are three of us. (lit., `We number three.')}
            \end{quote}
        }
    \end{dictentry}
    \begin{dictentry}{CiyaF}{v.intr.}\label{word:CiyaF}
        \dictdef*{
            to be counted, to be measured, to have one's quantity kept track of
            \begin{quote}
                \famword{[OXYGEN LEVEL] [CAREFUL]e\lilglot{} CiyaFe\lilglot{} tuuq}\\
                \textit{Oxygen levels must be carefully kept track of.}
            \end{quote}
        }
    \end{dictentry}
    \begin{dictentry}{iCaaF}{v.intr.}\label{word:iCaaF}
        \dictdef*{
            to count things, to measure things
            \begin{quote}
                \famword{nas iCaaFe\lilglot{} dak!}\\
                \textit{I can count!}
            \end{quote}
        }
    \end{dictentry}
    \begin{dictentry}{CiCiyaF}{v.tr.}\label{word:CiCiyaF}
        \dictdef*{
            to count \textit{smth.}, to measure \textit{smth.}, to keep track of \textit{smth.}'s quantity
            \begin{quote}
                \famword{nas [BLOOD SUGAR] CiCiyaFe\lilglot{} tuuq.}\\
                \textit{I need to measure my blood sugar.}
            \end{quote}
        }
    \end{dictentry}
    \begin{dictentry}{aaCiF}{v.tr.}\label{word:aaCiF}
        \dictdef*{
            to count out \textit{smth.}, to measure out \textit{smth.}, to set aside a quantity of \textit{smth.}
            \begin{quote}
                \famword{sesardam inPiDinilat KarLi aaCiCak.}\\
                \textit{Measure out three-hundred milliliters of water.}
            \end{quote}
        }
    \end{dictentry}
    \begin{dictentry}{CimiFu}{n.}\label{word:CimiFu}
        \dictdef*{
            measuring cup, a container for measuring the quantity of something by volume
        }
    \end{dictentry}
\end{dictroot}

\begin{dictroot}{c}{j}\label{root:C_J}
    %speed, fast, rush, urgent, horse
    \begin{dictentry}{CaJ}{n.}\label{word:CaJ}
        \dictdef*{
            speed
        }
    \end{dictentry}
    \begin{dictentry}{CiiY}{v.tr.}\label{word:CiiY}
        \dictdef*{
            to hurry over \textit{smth.}, to rush \textit{smth.}, to do \textit{smth.} in a hurry
        }
    \end{dictentry}
    \begin{dictentry}{CiyaJ}{v.intr.}\label{word:CiyaJ}
        \dictdef*{
            to be hurried, to be rushed, to be expedited
        }
    \end{dictentry}
    \begin{dictentry}{iCaaJ}{v.intr.}\label{word:iCaaJ}
        \dictdef*{
            to hurry, to rush, to make haste, to be quick about things
        }
    \end{dictentry}
    \begin{dictentry}{CiCiyaJ}{v.tr.}\label{word:CiCiyaJ}
        \dictdef*{
            to hasten \textit{smth.}, to cause \textit{smth.} to be rushed, to expedite, to cut corners on \textit{smth.}
        }
    \end{dictentry}
    \begin{dictentry}{aaCiY}{v.tr.}\label{word:aaCiY}
        \dictdef*{
            to be urgent to \textit{sme.}, to merit \textit{sme.}'s haste, cause \textit{sme.} to hurry over
            \begin{quote}
                \famword{[CLIMATE CHANGE] RaTa [WORLD] aaCiJibi}\\
                \textit{`Climate change is urgent for the entire world.'}
            \end{quote}
            }
    \end{dictentry}
    \begin{dictentry}{CuJu}{n.}\label{word:CuJu}
        \dictdef*{
            horse
        }
    \end{dictentry}
\end{dictroot}

\begin{dictroot}{c}{mp}\label{root:C_MP}
    %jump, kangaroo
    \begin{dictentry}{CiyaMP}{v.intr.}\label{word:CiyaMP}
        \dictdef*{
            to bounce
        }
    \end{dictentry}
    \begin{dictentry}{iCaaMP}{v.intr.}\label{word:iCaaMP}
        \dictdef*{
            to jump
        }
    \end{dictentry}
    \begin{dictentry}{CuMPu}{n.}\label{word:CuMPu}
        \dictdef*{
            kangaroo, wallaby
        }
    \end{dictentry}
\end{dictroot}

\begin{dictroot}{c}{n}\label{root:C_N}
    %to climb, to crawl, to scramble
    \begin{dictentry}{CiiN}{v.tr.}\label{word:CiiN}
        \dictdef*{
            to scale \textit{smth.}, climb up \textit{smth.}, to crawl along or through \textit{smth.}
        }
    \end{dictentry}
\end{dictroot}

\begin{dictroot}{c}{nk}\label{root:C_NK}
    \begin{dictentry}{CaNK}{n.}\label{word:CaNK}
        \dictdef*{
            roundness
        }
    \end{dictentry}
    \begin{dictentry}{CiiNK}{v.tr.}\label{word:CiiNK}
%we need more thought about this one here
        \dictdef*{
            to roll \textit{smth.} along
        }
    \end{dictentry}
%shouldn't these be CiyaNK and CiCiyaNK?
    \begin{dictentry}{CiyaNK}{v.intr.}\label{word:CiyaNK}
        \dictdef*{
            to roll, to tumble, to be rolled along
        }
    \end{dictentry}
    \begin{dictentry}{iCaaNK}{v.intr.}\label{word:iCaaNK}
        \dictdef{
            to tumble, to roll oneself along, to roll around
        }
    \end{dictentry}
    \begin{dictentry}{CaNKa}{adj.}\label{word:CaNKa}
        \dictdef{
            round
        }
        \dictdef{
            fat
        }
        \dictdef{
            full
        }
    \end{dictentry}
    \begin{dictentry}{uCiNKi}{adj.}\label{word:uCiNKi}
        \dictdef{
            feeling full, feeling fat, bloated
        }
    \end{dictentry}
\end{dictroot}

\begin{dictroot}{c}{r}\label{root:C_R}
    \begin{dictentry}{CanaR}{n.}\label{word:CanaR}
        \dictdef*{servant, maid, hired help, `the help'\\\textit{NB: derived from Danish \emph{tjener}, considered somewhat patronizing}}
    \end{dictentry}
\end{dictroot}

\begin{dictroot}{c}{w}\label{root:C_W}
    \begin{dictentry}{CiiW}{v.tr.}\label{word:CiiW}
        \dictdef*{
            to remember fondly, reminisce, be nostalgic about \emph{smth.} or \emph{sme.}
        }
    \end{dictentry}
\end{dictroot}

\begin{dictroot}{cl}{d}\label{root:CL_D}
    % to wear
    \begin{dictentry}{CLaD}{n.}\label{word:CLaD}
        \dictdef{
            clothing
        }
    \end{dictentry}
    \begin{dictentry}{CLiiD}{v.tr.}\label{word:CLiiD}
        \dictdef*{
            to wear \textit{smth.}
        }
    \end{dictentry}
    \begin{dictentry}{CLiyaD}{v.intr.}\label{word:CLiyaD}
        \dictdef*{
            \textit{(of clothing)} to be worn, to be put on
        }
    \end{dictentry}
    \begin{dictentry}{iCLaaD}{v.intr.}\label{word:iCLaaD}
        \dictdef*{
            to get dressed, to put clothes on
        }
    \end{dictentry}
    \begin{dictentry}{CLiCLiyaD}{v.tr.}\label{word:CLiCLiyaD}
        \dictdef*{
            \textit{(of a piece of clothing)} to put \textit{smth.} on, to get dressed in \textit{smth.}
        }
    \end{dictentry}
    \begin{dictentry}{aaCLiD}{v.tr.}\label{word:aaCLiD}
        \dictdef*{
            to dress \textit{sme.}
        }
    \end{dictentry}
\end{dictroot}

\begin{dictroot}{cm}{s}\label{root:CM_S}
    \begin{dictentry}{CMiiS}{v.tr}\label{word:CMiiS}
        \dictdef*{
            to cut, to slice
        }
    \end{dictentry}
    \begin{dictentry}{meCMiiS}{n.}\label{word:mCMiiS}
        \dictdef*{
            knife
        }
    \end{dictentry}
\end{dictroot}

%%%%%%%%%%%
%    D
%%%%%%%%%%%
\section*{D}

\begin{dictroot}{d}{l}\label{root:D_L}
    \begin{dictentry}{DaL}{n.}\label{word:DaL}
        \dictdef*{
            sight
        }
    \end{dictentry}  
    \begin{dictentry}{DiiL}{v.tr.}\label{word:DiiL}
        \dictdef{
            to stare intently at
        }
        \dictdef{
            to inspect, supervise
        }
    \end{dictentry}
    \begin{dictentry}{DiyaL}{v.intr.}\label{word:DiyaL}
        \dictdef*{
            to appear, seem, be perceived
        }
    \end{dictentry}
    \begin{dictentry}{iDaaL}{v.intr.}\label{word:iDaaL}
        \dictdef{
            to look, watch, stare
        }
        \dictdef{
            to supervise, watch over
        }
    \end{dictentry}
    \begin{dictentry}{DiDiyaL}{v.tr.}\label{word:DiDiyaL}
        \dictdef{
            to show, to display
        }
        \dictdef{
            to publish
        }
        \dictdef{
            to broadcast
        }
    \end{dictentry}
    \begin{dictentry}{aaDiL}{v.tr.}\label{word:aaDiL}
        \dictdef*{
            to draw \textit{sme.'s} attention, to make a scene
        }
    \end{dictentry}
    \begin{dictentry}{DaLa}{adj.}\label{word:DaLa}
        \dictdef*{
            visible
        }
    \end{dictentry}
    \begin{dictentry}{DanaL}{n.}\label{word:DanaL}
        \dictdef{
            audience, spectator, onlooker
        }
        \dictdef{
            supervisor
        }
    \end{dictentry}
    \begin{dictentry}{DurLi}{n.}\label{word:DurLi}
        \dictdef{
            eyeball
        }
        \dictdef{
            \textit{(fig.)} the apple of one's eye
        }
    \end{dictentry}
    \begin{dictentry}{DarLi}{n.}\label{word:DarLi}
        \dictdef*{
            tear, tears
        }
    \end{dictentry}
    \begin{dictentry}{iDuLa}{n.}\label{word:iDuLa}
        \dictdef{
            theater, movie theater
        }
        \dictdef{
            arena
        }
        \dictdef{
            stage
        }
    \end{dictentry}
    \begin{dictentry}{mDiL}{n.}\label{word:mDiL}
        \dictdef{
            lens
        }
        \dictdef{
            glasses
        }
    \end{dictentry}
    \begin{dictentry}{inDiL}{n.}\label{word:inDiL}
        \dictdef*{
            glance, brief look
        }
    \end{dictentry}
    \begin{dictentry}{DuliL}{n.}\label{word:DuliL}
        \dictdef*{
            eye, complex eye, compound eye, photosensitive tissue
        }
    \end{dictentry}
    %\begin{dictentry}{DuLu}{n.}
    %    \dictdef*{

    %    }
    %\end{dictentry}
    %\begin{dictentry}{DasiL}{n.}
    %    \dictdef*{

    %    }
    %\end{dictentry}
    %\begin{dictentry}{DajuLa}{n.}
    %    \dictdef*{
    %        mirror
    %    }
    %\end{dictentry}
    %\begin{dictentry}{DidiL}{n.}
    %    \dictdef*{

    %    }
    %\end{dictentry}
    \begin{dictentry}{aDiLu}{n.}\label{word:aDiLu}
        \dictdef{
            scene, performance
        }
        \dictdef{
            television program
        }
    \end{dictentry}
    %\begin{dictentry}{DimiLu}{n.}
    %    \dictdef*{

    %    }
    %\end{dictentry}
    %\begin{dictentry}{DuLi}{n., adj.}
    %    \dictdef*{

    %    }
    %\end{dictentry}
    %\begin{dictentry}{uDiLi}{adj.}
    %    \dictdef*{

    %    }
    %\end{dictentry}

\dictsubtitle{Compounds \& Secondary Derivations}
    \begin{dictentry}{HWaDa-mDiL}{n.}\label{word:HwaDa-mDiL}
        \dictdef{
            microscope
        }
    \end{dictentry}
    \begin{dictentry}{NaSamDiL}{n.}\label{word:NaSamDiL}
        \dictdef{
            optical (i.e. not outside visible spectrum) telescope, binoculars
        }
    \end{dictentry}
\end{dictroot}

\begin{dictroot}{d}{m}\label{root:D_M}
    % to meet, to come across
    % also to gather, to meet up
    \begin{dictentry}{DaM}{n.}\label{word:DaM}
        \dictdef{
            meeting, meet-up, gathering, rendezvous, date
        }
        \dictdef{
            introduction, meeting, first encounter
        }
        \dictdef{
            encounter, brush, confrontation
        }
    \end{dictentry}
    \begin{dictentry}{DiiM}{v.tr.}\label{word:DiiM}
        \dictdef*{
            to meet \textit{sme.}, to come across \textit{smth.}
        }
    \end{dictentry}
    \begin{dictentry}{DiyaM}{v.intr.}\label{word:DiyaM}
        \dictdef*{
            to be met, to be encountered, to be found, to be come across
        }
    \end{dictentry}
    \begin{dictentry}{iDaaM}{v.intr.}\label{word:iDaaM}
        \dictdef*{
            to meet up, to gather, to go somewhere for the purpose of meeting someone
        }
    \end{dictentry}
    \begin{dictentry}{DiDiyaM}{v.tr.}\label{word:DiDiyaM}
        \dictdef*{
            to introduce \textit{smth. or sme.}, to present \textit{smth. or sme.}
        }
    \end{dictentry}
    \begin{dictentry}{aaDiM}{v.tr.}\label{word:aaDiM}
        \dictdef*{
            to bring \textit{smth. or sme.} together, to rangle \textit{smth.} up, to herd together
        }
    \end{dictentry}
\dictsubtitle{Compounds \& Secondary Derivations}
    \begin{dictentry}{LaF-DaM}{n.}
        \dictdef*{
            romantic rendezvous, romantic date, meet-up for romantic purposes\\
            \textit{NB: more often just shortened to \famword{LaF} or \famword{DaM} in casual speech}
        }
    \end{dictentry}
\end{dictroot}

\begin{dictroot}{d}{n}\label{root:D_N}
    \begin{dictentry}{DanaN}{n.}\label{word:DanaN}
        \dictdef{
            Dane, Danish person
        }
        \dictdef{
            \textit{(slang)} cop, police
        }
    \end{dictentry}
    \begin{dictentry}{DuuN}{n.}\label{word:DuuN}
        \dictdef{
            Denmark
        }
        \dictdef{
            ethnically Danish people
        }
    \end{dictentry}
\end{dictroot}

\begin{dictroot}{dr}{j}\label{root:DR_J}
    \begin{dictentry}{DRaJ}{n.}\label{word:DRaJ}
        \dictdef*{
            magic, witchcraft
        }
        \dictdef*{
            a magic spell, a magical curse
        }
    \end{dictentry}
    \begin{dictentry}{DRiiY}{v.tr}\label{word:DRiiY}
        \dictdef*{
            to cast a spell on \textit{smth. or sme.}
        }
    \end{dictentry}
    \begin{dictentry}{DRanaJ}{n.}\label{word:DRanaJ}
        \dictdef*{
            witch, druid, sorcerer, wizard
        }
    \end{dictentry}
    \begin{dictentry}{DRuuJ}{n.}\label{word:DRuuJ}
        \dictdef{
            pagan people group(s), pagan people(s)
        }
        \dictdef{
            pagan, practicer of pagan religious practices
        }
    \end{dictentry}
\end{dictroot}

\begin{dictroot}{dr}{p}\label{root:DR_P}
    \begin{dictentry}{DRaP}{n.}\label{word:DRaP}
        \dictdef{%
        bad accent, mispronunciation
        }
        \dictdef{%
        funny voice, impression
            \begin{quote}
                \famword{buse wan DRaP lit.} \\
                \textit{`Their impression was really bad.'}
            \end{quote}
        }
        \dictdef{
            flub, speech error
        }
        \dictdef{
            misquote
        }
    \end{dictentry}
    \begin{dictentry}{DRiiP}{v.tr.}\label{word:DRiiP}
        \dictdef*{%
        to mimic \textit{sme.}, make fun of \textit{sme.}
        }
    \end{dictentry}
    \begin{dictentry}{DRiyaP}{v.intr.}\label{word:DRiyaP}
        \dictdef{
            to sound funny, to be mispronounced or said in a weird way
        }
        \dictdef{
            \textit{(fig.)} to be awkwardly worded, to be poorly put-together as a statement, to be a flub or speech error
        }
    \end{dictentry}
    \begin{dictentry}{iDRaaP}{v.intr.}\label{word:iDRaaP}
        \dictdef*{%
            to talk with an accent, talk in a funny voice
        }
    \end{dictentry}
    \begin{dictentry}{DRiDRiyaP}{v.tr.}\label{word:DRiDRiyaP}
        \dictdef{
            to mispronounce \textit{smth.}, to say \textit{smth.} in a funny way
        }
        \dictdef{
            \textit{(fig.)} to flub saying \textit{smth.}, to say \textit{smth.} poorly
        }
    \end{dictentry}
    \begin{dictentry}{aaDRiP}{v.tr.}\label{word:aaDRiP}
        \dictdef*{
            to misquote \textit{sme.}, to misrepresent what \textit{sme.} says
        }
    \end{dictentry}
\end{dictroot}

%%%%%%%%%%%
%    F
%%%%%%%%%%%
\section*{F}

\begin{dictroot}{f}{f}\label{root:F_F}
    \begin{dictentry}{FaF}{n.}\label{word:FaF}
        \dictdef*{scent, smell, fragrance}
    \end{dictentry}
    \begin{dictentry}{FiyaF}{v.intr.}\label{word:FiyaF}
        \dictdef*{to smell, to emit scent, to be fragrant}
    \end{dictentry}
\end{dictroot}

\begin{dictroot}{f}{l}\label{root:F_L}
    \begin{dictentry}{FaL}{n.}\label{word:FaL}
        \dictdef*{
            impression
        }
    \end{dictentry}
    \begin{dictentry}{FiiL}{v.tr.}\label{word:FiiL}
        \dictdef*{
            to notice \textit{smth.}
        }
    \end{dictentry}
    \begin{dictentry}{FiyaL}{v.intr.}\label{word:FiyaL}
        \dictdef*{
            to be distracted, have one's attention be diverted from something
        }
    \end{dictentry}
    \begin{dictentry}{iFaaL}{v.intr.}\label{word:iFaaL}
        \dictdef*{
            to be on the lookout, stay alert so as to catch an unplanned and spontaneous sensory impression
            \begin{quote}
                \famword{HWanaD wan BuRKu daw naswi iFaaL.} \\
                \textit{`We kept an eye out for the neighbor's dog.'}
            \end{quote}
            }
    \end{dictentry}
    \begin{dictentry}{FiFiyaL}{v.tr.}\label{word:FiFiyaL}
        \dictdef*{
            to distract \textit{sme.}, draw \textit{sme.'s} attention away from their current focus
        }
    \end{dictentry}
    \begin{dictentry}{aaFiL}{v.tr.}\label{word:aaFiL}
        \dictdef*{
            to divert \textit{sme.'s} attention towards something else, especially an incoming impression
            \begin{quote}
                \famword{bu MaNTa QasiH daw nas liWSaN dis aaFiLe{\lilglot} hwiili?} \\
                \textit{`Why didn't you warn me about that huge snake?'}
            \end{quote}
        }
    \end{dictentry}
    \begin{dictentry}{FanaL}{n.}\label{word:FanaL}
        \dictdef{
            watch, watchman
        }
        \dictdef{
            policeman, trooper
        }
    \end{dictentry}
    \begin{dictentry}{FuLaw}{n., adv.}\label{word:FuLaw}
        \dictdef{
            the left hand
        }
        \dictdef{
            to the left, on the left side
        }
    \end{dictentry}
\end{dictroot}

\begin{dictroot}{f}{m}\label{root:F_M}
    \begin{dictentry}{FaM}{n.}\label{word:FaM}
        \dictdef*{
            language, speech, way of speaking
        }
    \end{dictentry}
    \begin{dictentry}{FiiM}{v.tr.}\label{word:FiiM}
        \dictdef*{
            to say \textit{smth.}, to speak \textit{smth.}, to tell \textit{smth.}, to talk about \textit{smth.}
        }
    \end{dictentry}
    \begin{dictentry}{FiyaM}{v.intr.}\label{word:FiyaM}
        \dictdef*{
            to be said, to be distributed, to be published, to be broadcast
        }
    \end{dictentry}
    \begin{dictentry}{iFaaM}{v.intr.}\label{word:iFaaM}
        \dictdef*{
            to talk, to speak, to chatter
        }
    \end{dictentry}
    \begin{dictentry}{FiFiyaM}{v.tr.}\label{word:FiFiyaM}
        \dictdef*{
            to publish, to distribute, to broadcast
        }
    \end{dictentry}
    \begin{dictentry}{aaFiM}{v.tr.}\label{word:aaFiM}
        \dictdef*{
            to quote \textit{sme.}
        }
    \end{dictentry}
    \begin{dictentry}{iFuMa}{n.}\label{word:iFuMa}
        \dictdef{
            conversation
        }
        \dictdef{
            \textit{(internet)} server, board, forum
        }
    \end{dictentry}
    \begin{dictentry}{inFiM}{n.}\label{word:inFiM}
        \dictdef*{
            word
        }
    \end{dictentry}
    \begin{dictentry}{FuMi}{adj.}\label{word:FuMi}
        \dictdef*{
            wordy, dense, difficult to understand due to too many or too complicated words
        }
    \end{dictentry}
    \begin{dictentry}{FuliM}{n.}\label{word:FuliM}
        \dictdef{
            mouth
        }
        \dictdef{
            tongue
        }
    \end{dictentry}
    \begin{dictentry}{FuMu}{n.}\label{word:FuMu}
        \dictdef*{
            human, \textit{Homo sapiens sapiens}
        }
    \end{dictentry}
    \begin{dictentry}{FasiM}{n.}\label{word:FasiM}
        \dictdef*{
            thread (of a conversation, of a forum)
        }
    \end{dictentry}
    \begin{dictentry}{aFiMu}{n.}\label{word:aFiMu}
        \dictdef{
            book
        }
    \end{dictentry}
    \begin{dictentry}{FimiMu}{n.}\label{word:FimiMu}
        \dictdef{
            \textit{(of a radio or television network)} channel, station, frequency
        }
        \dictdef{
            \textit{(internet)} channel, chatroom
        }
    \end{dictentry}
\dictsubtitle{Compounds \& Secondary Derivations}
    \begin{dictentry}{inFiMini}{n.}\label{word:inFiMini}
        \dictdef*{
            letter, character, symbol
        }
    \end{dictentry}
    \begin{dictentry}{{\bigglot}a{\bigglot}a-FaM}{n.}\label{word:7a7a-FaM}
        \dictdef*{
            this language, \lang{}
        }
    \end{dictentry}
\end{dictroot}

\begin{dictroot}{f}{s}\label{root:F_S}
    \begin{dictentry}{FaS}{n.}\label{word:FaS}
        \dictdef{
            life, life force
        }
        \dictdef{
            birth
        }
    \end{dictentry}
    \begin{dictentry}{FiiS}{v.tr.}\label{word:FiiS}
        \dictdef{
            \textit{(of a human)} to give birth to \textit{sme.}
        }
        \dictdef{
            \textit{(of an animal that bears live young)} to give birth to its offspring
        }
        \dictdef{
            \textit{(of an animal that lays eggs)} to hatch its offspring
        }
    \end{dictentry}
    \begin{dictentry}{FiyaS}{v.intr.}\label{word:FiyaS}
        \dictdef*{
            to live, to be alive
        }
    \end{dictentry}
    \begin{dictentry}{iFaaS}{v.intr.}\label{word:iFaaS}
        \dictdef{
            to wake up, to awaken
        }
        \dictdef{
            to liven up, to become lively and energetic, to pep up
        }
    \end{dictentry}
    \begin{dictentry}{FiFiyaS}{v.tr.}\label{word:FiFiyaS}
        \dictdef*{
            to resuscitate \textit{sme.}
        }
    \end{dictentry}
    \begin{dictentry}{aaFiS}{v.tr.}\label{word:aaFiS}
        \dictdef{
            to wake \textit{sme.} up
        }
        \dictdef{
            to animate \textit{smth.}, bring life to \textit{smth}
        }
    \end{dictentry}
    \begin{dictentry}{FaSa}{adj.}\label{word:FaSa}
        \dictdef{
            alive
        }
        \dictdef{
            awake
        }
    \end{dictentry}
    \begin{dictentry}{uFiSi}{adj.}\label{word:uFiSi}
        \dictdef*{
            lively, spry
        }
    \end{dictentry}
    \begin{dictentry}{FanaS}{n.}\label{word:FanaS}
        \dictdef*{
            person, predominantly human or humanoid
        }
    \end{dictentry}
    \begin{dictentry}{FurSi}{n.}\label{word:FurSi}
        \dictdef*{
            egg
        }
    \end{dictentry}
    \begin{dictentry}{FarSi}{n.}\label{word:FarSi}
        \dictdef*{
            blood
        }
    \end{dictentry}
    \begin{dictentry}{iFuSa}{n.}\label{word:iFuSa}
        \dictdef{
            house
        }
        \dictdef{
            home, domicile
        }
    \end{dictentry}
    \begin{dictentry}{inFiS}{n.}\label{word:inFiS}
        \dictdef*{
            human offspring, especially newborn-through-toddler age
        }
    \end{dictentry}
    \begin{dictentry}{FuliS}{n.}\label{word:FuliS}
        \dictdef*{
            heart
        }
    \end{dictentry}
    \begin{dictentry}{FuSu}{n.}\label{word:FuSu}
        \dictdef*{
            animal, any species of the kingdom Animalia except humans
        }
    \end{dictentry}
\dictsubtitle{Compounds \& Secondary Derivations}
    \begin{dictentry}{JanaS-FaS}{n.}
        \dictdef*{Christmas, the birth of Christ}
    \end{dictentry}
\end{dictroot}

\begin{dictroot}{fr}{nc}\label{root:FR_NC}
    % France
    \begin{dictentry}{FRuuNC}{n.}\label{word:FRuuNC}
        \dictdef*{
            the Franks
        }
    \end{dictentry}
    \begin{dictentry}{FRaNCia}{n.}\label{word:FRaNCia}
        \dictdef*{
            France
        }
    \end{dictentry}
\end{dictroot}

%%%%%%%%%%%
%    H
%%%%%%%%%%%
\section*{H}

\begin{dictroot}{h}{h}\label{root:H_H}
    % to smoke
    \begin{dictentry}{HiiH}{v.tr.}\label{word:HiiH}
        \dictdef*{
            to smoke \textit{smth.}, to inhale smoke from burning \textit{smth.}
        }
    \end{dictentry}
    \begin{dictentry}{HiyaH}{v.intr.}\label{word:HiyaH}
        \dictdef{
            to be smoked, to be bathed in smoke, to be preserved by smoke
        }
        \dictdef{
            \textit{(fig.)} to inhale secondhand smoke
        }
    \end{dictentry}
    \begin{dictentry}{iHaaH}{v.intr.}\label{word:iHaaH}
        \dictdef{
            to smoke a tobacco or marijuana product, to be smoking, to deliberately inhale smoke
        }
        \dictdef{
            to emit smoke, to create smoke, to have smoke coming off of oneself
        }
    \end{dictentry}
    \begin{dictentry}{HiHiyaH}{v.tr.}\label{word:HiHiyaH}
        \dictdef{
            to smoke \textit{smth.}, to immerse in smoke, to cure by smoking
        }
        \dictdef{
            \textit{(fig.)} to smoke too near \textit{sme.}, to force \textit{sme.} to breathe in one's secondhand smoke
        }
    \end{dictentry}
    \begin{dictentry}{HaHa}{adj.}\label{word:HaHa}
        \dictdef*{
            smoky
        }
    \end{dictentry}
    \begin{dictentry}{HarHi}{n.}\label{word:HarHi}
        \dictdef*{
            smoke
        }
    \end{dictentry}
    \begin{dictentry}{HasiH}{n.}\label{word:HasiH}
        \dictdef*{
            plume of smoke
        }
    \end{dictentry}
    \begin{dictentry}{uHiHi}{adj.}\label{word:uHiHi}
        \dictdef*{
            confused, disoriented
        }
    \end{dictentry}
\end{dictroot}

\begin{dictroot}{h}{j}\label{root:H_J}
    \begin{dictentry}{HaJ}{n.}\label{word:HaJ}
        \dictdef*{
            light, brightness, illumination, luminosity
        }
    \end{dictentry}
    \begin{dictentry}{HiiY}{v.tr.}\label{word:HiiY}
        \dictdef{
            to illuminate, to brighten, to fill with light
        }
        \dictdef{
            \textit{(of a dwelling)} to move in \textit{smwh.}, to set up residence
        }
        \dictdef{
            \textit{(of a place of business)} to open \textit{smth.}, to begin business
        }
    \end{dictentry}
    \begin{dictentry}{HiyaJ}{v.intr.}\label{word:HiyaJ}
        \dictdef{
            to shine, to glow, to give out light
        }
        \dictdef{
            \textit{(impersonal)} to be bright out, to be sunny, to sun-shine, to be daylight, to be light out
            \begin{quote}
                \famword{HiyaJ, nemi iLaaSak!}\\
                \textit{The sun's shining, let's take a walk!}
            \end{quote}
        }
        \dictdef{
            \textit{(of a dwelling)} to have the lights on, to be currently filled with people going about their business
            \begin{quote}
                \famword{bu aFiSu HiyaJ FanaS irfituru.}\\
                \textit{That house has the lights on, someone must be inside.}
            \end{quote}
        }
        \dictdef{
            \textit{(of a place of business)} to be open, to be in operation, to be accepting customers
            \begin{quote}
                M\famword{AMA}-I\famword{NUMA se FS\,1 fun SJ\,3 daw iHaaJ}\\
                \textit{Mama-Inuma is open from 7\textsc{am} to 9\textsc{pm}.}
            \end{quote}
        }
        \dictdef{
            \textit{(of a device)} to be on, to be working
        }
    \end{dictentry}
    \begin{dictentry}{iHaaJ}{v.intr.}\label{word:iHaaJ}
        \dictdef{
            to turn on, to begin emitting light
        }
        \dictdef{
            \textit{(impersonal, of the sun)} to come out, to begin shining
        }
        \dictdef{
            to open, to begin business, to begin serving customers either for the day or in general
        }
        \dictdef{
            \textit{(of a device)} to turn on, to begin working
        }
    \end{dictentry}
    \begin{dictentry}{HiHiyaJ}{v.tr.}\label{word:HiHiyaJ}
        \dictdef{
            to cause to shed light, to light \textit{(a torch)}, to turn on \textit{(a lamp)}
            \begin{quote}
                \famword{wa PLaS fit HaTauru, mHiY HiHiyaJak.}\\
                \textit{It's dark in here, turn on the light.}
            \end{quote}
        }
    \end{dictentry}
    \begin{dictentry}{aaHiY}{v.tr.}\label{word:aaHiY}
        \dictdef{
            \textit{(of a device)} to turn \textit{smth.} on, to power \textit{smth.} up
            \begin{quote}
                \famword{nas mFiM aaHiYe\lilglot{} tuuq.}\\
                \textit{I need to turn on my phone.}
            \end{quote}
        }
    \end{dictentry}
    \begin{dictentry}{HaJa}{adj.}\label{word:HaJa}
        \dictdef{
            bright, light, glowing, alight
        }
        \dictdef{\textit{
            (of a place of business)} open, in operation, accepting customers
        }
        \dictdef{
            \textit{(of a device)} powered on, in operation, working
        }
    \end{dictentry}
    \begin{dictentry}{uHiYi}{adj.}\label{word:uHiYi}
        \dictdef*{
            dazzled, impressed, overwhelmed
        }
    \end{dictentry}
    \begin{dictentry}{HanaJ}{n.}\label{word:HanaJ}
        \dictdef*{
            the Sun
        }
    \end{dictentry}
    \begin{dictentry}{HurYi}{n.}\label{word:HurYi}
        \dictdef*{
            orb of light, as from around a torch, lantern, or other non-diffuse light source
        }
    \end{dictentry}
    \begin{dictentry}{HarYi}{n.}\label{word:HarYi}
        \dictdef{
            Aurora Borealis, the Northern Lights
        }
        \dictdef{
            \textit{(physics)} plasma, the state of matter consisting of partially ionized gas
        }
    \end{dictentry}
    \begin{dictentry}{iHuJa}{n.}\label{word:iHuJa}
        \dictdef*{
            day, daytime
        }
    \end{dictentry}
    \begin{dictentry}{mHiY}{n.}\label{word:mHiY}
        \dictdef{
            lamp, light \textit{(apparatus)}
        }
        \dictdef{
            projector
        }
    \end{dictentry}
    \begin{dictentry}{inHiY}{n.}\label{word:inHiY}
        \dictdef*{
            twinkle, sparkle
        }
    \end{dictentry}
    \begin{dictentry}{HuliY}{n.}\label{word:HuliY}
        \dictdef*{
            retina
        }
    \end{dictentry}
    \begin{dictentry}{HuJu}{n.}\label{word:HuJu}
        \dictdef*{
            firefly, lightning bug, glowworms, bioluminescent insect
        }
    \end{dictentry}
    \begin{dictentry}{HasiY}{n.}\label{word:HasiY}
        \dictdef{
            beam of light, sunbeam, ray of light
        }
        \dictdef{
            laser
        }
    \end{dictentry}
    \begin{dictentry}{HajuJa}{n.}\label{word:HajuJa}
        \dictdef{
            a field of light projected onto any unlit surface
            \begin{quote}
                \famword{CuSu HajuJa fit iYaaN.}\\
                \textit{The cat lay down to bask in the light.}
            \end{quote}
            }
        \dictdef{
            an image or moving images projected onto a screen, as in a cinema or office presentation
            \begin{quote}
                \famword{inFiM par HajuJa tui BaCauru\lilglot lit.}\\
                \textit{The words in that slide are very small.}
            \end{quote}
            }
        \dictdef{
            any backlit screen, as in a television or monitor
            \begin{quote}
                \famword{mFiM wan HajuJa daw fit nas iLaaS, was iHaaJe\lilglot{} kaje\lilglot{} hwii\lilglot{} wadan.}\\
                \textit{I stepped on my phone's screen, and now it won't turn on.}
            \end{quote}
            }
    \end{dictentry}
    \begin{dictentry}{HidiY}{n.}\label{word:HidiY}
        \dictdef*{
            photon
        }
    \end{dictentry}
    \begin{dictentry}{aHiYu}{n.}\label{word:aHiYu}
        \dictdef*{
            lantern, lightbulb
        }
    \end{dictentry}
    \begin{dictentry}{HimiYu}{n.}\label{word:HimiYu}
        \dictdef*{
            candle, torch
        }
    \end{dictentry}
    \begin{dictentry}{HuJaw}{n., adv.}\label{word:HuJaw}
        \dictdef{
            south
        }
        \dictdef{
            southwards, to the south, in the south
        }
        \dictdef{
            clockwise
        }
    \end{dictentry}
\dictsubtitle{Compounds \& Secondary Derivations}
    \begin{dictentry}{dajHuJa}{n.}\label{word:dajHuJa}
        \dictdef*{
            tomorrow
        }
    \end{dictentry}
    \begin{dictentry}{fajHuJa}{n.}\label{word:fajHuJa}
        \dictdef*{
            yesterday
        }
    \end{dictentry}
    \begin{dictentry}{wajHuJa}{n.}\label{word:wajHuJa}
        \dictdef*{
            today
        }
    \end{dictentry}
\end{dictroot}

\begin{dictroot}{h}{nk}\label{root:H_NK}
    % goose, honk, annoy
    \begin{dictentry}{HiiNK}{v.tr.}\label{word:HiiNK}
        \dictdef*{
            to annoy \textit{sme.}
        }
    \end{dictentry}
    \begin{dictentry}{iHaaNK}{v.intr.}\label{word:iHaaNK}
        \dictdef*{
            to honk
        }
    \end{dictentry}
    \begin{dictentry}{HaNKa}{adj.}\label{word:HaNKa}
        \dictdef*{
            annoying
        }
    \end{dictentry}
    \begin{dictentry}{HuNKu}{n.}\label{word:HuNKu}
        \dictdef*{
            goose
        }
    \end{dictentry}
\end{dictroot}

\begin{dictroot}{h}{r}\label{root:H_R}
    % rock
    \begin{dictentry}{HurRi}{n.}\label{word:HurRi}
        \dictdef*{
            rock
        }
    \end{dictentry}
\end{dictroot}

\begin{dictroot}{h}{t}\label{root:H_T}
    \begin{dictentry}{HaT}{n.}\label{word:HaT}
        \dictdef*{
            darkness
        }
    \end{dictentry}
    \begin{dictentry}{HiiT}{v.tr.}\label{word:HiiT}
        \dictdef{
            to darken, to shut out the light from, to shade
        }
        \dictdef{
            \textit{(of a dwelling)} to move out of \textit{smwh.}, to abandon
        }
        \dictdef{
            \textit{(of a place of business)} to close, to shut up
        }
    \end{dictentry}
    \begin{dictentry}{HiyaT}{v.intr.}\label{word:HiyaT}
        \dictdef{
            \textit{(impersonal)} to be dark out, to be murky
            \begin{quote}
                \famword{NaRKa-JaMJ fit RaTa iHuJa HiyaT.}\\
                \textit{`During the winter it's dark all day.'}
            \end{quote}
        }
        \dictdef{
            \textit{(of a dwelling)} to be abandoned, to have nobody home
        }
        \dictdef{
            \textit{(of a place of business)} to be closed, to be shut down, to not be accepting customers
        }
        \dictdef{
            \textit{(of a device)} to be turned off, to not be running, to not be in operation
        }
    \end{dictentry}
    \begin{dictentry}{iHaaT}{v.intr.}\label{word:iHaaT}
        \dictdef{
            to turn off, to cease emitting light
        }
        \dictdef{
            \textit{(impersonal)} to get dark, to stop being bright and sunny out
        }
        \dictdef{
            \textit{(of a place of business)} to close, to cease business, to shut down either for the day or in general
        }
        \dictdef{
            \textit{(of a device)} to turn off, to stop working
        }
    \end{dictentry}
    \begin{dictentry}{HiHiyaT}{v.tr.}\label{word:HiHiyaT}
        \dictdef*{
            to darken, to prevent from casting light, to put out \textit{(a torch or candle)}, to turn off \textit{(a lamp)}
        }
    \end{dictentry}
    \begin{dictentry}{aaHiT}{v.tr.}\label{word:aaHiT}
        \dictdef{
            \textit{(of a device)} to turn off, to shut down
        }
        \dictdef{
            \textit{(of a place of business)} to shut down, to close up, to end the operations of either for the day or in general
        }
    \end{dictentry}
    \begin{dictentry}{HaTa}{adj.}\label{word:HaTa}
        \dictdef{
            dark, without light
        }
        \dictdef{
            \textit{(of a dwelling)} abandoned, empty
        }
        \dictdef{
            \textit{(of a place of business)} closed, shut down
        }
        \dictdef{
            \textit{(of a device)} off, powered down, dead
        }
    \end{dictentry}
    \begin{dictentry}{HanaT}{n.}\label{word:HanaT}
        \dictdef*{
            the moon
        }
    \end{dictentry}
    \begin{dictentry}{iHuTa}{n.}\label{word:iHuTa}
        \dictdef*{
            night, nighttime
        }
    \end{dictentry}
    \begin{dictentry}{HuTi}{adj., n.}\label{word:HuTi}
        \dictdef*{
            black, the color black
        }
    \end{dictentry}
    \begin{dictentry}{HajuTa}{n.}\label{word:HajuTa}
        \dictdef*{
            curtains, shades, blinds
        }
    \end{dictentry}
\end{dictroot}

\begin{dictroot}{hc}{b}\label{root:HC_B}
    \begin{dictentry}{HeCanaB}{n.}\label{word:HCanaB}
        \dictdef*{
            hijabi, Muslim woman who wears a headscarf
        }
    \end{dictentry}
    \begin{dictentry}{HeCaBjalaB}{n.}\label{word:HCaBjalaB}
        \dictdef*{
            hijab, Muslim headscarf
        }
    \end{dictentry}
\end{dictroot}

%%%%%%%%%%%
%    J
%%%%%%%%%%%
\section*{J}

\begin{dictroot}{j}{b}\label{root:J_B}
    \begin{dictentry}{JaB}{n.}\label{word:JaB}
        \dictdef*{
            goodness, prosperity, good vibes, good luck
        }
    \end{dictentry}
    \begin{dictentry}{YiiB}{v.tr.}\label{word:YiiB}
        \dictdef{
            to treat \textit{sme.} well, to be kind to \textit{sme.}
        }
        \dictdef{
            to improve \textit{smth.}, to make \textit{smth.} better, to renovate \textit{smth.}, to overhaul \textit{smth.}, to spruce up
        }
    \end{dictentry}
    \begin{dictentry}{YiyaB}{v.intr.}\label{word:YiyaB}
        \dictdef*{
            to have good fortune, to be lucky
        }
    \end{dictentry}
    \begin{dictentry}{iYaaB}{v.intr.}\label{word:iYaaB}
        \dictdef*{
            to be good, to behave oneself, to do good deeds
        }
    \end{dictentry}
    \begin{dictentry}{YiYiyaB}{v.tr.}\label{word:YiYiyaB}
        \dictdef*{
            to bless \textit{sme.}, to bestow good fortune upon \textit{sme.}, to give \textit{sme.} a blessing
        }
    \end{dictentry}
    \begin{dictentry}{aaYiB}{v.tr.}\label{word:aaYiB}
        \dictdef*{
            to be a good influence on \textit{sme.}, to cause \textit{sme.} to behave well, to encourage \textit{sme.}
        }
    \end{dictentry}
    \begin{dictentry}{JaBa}{adj.}\label{word:JaBa}
        \dictdef*{
            good, fortunate, lucky, blessed
        }
    \end{dictentry}
    \begin{dictentry}{uYiBi}{adj.}\label{word:uYiBi}
        \dictdef*{
            happy, blessed, enthusiastic, amused
        }
    \end{dictentry}
    \begin{dictentry}{JanaB}{n.}\label{word:JanaB}
        \dictdef*{
            friend, buddy, pal, companion
        }
    \end{dictentry}
\dictsubtitle{Compounds \& Secondary Derivations}
    \begin{dictentry}{JaBaila}{v.tr.}\label{word:JaBaila}
        \dictdef*{
            to like \textit{smth.}, to love \textit{smth.} \textit{(used of inanimate objects)}, to appreciate, to enjoy
            \begin{quote}
                \famword{CurKLi bus hwaJaBaila, iNaaMuc BiiN.}\\
                \textit{`He must love chocolate, he makes a habit of eating it.'}
            \end{quote}
        }
    \end{dictentry}
\end{dictroot}

\begin{dictroot}{j}{l}\label{root:J_L}
    % many
    \begin{dictentry}{JaL}{n.}\label{word:JaL}
        \dictdef*{
            many, multitude, much, a large amount
        }
    \end{dictentry}
    \begin{dictentry}{JaLa}{adj.}\label{word:JaLa}
        \dictdef*{
            many, a lot of, quite a few
        }
    \end{dictentry}
\end{dictroot}

\begin{dictroot}{j}{mj}\label{root:J_MJ}
    % season
    \begin{dictentry}{JaMJ}{n.}\label{word:JaMJ}
        \dictdef*{
            season
        }
    \end{dictentry}
    %TODO: add the seasons of the traditional (and post-IE) 7a7a-FaM year
\end{dictroot}

\begin{dictroot}{j}{n}\label{root:J_N}
    \begin{dictentry}{YiiN}{v.tr.}\label{word:YiiN}
        \dictdef*{
            to take a break from \textit{smth.}, to take a vacation from \textit{smth.}
        }
    \end{dictentry}
    \begin{dictentry}{YiyaN}{v.intr.}\label{word:YiyaN}
        \dictdef*{
            to rest, to relax, to be resting, to destress
        }
    \end{dictentry}
    \begin{dictentry}{iYaaN}{v.intr.}\label{word:iYaaN}
        \dictdef{to wait}
        \dictdef{to lie down, to relax, to begin napping, to lie down with the intent to relax}
    \end{dictentry}
    \begin{dictentry}{YiYiyaN}{v.tr.}\label{word:YiYiyaN}
        \dictdef*{
            to relax \textit{sme.}, to calm \textit{sme.} down, to destress \textit{sme.}
        }
    \end{dictentry}
    \begin{dictentry}{aaYiN}{v.tr.}\label{word:aaYiN}
        \dictdef{
            to give \textit{sme.} a break, to give \textit{sme.} a vacation
        }
        \dictdef{
            to pause \textit{smth.}
        }
    \end{dictentry}
\end{dictroot}

\begin{dictroot}{j}{ps}\label{root:J_PS}
    % sibling (distance)
    \begin{dictentry}{JanaPS}{n.}\label{word:JanaPS}
        \dictdef{
            sibling considered less close than another sibling, for example due to large age gap or being step-siblings
        }
        \dictdef{
            cousin, child of parent's sibling
        }
    \end{dictentry}
\end{dictroot}

\begin{dictroot}{j}{s}\label{root:J_S}
    % god, spirit, ghost, holy, magical
    \begin{dictentry}{JaS}{n.}\label{word:JaS}
        \dictdef*{
            God
        }
    \end{dictentry}
    \begin{dictentry}{JaSa}{adj.}\label{word:JaSa}
        \dictdef{
            holy
        }
        \dictdef{
            magical
        }
    \end{dictentry}
    \begin{dictentry}{JanaS}{n.}\label{word:JanaS}
        \dictdef*{
            Jesus
        }
    \end{dictentry}
\end{dictroot}

\begin{dictroot}{j}{t}\label{root:J_T}
    % yeet, throw, travel
    \begin{dictentry}{YiiT}{v.tr.}\label{word:YiiT}
        \dictdef{
            to throw \textit{smth.}
        }
        \dictdef{
            to give \textit{smth.} (to someone)
            \begin{quote}
                \famword{nas LanaF daw CurKLi YiiT.}\\
                \textit{`I gave chocolates to my romantic partner.'}
            \end{quote}
        }
    \end{dictentry}
\end{dictroot}

\begin{dictroot}{j}{w}\label{root:J_W}
    \begin{dictentry}{JaW}{n.}\label{word:JaW}
        \dictdef*{
            memory, recollection, something that is remembered
        }
    \end{dictentry}
    \begin{dictentry}{YiiW}{v.tr.}\label{word:YiiW}
        \dictdef*{
            to remember, to recall, to recollect
        }
    \end{dictentry}
    \begin{dictentry}{aYiMu}{n.}\label{word:aYiMu}
        \dictdef*{
            one's memory, the figurative container for one's memories
        }
    \end{dictentry}
\end{dictroot}

\begin{dictroot}{jp}{n}\label{root:JP_N}
    \begin{dictentry}{JPuuN}{n.}\label{word:JPuuN}
        \dictdef*{
            the Japanese people or ethnicity
        }
    \end{dictentry}
    \begin{dictentry}{JPaNia}{n.}\label{word:JPaNia}
        \dictdef*{
            Japan
        }
    \end{dictentry}
\end{dictroot}

\begin{dictroot}{jw}{rj}\label{root:JW_RJ}
    %Irish, Ireland
    \begin{dictentry}{JWaRJa}{adj.}\label{JWaRJa}
        \dictdef*{
            Irish, from Ireland
        }
    \end{dictentry}
    \begin{dictentry}{iYWuRJa}{n.}\label{word:iYWuRJa}
        \dictdef{
            Ireland, the island of Ireland in the British Isles upon which both the modern states of Ireland and Northern Ireland are located
        }
        \dictdef{
            \textit{(attested but considered incorrect)} the Republic of Ireland
        }
    \end{dictentry}
    \begin{dictentry}{JWuuRJ}{n.}\label{word:JWuuRJ}
        \dictdef*{
            Irish people, the Irish as a people group
        }
    \end{dictentry}
    \begin{dictentry}{JWaRYia}{n.}\label{word:JWaRJia}
        \dictdef*{
            the Republic of Ireland, Southern Ireland
        }
    \end{dictentry}
\dictsubtitle{Compounds \& Secondary Derivations}
    \begin{dictentry}{NurKaw-iYWuRJa}{n.}\label{word:NurKaw-iYWuRJa}
        \dictdef*{
            Northern Ireland
            }
    \end{dictentry}
\end{dictroot}

%%%%%%%%%%%
%    K
%%%%%%%%%%%
\section*{K}

\begin{dictroot}{k}{j}\label{root:K_J}
    % write
    \begin{dictentry}{KaJ}{n.}
        \dictdef*{
            writing, text
        }
    \end{dictentry}
    \begin{dictentry}{KiiY}{v.tr.}\label{word:KiiY}
        \dictdef*{
            to write \textit{smth.}, to write \textit{smth.} down
        }
    \end{dictentry}
    \begin{dictentry}{KiyaJ}{v.intr.}\label{word:KiyaJ}
        \dictdef{
            to be written, to be written down, to be published, to be printed
        }
        \dictdef{
            \textit{(of an instant message)} to be sent
        }
    \end{dictentry}
    \begin{dictentry}{iKaaJ}{v.intr.}\label{word:iKaaJ}
        \dictdef*{
            to write, to write something, to do some writing
        }
    \end{dictentry}
    \begin{dictentry}{KiKiyaJ}{v.tr.}\label{word:KiKiyaJ}
        \dictdef{
            to publish \textit{smth.}, to print \textit{smth.}
        }
        \dictdef{
            to sent \textit{(an instant message)}
        }
    \end{dictentry}
    \begin{dictentry}{aaKiY}{v.tr.}\label{word:aaKiY}
        \dictdef*{
            \textit{(of a contract, law, or equivalent written accord)} to sign, to ratify, to `put on paper', to officially commit oneself to
        }
        \dictdef*{
            \textit{(figurative)} to agree to \textit{smth.}, to promise \textit{smth.}
        }
    \end{dictentry}
    \begin{dictentry}{KanaJ}{n.}\label{word:KanaJ}
        \dictdef*{
            writer, author, scribe
        }
    \end{dictentry}
    \begin{dictentry}{KarYi}{n.}\label{word:KarYi}
        \dictdef*{
            ink
        }
    \end{dictentry}
    \begin{dictentry}{mKiY}{n.}\label{word:mKiY}
        \dictdef*{
            pen, pencil, writing implement
        }
    \end{dictentry}
    \begin{dictentry}{KuliY}{n.}\label{word:KuliY}
        \dictdef*{
            hand, fingers
        }
    \end{dictentry}
    \begin{dictentry}{KajuJa}{n.}\label{word:KajuJa}
        \dictdef*{
            paper, sheet of paper meant for writing
        }
    \end{dictentry}
    \begin{dictentry}{aKiYu}{n.}\label{word:aKiYu}
        \dictdef*{
            book, scroll, tome, text
        }
    \end{dictentry}
\end{dictroot}

\begin{dictroot}{k}{k}\label{root:K_K}
    % cookie, biscuit, cake?
    \begin{dictentry}{KaK}{n.}\label{word:KaK}
        \dictdef*{
            an HTTP cookie, a token sent between a server and a web browser to maintain a state between HTTP transactions
        }
    \end{dictentry}
    \begin{dictentry}{KuRKi}{n.}\label{word:KurKi}
        \dictdef*{
            small baked good, cookie, biscuit
        }
    \end{dictentry}
    \begin{dictentry}{KarKi}{n.}\label{word:KarKi}
        \dictdef*{
            cookie dough, cookie batter
        }
    \end{dictentry}
    \begin{dictentry}{KasiK}{n.}\label{word:KasiK}
        \dictdef{
            long \& thin cookie (e.g., biscotti)
        }
        \dictdef{
            \textit{(figurative, slang)} penis
        }
    \end{dictentry}
    \begin{dictentry}{KajuKa}{n.}\label{word:KajuKa}
        \dictdef*{
            thin \& flat cookie (e.g., pizelle, stroopwaffel)
        }
    \end{dictentry}
\end{dictroot}

\begin{dictroot}{k}{l}\label{root:K_L}
    \begin{dictentry}{KaL}{n.}\label{word:KaL}
        \dictdef*{
            humidity, wetness, dampness
            \begin{quote}
                \famword{iFuSa JaBauru, da LajuSa KaLila.}\\
                \textit{`The house is lovely, but the floors are damp.'}
            \end{quote}
        }
    \end{dictentry}
    \begin{dictentry}{iKaaL}{v.intr.}\label{word:iKaaL}
        \dictdef{
            \textit{(impersonal)} to be a rainy day
            \begin{quote}
                \famword{wajHuJa iKaaL.}\\
                \textit{`Today's a rainy day.'}
            \end{quote}
        }
        \dictdef{
            \textit{(impersonal)} to be raining
            \begin{quote}
                \famword{iMuNTa daw nas iYaaTe\lilglot{} kaj da bus fit iKaaL.}\\
                \textit{`I wanted to go to the mountains, but it's raining there.'}
            \end{quote}
        }
    \end{dictentry}
    \begin{dictentry}{KaLa}{adj.}\label{word:KaLa}
        \dictdef{
            covered in water, saturated with water, wet, soaked
            \begin{quote}
                \famword{KaLa mLiS SaFRaariak.}\\
                \textit{`Warm up your wet shoes.'}
            \end{quote}
        }
        \dictdef{
            fluid, liquid, melted
            \begin{quote}
                \famword{KasiL KaLa CurKLiakiuru.} \\
                \textit{`The river was made of melted chocolate'}
                %KNUT PLEASE WEIGH IN ON THIS
                %I LIKE IT THIS IS GOOD -K
            \end{quote}
        }
    \end{dictentry}
    \begin{dictentry}{uKiLi}{adj.}\label{word:uKiLi}
        \dictdef{
            cold from being wet, soaked, shivering
        }
        \dictdef{
            \textit{(fig.)} anxious, uncomfortable, shaky
        }
    \end{dictentry}
    \begin{dictentry}{KurLi}{n.}\label{word:KurLi}
        \dictdef*{
            the Earth, the globe
        }
    \end{dictentry}
    \begin{dictentry}{KarLi}{n.}\label{word:KarLi}
        \dictdef*{
            liquid water, fresh water, water not part of a body of water or stream, water served as a beverage
            \begin{quote}
                \famword{mi NarKa KarLiilali?}\\
                \textit{`Do you have any cold water?'}
            \end{quote}
        }
    \end{dictentry}
    \begin{dictentry}{KasiL}{n.}\label{word:KasiL}
        \dictdef{
            river, stream
        }
        \dictdef{
            stream or sprinkle of water, as from a faucet or tap
            \begin{quote}
                \famword{iSuNa BaBa KasiLila.}\\
                \textit{`The shower is low-flow'}\\
                (lit., \textit{`The shower has a mild stream'})
            \end{quote}
        }
    \end{dictentry}
    \begin{dictentry}{KajuLa}{n.}\label{word:KajuLa}
        \dictdef{
            the surface of a body of water
            \begin{quote}
                \famword{naswi KajuLa tui FiiLami\lilglot{} dak.}\\
                \textit{`We could see ourselves on the water's surface.'}
            \end{quote}
        }
        \dictdef{
            puddle
            \begin{quote}
                \famword{nas KajuLa daw tui iLaaS.}\\
                \textit{`I stepped in a puddle.'}
            \end{quote}
        }
        \dictdef{
            map
            \begin{quote}
                \famword{par KajuLa SWuuTilali?}\\
                \textit{`Is Switzerland on that map?'}
            \end{quote}
        }
    \end{dictentry}
    \begin{dictentry}{KidiL}{n.}\label{word:KidiL}
        \dictdef*{
            salt, sea salt
        }
    \end{dictentry}
    \begin{dictentry}{KuLu}{n.}\label{word:KuLu}
        \dictdef*{
            fish, fish-adjacent aquatic animal
        }
    \end{dictentry}
    \begin{dictentry}{KuLi}{adj., n.}\label{word:KuLi}
        \dictdef*{
            blue, the color blue
        }
    \end{dictentry}
    \begin{dictentry}{aKiLu}{n.}\label{word:aKiLu}
        \dictdef*{
            bottle
            \begin{quote}
                \famword{ha MARK kaj aKiLulat MarHiila.}\\
                \textit{`Mark owned five bottles of milk.'}
            \end{quote}
        }
    \end{dictentry}
    \begin{dictentry}{KimiLu}{n.}\label{word:KimiLu}
        \dictdef*{
            drinking glass
        }
    \end{dictentry}
    \begin{dictentry}{KuLaw}{n., adv.}\label{word:KuLaw}
        \dictdef{
            east
        }
        \dictdef{
            eastwards, to the east, in the east
        }
    \end{dictentry}
\dictsubtitle{Compounds \& Secondary Derivations}
    \begin{dictentry}{KasiLiri}{v.intr.}\label{word:KasiLiri}
        \dictdef*{
            \textit{(euphemistic)} to pee
        }
    \end{dictentry}
\end{dictroot}

\begin{dictroot}{k}{n}\label{root:K_N}
    % to touch
    % to recognize
    \begin{dictentry}{KaN}{n.}\label{word:KaN}
        \dictdef{
            a touch, a moment of contact
        }
        \dictdef{
            a moment of recognition
        }
        \dictdef{
            recognition, the ability to recognize things, authentication
        }
    \end{dictentry}
    \begin{dictentry}{KiiN}{v.tr.}\label{word:KiiN}
        \dictdef{
            to touch \textit{smth.} or \textit{sme.}
        }
        \dictdef{
            to recognize \textit{sme.}, to identify \textit{sme.}
        }
    \end{dictentry}
    \begin{dictentry}{KiyaN}{v.intr.}\label{word:KiyaN}
        \dictdef*{
            to be recognizable, to be recognized, to be known, to be familiar
        }
    \end{dictentry}
    \begin{dictentry}{iKaaN}{v.intr.}\label{word:iKaaN}
        \dictdef{
            to identify oneself, to make oneself known, to produce identification for oneself
        }
        \dictdef{
            to present oneself to the public, to perform
        }
        \dictdef{
            to be famous
        }
    \end{dictentry}
    \begin{dictentry}{KiKiyaN}{v.tr.}\label{word:KiKiyaN}
        \dictdef*{
            to identify \textit{sme.}, to establish \textit{sme.'s} identity
        }
    \end{dictentry}
    \begin{dictentry}{aaKiN}{v.tr.}\label{word:aaKiN}
        \dictdef*{
            to promote, to advertise
        }
    \end{dictentry}
    \begin{dictentry}{KuliN}{n.}\label{word:KuliN}
        \dictdef*{
            face
        }
    \end{dictentry}
    \begin{dictentry}{KajuNa}{n.}\label{word:KajuNa}
        \dictdef*{
            ID, card or documents for identification
        }
    \end{dictentry}
\end{dictroot}

\begin{dictroot}{k}{p}\label{root:K_P}
    % marriage, spouse
    \begin{dictentry}{KiiP}{v.tr.}\label{word:KiiP}
        \dictdef*{
            to marry \textit{sme.}, to officiate \textit{sme.'s} wedding, to perform \textit{sme.'s} marriage ceremony
        }
    \end{dictentry}
    \begin{dictentry}{iKaaP}{v.intr.}\label{word:iKaaP}
        \dictdef*{
            to hold a wedding, to have a marriage party\\
            \textit{NB: `to get married to X' is usually expressed in \lang{} by saying \famword{X-la iKaaP} (lit., `to hold a wedding with X'.)}
        }
    \end{dictentry}
    \begin{dictentry}{KiyaP}{v.intr.}\label{word:KiyaP}
        \dictdef*{
            to be married, to get married
        }
    \end{dictentry}
    \begin{dictentry}{KiKiyaP}{v.tr.}\label{word:KiKiyaP}
        \dictdef*{
            to marry \textit{sme.} off, to give \textit{sme.} away in marriage
        }
    \end{dictentry}
    \begin{dictentry}{aaKiP}{v.tr.}\label{word:aaKiP}
        \dictdef{
            to arrange \textit{sme.'s} marriage, to arrange for \textit{sme.} to get married
        }
        \dictdef{
            \textit{(fig.)} to set \textit{sme.} up, to arrange a blind date for \textit{sme.}
        }
    \end{dictentry}
    \begin{dictentry}{KanaP}{n.}\label{word:KanaP}
        \dictdef*{
            spouse, husband, wife, groom, bride, life partner
        }
    \end{dictentry}
    \begin{dictentry}{iKuPa}{n.}\label{word:iKuPa}
        \dictdef*{
            wedding, marriage celebration
        }
    \end{dictentry}
\end{dictroot}

\begin{dictroot}{k}{pk}\label{root:K_PK}
    % fiancé(e), engagement
    \begin{dictentry}{KiiPK}{v.tr.}\label{word:KiiPK}
        \dictdef*{
            to propose to \textit{sme.}, to ask \textit{sme.} for their hand in marriage.
        }
    \end{dictentry}
    \begin{dictentry}{iKaaPK}{v.intr.}\label{word:iKaaPK}
        \dictdef*{
            \textit{(of a couple)} to get engaged, to announce one's engagement, to decide to get married
        }
    \end{dictentry}
    \begin{dictentry}{KiyaPK}{v.intr.}\label{word:KiyaPK}
        \dictdef*{
            to agree to be engageds to \textit{}
        }
    \end{dictentry}
    \begin{dictentry}{KanaPK}{n.}\label{word:KanaPK}
        \dictdef*{
            fiancé, fiancée
        }
    \end{dictentry}
\end{dictroot}

\begin{dictroot}{k}{t}\label{root:K_T}
    % Scotland, Scottish (from the Caits/Cats/Catts)
    \begin{dictentry}{KarTi}{n.}\label{word:KarTi}
        \dictdef*{
            Scotch, Scotch whiskey
        }
    \end{dictentry}
    \begin{dictentry}{KuuT}{n.}\label{word:KuuT}
        \dictdef{
            the Scots, the Scottish people
        }
        \dictdef{
            Scotland
        }
    \end{dictentry}
\end{dictroot}

\begin{dictroot}{k}{w}\label{root:K_W}
    \begin{dictentry}{KaW}{n.}\label{word:KaW}
        \dictdef*{
            growth
        }
    \end{dictentry}
    \begin{dictentry}{KiiW}{v.tr.}\label{word:KiiW}
        \dictdef{
            to plant \textit{smth.}, to sow \textit{smth.}
        }
        \dictdef{
            to conceive \textit{sme.}
        }
        \dictdef{
            \textit{(fig.)} to come up with \textit{smth.}, to conceive of \textit{smth.}, to think up \textit{smth.}
        }
    \end{dictentry}
    \begin{dictentry}{KiyaW}{v.intr.}\label{word:KiyaW}
        \dictdef{
            to grow, to increase in size
        }
        \dictdef{
            to mature, to grow up, to reach maturity, to come of age
        }
    \end{dictentry}
    \begin{dictentry}{iKaaW}{v.intr.}\label{word:iKaaW}
        \dictdef{
            to sow seeds, to plant crops
        }
        \dictdef{
            to conceive a child, to have sex with the intent to conceive children
        }
        \dictdef{
            \textit{(fig.)} to brainstorm
        }
    \end{dictentry}
    \begin{dictentry}{KiKiyaW}{v.tr.}\label{word:KiKiyaW}
        \dictdef{
            to grow \textit{smth.}, to tend \textit{smth.}, to raise \textit{smth.}
        }
        \dictdef{
            to raise \textit{sme.}, to care for \textit{sme.}, to read \textit{sme.}, to bring \textit{sme.} up
        }
    \end{dictentry}
    \begin{dictentry}{aaKiW}{v.tr.}\label{word:aaKiW}
        \dictdef{
            to fertilize \textit{smth.}, to add nutrients to \textit{smth.} to encourage the growth of crops
        }
        \dictdef{
            to fertilize \textit{(an ovum)}, to pollenate
        }
        \dictdef{
            to inspire \textit{sme.}, to give \textit{sme.} an idea
        }
    \end{dictentry}
    \begin{dictentry}{KaWa}{adj.}\label{word:KaWa}
        \dictdef{
            fully grown, mature, adult
        }
        \dictdef{
            grown-up, mature, serious
        }
    \end{dictentry}
    \begin{dictentry}{uKiWi}{adj.}\label{word:uKiWi}
        \dictdef{
            feeling enriched, like one is growing as a person, self-fulfilled
        }
        \dictdef{
            stoned, high (on weed)
        }
    \end{dictentry}
    \begin{dictentry}{KanaW}{n.}\label{word:KanaW}
        \dictdef*{
            adult, grown up
        }
    \end{dictentry}
    \begin{dictentry}{KurWi}{n.}\label{word:KurWi}
        \dictdef*{
            seed, bulb
        }
    \end{dictentry}
    \begin{dictentry}{KarWi}{n.}\label{word:KarWi}
        \dictdef*{
            tea
        }
    \end{dictentry}
    \begin{dictentry}{iKuWa}{n.}\label{word:iKuWa}
        \dictdef*{
            forest, woods, grove
        }
    \end{dictentry}
    \begin{dictentry}{mKiW}{n.}\label{word:mKiW}
        \dictdef{
            stake, trellis, etc. used to support a growing plant
        }
        \dictdef{
            cane, walker, walking stick
        }
    \end{dictentry}
    \begin{dictentry}{inKiW}{n.}\label{word:inKiW}
        \dictdef*{
            seedling, sapling, baby plant
        }
    \end{dictentry}
    \begin{dictentry}{KuWi}{adj., n.}\label{word:KuWi}
        \dictdef*{
            green, the color green
        }
    \end{dictentry}
    \begin{dictentry}{KuliW}{n.}\label{word:KuliW}
        \dictdef*{
            shoulder
        }
    \end{dictentry}
    \begin{dictentry}{KuWu}{n.}\label{word:KuWu}
        \dictdef{
            frog, toad
        }
        \dictdef{
            \textit{(fig., slang)} pothead, smoker
        }
    \end{dictentry}
    \begin{dictentry}{KasiW}{n.}\label{word:KasiW}
        \dictdef{
            plant
        }
        \dictdef{
            tree
        }
        \dictdef{\textit{
            (fig., slang)} joint
        }
    \end{dictentry}
    \begin{dictentry}{KajuWa}{n.}\label{word:KajuWa}
        \dictdef{
            leaf
        }
        \dictdef{
            tea leaves, looseleaf tea
        }
        \dictdef{
            \textit{(slang)} tobacco or cannabis
        }
    \end{dictentry}
    \begin{dictentry}{KidiW}{n.}\label{word:KidiW}
        \dictdef{
            dried and/or powdered plant matter used for any purpose
        }
        \dictdef{
            ground or powdered tea, as in matcha or inside tea bags
        }
        \dictdef{
            ground tobacco or cannabis for smoking
        }
    \end{dictentry}
    \begin{dictentry}{aKiWu}{n.}\label{word:aKiWu}
        \dictdef*{
            garden, greenhouse, arbor
        }
    \end{dictentry}
    \begin{dictentry}{KimiWu}{n.}\label{word:KimiWu}
        \dictdef*{
            pot, plot, planter
        }
    \end{dictentry}
\dictsubtitle{Compounds \& Secondary Derivations}
    \begin{dictentry}{KiyaWana}{n.}\label{word:KiyaWana}
        \dictdef*{
            adolescent, teenager, older child
        }
    \end{dictentry}
    \begin{dictentry}{KaLa-KasiW}{n.}\label{word:KaLa-KasiW}
        \dictdef*{
            cucumber
        }
    \end{dictentry}
    \begin{dictentry}{uKiWiisa}{n.}\label{word:uKiWiisa}
        \dictdef*{
            self-fulfillment, enrichment, personal growth
        }
    \end{dictentry}
\end{dictroot}

\begin{dictroot}{kl}{tw}\label{root:KL_TW}
    \begin{dictentry}{KLaTW}{n.}\label{word:KLaTW}
        \dictdef*{
            swamp, marsh, bog
        }
    \end{dictentry}
    \begin{dictentry}{iKLuTWa}{n.}\label{word:iKLuTWa}
        \dictdef{
            swampland, marshland, bog country
        }
    \end{dictentry}
\end{dictroot}

\begin{dictroot}{kr}{d}\label{root:KR_D}
    % grandparent-grandchild (distance)
    \begin{dictentry}{KRanaD}{n.}\label{word:KRanaD}
        \dictdef{
            grandparent considered less close due to not being related by blood or otherwise
        }
        \dictdef{
            grandchild considered less close due to not being related by blood or otherwise
        }
    \end{dictentry}
\end{dictroot}

%%%%%%%%%%%
%    L
%%%%%%%%%%%
\section*{L}

\begin{dictroot}{l}{f}
    % to love (romantically)
    \begin{dictentry}{LiiF}{v.tr.}
        \dictdef*{
            to love \textit{sme.} romantically
        }
    \end{dictentry}
\end{dictroot}

\begin{dictroot}{l}{ks}
    %salmon
    \begin{dictentry}{LuKSu}{n.}
        \dictdef*{
            salmon
        }
    \end{dictentry}
\end{dictroot}

\begin{dictroot}{l}{s}
    %to walk
    \begin{dictentry}{LiiS}{v.tr.}
        to walk to \textit{smwh.}
    \end{dictentry}
\end{dictroot}

\begin{dictroot}{l}{t}
    \begin{dictentry}{\famwordold{}{l}{a}{t}{}}{n.}
        \dictdef*{trash, refuse, waste}
    \end{dictentry}
    \begin{dictentry}{\famwordold{i}{l}{aa}{t}{}}{v.intr.}
        \dictdef*{to excrete, poop}
    \end{dictentry}
    \begin{dictentry}{\famwordold{}{l}{ii}{t}{}}{v.tr.}
        \dictdef*{to excrete \textit{smth.}}
    \end{dictentry}
    \begin{dictentry}{\famwordold{aa}{l}{i}{t}{}}{v.tr.}
        \dictdef*{to give \textit{sme.} the shits, cause stomach upset}
    \end{dictentry}
    \begin{dictentry}{\famwordold{}{l}{ur}{t}{i}}{n.}
        \dictdef*{feces, poo, a piece of poop}
    \end{dictentry}
    \begin{dictentry}{\famwordold{}{l}{ar}{t}{i}}{n.}
        \dictdef*{diarrhea, liquid shit}
    \end{dictentry}
    \begin{dictentry}{\famwordold{i}{l}{u}{t}{a}}{n.}
        \dictdef{bathroom, outhouse}
        \dictdef{midden}
    \end{dictentry}
    \begin{dictentry}{\famwordold{}{l}{uli}{t}{}}{n.}
        \dictdef*{colon, lower intestines}
    \end{dictentry}
    \begin{dictentry}{\famwordold{}{l}{asi}{t}{}}{n.}
        \dictdef*{a long, skinny turd}
    \end{dictentry}
    \begin{dictentry}{\famwordold{a}{l}{i}{t}{u}}{n.}
        \dictdef{diaper}
        \dictdef{septic tank}
    \end{dictentry}
    \begin{dictentry}{\famwordold{}{l}{imi}{t}{u}}{n.}
        \dictdef*{toilet, toilet bowl}
    \end{dictentry}
    \begin{dictentry}{\famwordold{u}{l}{i}{t}{i}}{adj.}
        \dictdef{constipated}
        \dictdef{\textit{(fig.)} irritable, easily annoyed}
    \end{dictentry}
\end{dictroot}

\begin{dictroot}{l}{w}
    % high, top, above, rise/raise
    \begin{dictentry}{LaW}{n.}
        \dictdef*{
            the top, topside
        }
    \end{dictentry}
    \begin{dictentry}{LaWjalaW}{n.}
        \dictdef*{
            coat, jacket, sweater, etc.---any `topclothes' that cover the other clothing on the upper body
        }
    \end{dictentry}
\end{dictroot}

\begin{dictroot}{lh}{m}
    \begin{dictentry}{LeHaM}{n.}
        \dictdef*{
            doubt, question, suspicion
        }
    \end{dictentry}
    \begin{dictentry}{LeHiiM}{v.tr}
        \dictdef*{
            to doubt \textit{smth.}, to question \textit{smth.}
        }
    \end{dictentry}
    \begin{dictentry}{LeHiyaM}{v.intr.}
        \dictdef*{
            to be questioned, to be doubted, to be scrutinized
        }
    \end{dictentry}
    \begin{dictentry}{iLHaaM}{v.intr.}
        \dictdef*{
            to be suspicious, to feel suspicion about things, to be full of doubt
        }
    \end{dictentry}
    \begin{dictentry}{LeHiLHiyaM}{v.tr.}
        \dictdef*{
            to question \textit{sme.}, to interrogate \textit{sme.}
        }
    \end{dictentry}
    \begin{dictentry}{aaLHiM}{v.tr.}
        \dictdef*{
            to be suspicious to \textit{sme.}, to arouse \textit{sme.'s} suspicion
        }
    \end{dictentry}
\end{dictroot}

\begin{dictroot}{lj}{d}
    % listen, sound, loud, ear
    \begin{dictentry}{LYiiD}{v.tr.}
        \dictdef*{
            to listen to a sound
        }
    \end{dictentry}
\end{dictroot}

%%%%%%%%%%%
%    M
%%%%%%%%%%%
\section*{M}

\begin{dictroot}{m}{b}
    % slow
    \begin{dictentry}{MaBa}{adj.}
        \dictdef*{
            slow
        }
    \end{dictentry}
\end{dictroot}

\begin{dictroot}{m}{c}
    %monkey -- something else as well maybe?
    \begin{dictentry}{MuCu}{n.}
        \dictdef*{
            monkey
        }
    \end{dictentry}
\end{dictroot}

\begin{dictroot}{m}{h}
    \begin{dictentry}{MaH}{n.}
        \dictdef{
            name, moniker, title
            \begin{quote}
                \famword{wa aFiMu wan MaH BaJauru.}\\
                This book's title is bad.
            \end{quote}
        }
        \dictdef{
            brand, a mark on an item or piece of livestock to show ownership
            \begin{quote}
                \famword{[STAR] ha JENSEN carde wan MaHuru.}\\
                A star is Sir Jensen's brand.
            \end{quote}
        }
        \dictdef{
            \textit{(figurative)} brand, symbolic identity, the name or logo of a product or organization
            \begin{quote}
                \famword{nike MaH wan LuliS-CLurDi nas kaj.}\\
                I want Nike-brand shoes.
            \end{quote}
        }
    \end{dictentry}
    \begin{dictentry}{MiiH}{v.tr.}
        \dictdef{
            to be called \textit{smth.}, to be named \textit{smth.}, to go by \textit{smth.}
            \begin{quote}
                \famword{mi lis MiiHli?}\\
                What is your name?
            \end{quote}
        }
        \dictdef{
            to bear \textit{sme.'s} brand
            \begin{quote}
                \famword{MuHuaj ha JENSEN carde MiiH.}\\
                That bull bears Sir Jensen's brand.
            \end{quote}
        }
    \end{dictentry}
    \begin{dictentry}{MiyaH}{v.intr.}
        \dictdef*{
            to lactate, to produce milk
        }
    \end{dictentry}
    \begin{dictentry}{iMaaH}{v.intr.}
        \dictdef*{
            to call out one's name, to sound off
        }
    \end{dictentry}
    \begin{dictentry}{MiMiyaH}{v.tr}
        \dictdef*{
            to milk \textit{smth.}, to extract milk from \textit{smth.}
        }
    \end{dictentry}
    \begin{dictentry}{aaMiH}{v.tr.}
        \dictdef*{
            to name \textit{sme.}, to call \textit{sme.} by something
        }
        \dictdef{
            to call \textit{sme.'s} name out, to call on \textit{sme.}
        }
    \end{dictentry}
    \begin{dictentry}{MaHa}{adj.}
        \dictdef{
            said, aforementioned, named
            \begin{quote}
                \famword{MaHa FanaS \bigglot ana\bigglot uru.}\\
                The aforementioned person is a fool.
            \end{quote}
        }
        \dictdef{
            dairy, made from cow's milk
            \begin{quote}
                \famword{MaHa TaP nas NiiMe\lilglot{} dake\lilglot{} hwii.}\\
                \textit{`I can't eat any dairy.'}
            \end{quote}
        }
    \end{dictentry}
    \begin{dictentry}{ManaH}{n.}
        \dictdef{
            milkmaid, dairy farmer
        }
        \dictdef{
            wet nurse
        }
    \end{dictentry}
    \begin{dictentry}{MurHi}{n.}
        \dictdef*{
            butter
        }
    \end{dictentry}
    \begin{dictentry}{MarHi}{n.}
        \dictdef{
            cow's milk
        }
        \dictdef{
            any milk or milk-like drink
        }
    \end{dictentry}
    \begin{dictentry}{iMuHa}{n.}
        \dictdef{
            dairy farm
        }
        \dictdef{
            \textit{(fig., slang)} titty bar
        }
    \end{dictentry}
    %\begin{dictentry}{mMiH}{n.}
    %    \dictdef*{}
    %\end{dictentry}
    \begin{dictentry}{inMiH}{n.}
        \dictdef{
            calf, baby cow
        }
        \dictdef{
            \textit{(fig., slang)} small breasts
        }
    \end{dictentry}
    \begin{dictentry}{MuliH}{n.}
        \dictdef*{
            breast, tit, udder, teat
        }
    \end{dictentry}
    \begin{dictentry}{MuHu}{n.}
        \dictdef{
            cow, head of cattle
        }
        \dictdef{
            \textit{(fig., slang)} stripper, busty woman
        }\footnote{Disclaimer: this was sparky's idea}
    \end{dictentry}
    \begin{dictentry}{MasiH}{n.}
        \dictdef*{
            stick of ice cream, ice pop
        }
    \end{dictentry}
    \begin{dictentry}{MajuHa}{n.}
        \dictdef*{
            sign, nameplate
        }
    \end{dictentry}
    \begin{dictentry}{MidiH}{n.}
        \dictdef{
            powdered milk, milk powder
        }
        \dictdef{
            baby formula
        }
    \end{dictentry}
    \begin{dictentry}{MuHi}{adj., n.}
        \dictdef*{
            cream-colored, the color cream, off-white
        }
    \end{dictentry}
    \begin{dictentry}{aMiHu}{n.}
        \dictdef*{
            milk carton
        }
    \end{dictentry}
    \begin{dictentry}{MimiHu}{n.}
        \dictdef*{
            baby bottle
        }
    \end{dictentry}
    \begin{dictentry}{uMiHi}{adj.}
        \dictdef{
            warm and cozy, completely safe and secure (as if being nursed by one's mother as a baby)
        }
        \dictdef{
            milk drunk, in a food coma
        }
    \end{dictentry}
\dictsubtitle{Compounds \& Secondary Derivations}
    \begin{dictentry}{MarHibin}{n}
        \dictdef*{
            cream, milk fat
        }
    \end{dictentry}
\end{dictroot}

\begin{dictroot}{m}{l}
    % dirty, sick
    \begin{dictentry}{MaLa}{adj.}
        \dictdef{
            unhealthy, ill
        }
        \dictdef{
            dirty
        }
    \end{dictentry}
\end{dictroot}

\begin{dictroot}{m}{m}
    %parent-child (close)
    %BUT ALSO to joke, to make fun of
    \begin{dictentry}{MaM}{n.}
        \dictdef*{
            joke
        }
    \end{dictentry}
    \begin{dictentry}{MiiM}{v.tr.}
        \dictdef*{
            to joke about \textit{smth.}
        }
    \end{dictentry}
    \begin{dictentry}{MaMa}{adj.}
        \dictdef{
            funny
        }
        \dictdef{
            parental
        }
    \end{dictentry}
    \begin{dictentry}{ManaM}{n.}
        \dictdef{
            mother, father, parent
        }
        \dictdef{
            daughter, son, child
        }
        \dictdef{
            comedian
        }
    \end{dictentry}
\end{dictroot}

\begin{dictroot}{m}{n}
    % culture
    \begin{dictentry}{MaN}{n.}
        \dictdef*{
            culture
        }
    \end{dictentry}
    \begin{dictentry}{MaNa}{adj.}
        \dictdef*{
            cultured, mannered
        }
    \end{dictentry}
\end{dictroot}

\begin{dictroot}{m}{nt}
    %tall, mountain, long on the natural axis of measurement
    \begin{dictentry}{MiiNT}{v.tr.}
        \dictdef*{
            to place importance on \textit{smth.}, to care about \textit{smth.}, to have \textit{smth.} be important to oneself
        }
    \end{dictentry}
    \begin{dictentry}{MiyaNT}{v.intr.}
        \dictdef*{
            to matter, to be important
        }
    \end{dictentry}
    \begin{dictentry}{iMaaNT}{v.intr.}
        \dictdef*{
            to care, to give a shit
        }
    \end{dictentry}
    \begin{dictentry}{MiMiyaNT}{v.tr.}
        \dictdef*{
            to elevate, to emphasize, to focus on \textit{smth.}, to have \textit{smth.} as one's theme or thesis
        }
    \end{dictentry}
    \begin{dictentry}{aaMiNT}{v.tr.}
        \dictdef*{
            to convince \textit{sme.}, to make \textit{sme.} care, to make \textit{sme.} aware (of something)
        }
    \end{dictentry}
    \begin{dictentry}{MaNTa}{adj.}
        \dictdef{
            large, long, tall, wide, of great proportion in its natural axis of measurement
        }
        \dictdef{
            important, grave
        }
    \end{dictentry}
    \begin{dictentry}{MuNTaw}{n., adv.}
        \dictdef{
            west
        }
        \dictdef{
            westwards, to the west, in the west
        }
    \end{dictentry}
\end{dictroot}

\begin{dictroot}{m}{rk}
    %America -- iMuRKa for continents, MaRKia for USA
    %MuuRK = NatAm?
    \begin{dictentry}{iMuRKa}{n.}
        \dictdef*{
            the Americas
        }
    \end{dictentry}
    \begin{dictentry}{MuuRK}{n.}
        \dictdef*{
            Native American people
        }
    \end{dictentry}
    \begin{dictentry}{MaRKia}{n.}
        \dictdef*{
            the United States of America
        }
    \end{dictentry}
\end{dictroot}

\begin{dictroot}{m}{w}
    % game
    \begin{dictentry}{MaW}{n.}
        \dictdef*{
            game
        }
    \end{dictentry}
\end{dictroot}

\begin{dictroot}{mk}{f}
    \begin{dictentry}{MeKiiF}{v.tr.}
        \dictdef*{
            to knot up \textit{smth.}, to tie a knot in \textit{smth.}
            \begin{quote}
                \famword{MeKasiF ha \textrm{K}NUT MeKiiF}\\
                \textit{`Knut tied a knot in the rope.'}
            \end{quote}
        }
    \end{dictentry}
    \begin{dictentry}{MeKiyaF}{v.intr.}
        \dictdef{to be tied in knots, to be knotted up, to be tangled}
        \dictdef{to be knitted, to be made by forming interconnected knots}
        \dictdef{\textit{(of a situation)} to be complicated, to be overly complex, to be hard to deal with}
    \end{dictentry}
    \begin{dictentry}{iMKaaF}{v.intr.}
        \dictdef{to tie a knot, to tie knots in something, to tangle stuff up}
        \dictdef{to knit something, to knit stuff, to do knitting}
        \dictdef{to obstruct things, to get in the way, to tangle things up}
    \end{dictentry}
    \begin{dictentry}{MeKiMKiyaF}{v.tr.}
        \dictdef{to make \textit{smth.} by knitting, to knit \textit{smth.}}
        \dictdef{to overcomplicate, to cause \textit{smth.} to become complicated and hard to deal with}
    \end{dictentry}
    \begin{dictentry}{MeKurFi}{n.}
        \dictdef*{knot}
    \end{dictentry}
    \begin{dictentry}{MeKasiF}{n.}
        \dictdef*{rope, yarn}
    \end{dictentry}
\end{dictroot}

%%%%%%%%%%%
%    N
%%%%%%%%%%%
\section*{N}

\begin{dictroot}{n}{jd}
    \begin{dictentry}{\famwordold{}{n}{a}{jd}{}}{n.}
        \dictdef*{solitude}
    \end{dictentry}
    \begin{dictentry}{\famwordold{i}{n}{aa}{jd}{}}{v.intr.}
       \dictdef{to get separated, to get left alone}
       \dictdef{to get left behind}
    \end{dictentry}
    \begin{dictentry}{\famwordold{}{n}{ii}{yd}{}}{v.tr.}
        \dictdef{to separate \textit{sme.} from a group}
        \dictdef{to isolate \textit{smth.} from out of a mixture}
        \dictdef{to single out, highlight, emphasize \textit{smth.}}
    \end{dictentry}
    \begin{dictentry}{\famwordold{aa}{n}{i}{yd}{}}{v.tr.}
        \dictdef*{to exclude \textit{smth.} or \textit{sme.}}
    \end{dictentry}
    \begin{dictentry}{\famwordold{}{n}{a}{jd}{a}}{adj.}
        \dictdef{only, solely}
        \dictdef{lone, single}
    \end{dictentry}
    \begin{dictentry}{\famwordold{}{n}{ana}{jd}{}}{n.}
        \dictdef*{outcast}
    \end{dictentry}
    \begin{dictentry}{\famwordold{i}{n}{u}{jd}{a}}{n.}
        \dictdef*{isolated area}
    \end{dictentry}
    \begin{dictentry}{\famwordold{u}{n}{i}{yd}{i}}{n.}
        \dictdef*{lonely}
    \end{dictentry}
\end{dictroot}

\begin{dictroot}{n}{kl}
    % aunt/uncle-niece/nephew
    \begin{dictentry}{NanaKL}{n.}
        \dictdef{
            aunt, uncle, parent's sibling
        }
        \dictdef{
            niece, nephew, sibling's child
        }
    \end{dictentry}
\end{dictroot}

\begin{dictroot}{n}{m}
    \begin{dictentry}{\famwordold{}{n}{a}{m}{}}{n.}
        \dictdef{food, meal, sustenance}
        \dictdef{\textit{(fig.)} fuel}
    \end{dictentry}
    \begin{dictentry}{\famwordold{i}{n}{aa}{m}{}}{v.intr.}
        \dictdef*{to eat \textit{(intr.)}}
    \end{dictentry}
    \begin{dictentry}{\famwordold{}{n}{ii}{m}{}}{v.tr.}
        \dictdef*{to eat \textit{smth.}}
    \end{dictentry}
    \begin{dictentry}{\famwordold{aa}{n}{i}{m}{}}{v.tr.}
        \dictdef{%
            to feed \textit{sme.}
            \begin{quote}
                nas \famwordold{in}{b}{i}{\bigglot}{ini} \famwordold{}{p}{ur}{l}{i} fun \famwordold{aa}{n}{i}{m}{}\\
                \textit{`I fed my littlest sibling an apple.'}
            \end{quote}
        }
        \dictdef{\textit{(lit. or fig.)} to satisfy \textit{sme.}, to sate \textit{sme./smth.} }
    \end{dictentry}
    \begin{dictentry}{\famwordold{}{n}{a}{m}{a}}{adj.}
        \dictdef*{satisfying, filling, tasty}
    \end{dictentry}
    \begin{dictentry}{\famwordold{}{n}{ana}{m}{}}{n.}
        \dictdef{%
            chef, cook
            \begin{quote}
                \famwordold{}{n}{ana}{m}{} daw, \famwordold{}{n}{a}{m}{} \famwordold{}{j}{a}{b}{auru\lilglot} lituc \famwordold{}{f}{ii}{m}{ak}\\
                \textit{`Compliments to the chef.' (\emph{lit.,} `Tell the chef the meal was very good.')}
            \end{quote}%
            }
        \dictdef{%
            feeder, one who feeds (and potentially otherwise cares for) someone or something, -sitter.
            \begin{quote}
                nemi \famwordold{}{c}{u}{s}{u} wan \famwordold{}{n}{ana}{m}{} \famwordold{aa}{t}{i}{lw}{} tuuq\\
                \textit{`We have to hire someone to feed the cat.'}
            \end{quote}
            }
        \dictdef{%
            fulfiller, provider, satisfier
            \begin{quote}
                \famwordold{}{f}{ana}{s}{aj} \famwordold{a}{m}{i}{m}{u} wan \famwordold{}{n}{ana}{m}{uru\lilglot} tuuquc nas wan \famwordold{}{m}{ana}{m}{} \famwordold{}{f}{ii}{m}{}\\
                \textit{`My mother says that a man must be his family's provider.'}
            \end{quote}
        }
    \end{dictentry}
    \begin{dictentry}{\famwordold{}{n}{ur}{m}{i}}{n.}
        \dictdef*{food}
    \end{dictentry}
    \begin{dictentry}{\famwordold{}{n}{ar}{m}{i}}{n.}
        \dictdef*{soup}
    \end{dictentry}
    \begin{dictentry}{\famwordold{i}{n}{u}{m}{a}}{n.}
        \dictdef*{%
            kitchen, dining room
            \begin{quote}
                \famwordold{}{m}{ana}{m}{} \famwordold{}{r}{a}{t}{a} \famwordold{i}{h}{u}{j}{a} fit \famwordold{i}{n}{u}{m}{a} fituru.\\
                \textit{`Mother has been in the kitchen all day.'}
            \end{quote}
        }
    \end{dictentry}
    \begin{dictentry}{\famwordold{m}{n}{i}{m}{}}{n.}
        \dictdef*{eating or cooking utensil}
    \end{dictentry}
    \begin{dictentry}{\famwordold{in}{n}{i}{m}{}}{n.}
        \dictdef*{snack, morsel}
    \end{dictentry}
    \begin{dictentry}{\famwordold{}{n}{uli}{m}{}}{n.}
        \dictdef{
            mouth
            \begin{quote}
                \famwordold{}{k}{ur}{k}{i} \famwordold{}{n}{uli}{m}{} daw fit nas \famwordold{aa}{pl}{i}{s}{}\\
                \textit{`I put the cookie in my mouth.'}
            \end{quote}
            }
        \dictdef{
            teeth
            \begin{quote}
                \famwordold{}{n}{uli}{m}{} mi \famwordold{i}{s}{aa}{j}{uc} fidul \famwordold{}{s}{ii}{n}{ak}!\\
                \textit{`Brush your teeth before bed!'}
            \end{quote}
            }
    \end{dictentry}
    \begin{dictentry}{\famwordold{}{n}{u}{m}{u}}{n.}
        \dictdef*{locust}
    \end{dictentry}
    \begin{dictentry}{\famwordold{}{n}{asi}{m}{}}{n.}
        \dictdef*{chopsticks}
    \end{dictentry}
    \begin{dictentry}{\famwordold{}{n}{aju}{m}{a}}{n.}
        \dictdef*{plate, platter, surface for eating}
    \end{dictentry}
    \begin{dictentry}{\famwordold{}{n}{idi}{m}{}}{n.}
        \dictdef*{ground spice or seasoning}
    \end{dictentry}
    \begin{dictentry}{\famwordold{a}{n}{i}{m}{u}}{n.}
        \dictdef*{jar}
    \end{dictentry}
    \begin{dictentry}{\famwordold{}{n}{imi}{m}{u}}{n.}
        \dictdef*{bowl}
    \end{dictentry}
    \begin{dictentry}{\famwordold{u}{n}{i}{m}{i}}{adj.}
        \dictdef*{hungry}
    \end{dictentry}
\dictsubtitle{Compounds \& Secondary Derivations}
    \begin{dictentry}{\famwordold{in}{f}{i}{s}{}\famwordold{}{n}{ana}{m}{}}{n.}
        \dictdef*{babysitter, nanny}
    \end{dictentry}
    \begin{dictentry}{\famwordold{m}{n}{i}{m}{aj}}{n.}
        \dictdef*{spoon}
    \end{dictentry}
    \begin{dictentry}{\famwordold{m}{n}{i}{m}{un}}{n.}
        \dictdef*{fork}
    \end{dictentry}
    \begin{dictentry}{\famwordold{}{n}{a}{m}{anara}}{int.}
        \dictdef*{bon appetit, have a nice meal}
    \end{dictentry}
    \begin{dictentry}{\famwordold{}{n}{ar}{m}{ibin}}{n.}
        \dictdef*{stew}
    \end{dictentry}
    \begin{dictentry}{\famwordold{}{f}{a}{s}{a}-\famwordold{}{n}{a}{m}{}}{n.}
        \dictdef*{breakfast, morning meal}
    \end{dictentry}
    \begin{dictentry}{\famwordold{}{h}{a}{j}{a}-\famwordold{}{n}{a}{m}{}}{n.}
        \dictdef*{lunch, midday meal}
    \end{dictentry}
    \begin{dictentry}{\famwordold{}{h}{a}{t}{a}-\famwordold{}{n}{a}{m}{}}{n.}
        \dictdef*{midnight snack}
    \end{dictentry}
    \begin{dictentry}{\famwordold{}{s}{a}{j}{a}-\famwordold{}{n}{a}{m}{}}{n.}
        \dictdef*{dinner, supper, evening meal}
    \end{dictentry}
\end{dictroot}

\begin{dictroot}{n}{rk}
    %cold, Norway
    \begin{dictentry}{NaRKa}{adj.}
        \dictdef{
            cold
        }
        \dictdef{
            Norwegian
        }
    \end{dictentry}
    \begin{dictentry}{NuuRK}{n.}
        \dictdef{
            Norway
        }
        \dictdef{
            the Norwegian people
        }
    \end{dictentry}
\end{dictroot}

\begin{dictroot}{n}{s}
    \begin{dictentry}{NaS}{n.}
        \dictdef{
            distance, far-ness
        }
        \dictdef{
            10km, a standardized Scandinavian mile
        }
    \end{dictentry}
    \begin{dictentry}{NaSa}{adj.}
        \dictdef*{
            far
        }
    \end{dictentry}
    \begin{dictentry}{inNiS}{n.}
        \dictdef*{
            a kilometer
        }
    \end{dictentry}
\dictsubtitle{Compounds \& Secondary Derivations}
    \begin{dictentry}{KLaTW-NaS}{n.}
        \dictdef*{
            a `bog-mile', roughly 1.25km
        }
    \end{dictentry}
\end{dictroot}

\begin{dictroot}{n}{w}
    %die, death
    \begin{dictentry}{NaW}{n.}
        \dictdef*{
            death
        }
    \end{dictentry}
    \begin{dictentry}{NiiW}{v.tr.}
        \dictdef*{
            to murder \textit{sme.}, to kill \textit{sme.} (intentionally)
        }
    \end{dictentry}
    \begin{dictentry}{NiyaW}{v.intr.}
        \dictdef*{
            to die
        }
    \end{dictentry}
    \begin{dictentry}{NiNiyaW}{v.tr.}
        \dictdef*{
            to cause \textit{sme.} to die, to kill \textit{sme.} (with or without intention), to be the means of one's demise
        }
    \end{dictentry}
\end{dictroot}

%%%%%%%%%%%
%    P
%%%%%%%%%%%
\section*{P}

\begin{dictroot}{p}{d}
    \begin{dictentry}{PaD}{n.}
        \dictdef*{
            liter, 1000 milliliters, pot \textit{(derived from Danish `pot')}
        }
    \end{dictentry}
    \begin{dictentry}{inPiD}{n.}
        \dictdef*{
            pint, ¼ liter, 250 milliliters, ¼ pot
        }
    \end{dictentry}
\dictsubtitle{Compounds \& Secondary Derivations}
    \begin{dictentry}{inPiDini}{n.}
        \dictdef{
            milliliter
        }
    \end{dictentry}
\end{dictroot}

\begin{dictroot}{p}{l}
    \begin{dictentry}{PaL}{n.}
        \dictdef{
            fertility, fruitfulness
        }
        \dictdef{
            plenty, abundance of some desired thing or quality
        }
    \end{dictentry}
    \begin{dictentry}{PiiL}{v.tr.}
        \dictdef*{
            to result in \textit{smth.}, to produce \textit{smth.}, to bear
        }
    \end{dictentry}
    \begin{dictentry}{PiyaL}{v.intr.}
        \dictdef*{
            to bear fruit, \textit{(of a method or tool)} to produce results
        }
    \end{dictentry}
    \begin{dictentry}{iPaaL}{v.intr.}
        \dictdef*{
            \textit{(of a person)} to get results, to be effective at producing positive outcomes
        }
    \end{dictentry}
    \begin{dictentry}{PiPiyaL}{v.tr.}
        \dictdef*{
            to cause \textit{smth.} to bear fruit, to fertilize, to cause \textit{a method or tool} to produce results
        }
    \end{dictentry}
    \begin{dictentry}{aaPiL}{v.tr.}
        \dictdef*{
            to get results from \textit{sme.}, to motivate \textit{sme.}
        }
    \end{dictentry}
    \begin{dictentry}{PaLa}{adj.}
        \dictdef{
            fruitful, fertile
        }
        \dictdef{
            plentiful, abundant
        }
    \end{dictentry}
    \begin{dictentry}{PanaL}{n.}
        \dictdef*{
            a hard worker, an overachiever, someone who gets results
        }
    \end{dictentry}
    \begin{dictentry}{PurLi}{n.}
        \dictdef{
            fruit, vegetable, nut, the edible product of a plant
        }
        \dictdef{
            apple, the fruit of \textit{Malus domestica} specifically
        }
        \dictdef{
            \emph{(fig.)} end result, effect, consequence
        }
    \end{dictentry}
    \begin{dictentry}{ParLi}{n.}
        \dictdef*{
            juice, fruit juice
        }
    \end{dictentry}
    \begin{dictentry}{iPuLa}{n.}
        \dictdef{
            orchard, plantation, fruit farm
        }
    \end{dictentry}
    \begin{dictentry}{mPiL}{n.}
        \dictdef{
            fertilizer, substance intended to improve the amount of fruit born by a plant
        }
        \dictdef{
            any device or substance intended to promote human fertility
        }
        \dictdef{
            any device or substance intended to increase the output of a method or tool
        }
    \end{dictentry}
    \begin{dictentry}{inPiL}{n.}
        \dictdef*{
            berry, nut, small fruit
        }
    \end{dictentry}
    \begin{dictentry}{PuliL}{n.}
        \dictdef{
            womb, female reproductive system
        }
        \dictdef{
            \textit{(botany)} the fruit-producing part of a flower, pistil, ovaries \textit{(of a plant)}
        }
    \end{dictentry}
    %\begin{dictentry}{\famwordold{}{p}{u}{l}{u}}{n.}
    %    \dictdef*{squirrel}
    %\end{dictentry}
    \begin{dictentry}{PasiL}{n.}
        \dictdef*{
            long, slender fruit or vegetable (e.g., cucumber, eggplant, zucchini, yellow squash); cucumiform
        }
    \end{dictentry}
    \begin{dictentry}{PidiL}{n.}
        \dictdef*{
            solute, the component that is dissolved into some solvent to form a solution
        }
    \end{dictentry}
    \begin{dictentry}{PuLi}{adj., n.}
        \dictdef*{
            red, reddish-pink, the color red/reddish-pink
        }
    \end{dictentry}
    \begin{dictentry}{aPiLu}{n.}
        \dictdef*{
            jar, can
        }
    \end{dictentry}
    \begin{dictentry}{PimiLu}{n.}
        \dictdef*{
            basket
        }
    \end{dictentry}
\dictsubtitle{Compounds \& Secondary Derivations}
    \begin{dictentry}{SaH-PurLi}{n.}
        \dictdef*{
            onion, onion bulb
        }
    \end{dictentry}
\end{dictroot}

\begin{dictroot}{p}{n}
    % bread
    \begin{dictentry}{PurNi}{n.}
        \dictdef{
            loaf of bread
        }
        \dictdef{
            bread roll?
        }
    \end{dictentry}
    \begin{dictentry}{PasiN}{n.}
        \dictdef*{
            baguette
        }
    \end{dictentry}
\end{dictroot}

\begin{dictroot}{p}{nd}
    %pound
    \begin{dictentry}{PaND}{n.}
        \dictdef*{
            half kilo, 500 grams, pound \textit{derived from Danish `pund'}
        }
    \end{dictentry}
    \begin{dictentry}{inPiND}{n.}
        \dictdef*{
            gram
        }
    \end{dictentry}
\dictsubtitle{Compounds \& Secondary Derivations}
    \begin{dictentry}{inPiNDini}{n.}
        \dictdef*{
            milligram
        }
    \end{dictentry}
    \begin{dictentry}{KILO-PaND}{n.}
        \dictdef*{
            kilogram, 1000 grams, two \famword{PaND}\\
            \textit{NB: often shortened to just \famword{KILO}}
        }
    \end{dictentry}
\end{dictroot}

\begin{dictroot}{p}{s}
    % piss
    \begin{dictentry}{iPaaS}{v.intr.}
        \dictdef{
            to piss, to urinate, to pee
        }
    \end{dictentry}
    \begin{dictentry}{ParSi}{n.}
        \dictdef*{
            urine
        }
    \end{dictentry}
\end{dictroot}

\begin{dictroot}{pl}{d}
    % swim
    \begin{dictentry}{iPLaaD}{v.intr.}
        \dictdef*{
            to swim with one's body completely submerged in water, as done by a fish, scuba diver, or submarine
        }
    \end{dictentry}
    \begin{dictentry}{PLuDu}{n.}
        \dictdef{
            fish, aquatic non-mammalian vertebrate
        }
        \dictdef{
            any sea creature that swims completely submerged in water
        }
    \end{dictentry}
\end{dictroot}

\begin{dictroot}{pl}{s}
    \begin{dictentry}{PLaS}{n.}
        \dictdef*{self-propelled motion, movement}
    \end{dictentry}
    \begin{dictentry}{PLiiS}{v.tr.}
        \dictdef*{to move (oneself) to \textit{smwh.}, to go to \textit{smwh.}}
    \end{dictentry}
    \begin{dictentry}{PLiyaS}{v.intr.}
        \dictdef*{
            to be located \textit{(smwh.)}, to be present \textit{(smwh.)}
        }
    \end{dictentry}
    \begin{dictentry}{iPLaaS}{v.intr.}
        \dictdef*{to dwell, to settle \textit{(in some location)}, to establish one's home}
    \end{dictentry}
    \begin{dictentry}{PLiPLiyaS}{v.tr.}
        \dictdef{
            to place \textit{smth.} \textit{(smwh.)}
        }
        \dictdef{
            to establish \textit{smth.} \textit{(smwh.)}, to build \textit{smth.} \textit{(smwh.)}, to set \textit{smth.} up \textit{(smwh.)}
        }
    \end{dictentry}
    \begin{dictentry}{aaPLiS}{v.tr.}
        \dictdef{
            to cause \textit{sme.} to move (themselves) \textit{(smwh.)}
        }
        \dictdef{
            to cause \textit{sme.} to settle down \textit{(smwh.)}, to set \textit{sme.} up (with a place to live or job)
        }
    \end{dictentry}
    \begin{dictentry}{PLanaS}{n.}
        \dictdef{driver, chauffeur}
        \dictdef{mover, hauler}
    \end{dictentry}
    \begin{dictentry}{PLarSi}{n.}
        \dictdef{gasoline}
        \dictdef{\textit{(slang)} coffee}
    \end{dictentry}
    \begin{dictentry}{iPLuSa}{n.}
        \dictdef{place, location}
        \dictdef{stop, station, terminal}
    \end{dictentry}
    \begin{dictentry}{mPLiS}{n.}
        \dictdef*{mode of transportation}
    \end{dictentry}
    \begin{dictentry}{PLuSu}{n.}
        \dictdef{car, motor vehicle, automobile}
        \dictdef{\textit{(dated)} mount, ridden animal}
    \end{dictentry}
    \begin{dictentry}{PLasiS}{n.}
        \dictdef*{line, row}
    \end{dictentry}
    \begin{dictentry}{PLajuSa}{n.}
        \dictdef*{layer}
    \end{dictentry}
    \begin{dictentry}{PLidiS}{n.}
        \dictdef*{\textit{(slang)} cocaine}
    \end{dictentry}
    \begin{dictentry}{aPLiSu}{n.}
        \dictdef*{engine}
    \end{dictentry}
    \begin{dictentry}{PLimiSu}{n.}
        \dictdef*{generic container, bin, box}
    \end{dictentry}
\end{dictroot}

\begin{dictroot}{pr}{l}
    \begin{dictentry}{PRiiL}{v.tr.}
        \dictdef*{to forget \textit{smth.}, to have \textit{smth.} slip one's mind}
    \end{dictentry}
\end{dictroot}

%%%%%%%%%%%
%    Q
%%%%%%%%%%%
\section*{Q}

\begin{dictroot}{q}{\bigglot}
    % to believe, opine, conclude
    \begin{dictentry}{iQaa\bigglot}{n.}
        \dictdef*{
            to think, believe, be of the opinion of
        }
    \end{dictentry}
\end{dictroot}

\begin{dictroot}{q}{h}
    \begin{dictentry}{QiiH}{v.tr.}
        \dictdef*{
            to weave in and out of \textit{smth.}, to move through by turning and twisting, to make ones way through \textit{smth.} by winding in and out or from side to side
            }
    \end{dictentry}
    \begin{dictentry}{QiyeH}{v.intr.}
        \dictdef{
            to be woven, to be made to weave in and out of something, to be twisted and turned around something
            \begin{quote}
                \famword{TBD}//
                \textit{A strand of grass was woven in her hair.}
            \end{quote}
        }
        \dictdef{
            to be formed by weaving things together, to be woven
            \begin{quote}
                \famword{TBD}//
                \textit{This cloth was woven 100 years ago.}
            \end{quote}
        }
    \end{dictentry}
    \begin{dictentry}{iQaaH}{v.intr.}
        \dictdef*{
            to dodge and weave, to move back and forth in a serpentine manner
        }
    \end{dictentry}
    \begin{dictentry}{QiQiyaH}{v.tr.}
        \dictdef{
            to weave \textit{smth.} together, to cause \textit{smth.} to be woven together
            \begin{quote}
                \famword{TBD}\\
                \textit{He wove the threads together to make a blanket.}
            \end{quote}
        }
        \dictdef{
            to weave \textit{smth.}, to create \textit{smth.} by weaving
            \begin{quote}
                \famword{TBD}\\
                \textit{My grandmother wove me this sweater.}
            \end{quote}
        }
    \end{dictentry}
    \begin{dictentry}{QajuHa}{n.}
        \dictdef*{
            fabric, cloth, woven fabric, cloth
        }
    \end{dictentry}
    \begin{dictentry}{QuHu}{n.}
        \dictdef*{lizard, reptile}
    \end{dictentry}
    \begin{dictentry}{QasiH}{n.}
        \dictdef*{snake}
    \end{dictentry}
\end{dictroot}

\begin{dictroot}{q}{r}
    %squirrel
    \begin{dictentry}{QuRu}{n.}
        \dictdef*{
            squirrel
        }
    \end{dictentry}
\end{dictroot}

\begin{dictroot}{q}{st}
    % barley
    \begin{dictentry}{QurSTi}{n.}
        \dictdef*{
            barley grain
        }
    \end{dictentry}
    \begin{dictentry}{iQuSTa}{n.}
        \dictdef*{
            barley field
        }
    \end{dictentry}
    \begin{dictentry}{QasiST}{n.}
        \dictdef*{
            stalk of barley
        }
    \end{dictentry}
    \begin{dictentry}{QidiST}{n.}
        \dictdef*{
            barley flour
        }
    \end{dictentry}
\end{dictroot}

%%%%%%%%%%%
%    R
%%%%%%%%%%%
\section*{R}

\begin{dictroot}{r}{\bigglot}
    % rule, law
    \begin{dictentry}{Ra\bigglot}{n.}
        \dictdef{
            rule, commandment, law, ordinance
        }
        \dictdef{
            law, justice, set of rules by which society runs
        }
    \end{dictentry}
    \begin{dictentry}{Rana\bigglot}{n.}
        \dictdef*{
            legislator, judge, lawmaker, legislative representative
        }
    \end{dictentry}
    \begin{dictentry}{iRu{\bigglot}a}{n.}
        \dictdef*{
            legislature
        }
    \end{dictentry}
\end{dictroot}

\begin{dictroot}{r}{kd}
    % wood
    \begin{dictentry}{RurKDi}{n.}
        \dictdef*{
            piece of wood
        }
    \end{dictentry}
\end{dictroot}

\begin{dictroot}{r}{m}
    \begin{dictentry}{RiiM}{v.tr.}
        \dictdef*{
            to hold \textit{smth.}, carry \textit{smth.}
        }
    \end{dictentry}
    \begin{dictentry}{RiyaM}{v.intr.}
        \dictdef*{
            to be held, to be grasped, to be kept
            }
    \end{dictentry}
    \begin{dictentry}{RiRiyaM}{v.tr.}
        \dictdef*{
            to hand \textit{smth.} (to \textit{sme.}), to hand \textit{smth.} over, to pass \textit{smth.} (to \textit{sme.})
        }
    \end{dictentry}
    \begin{dictentry}{aaRiM}{v.tr.}
        \dictdef*{
            to grab \textit{smth.}, to grasp \textit{smth.}, to take hold of \textit{smth.}
        }
    \end{dictentry}
\end{dictroot}

\begin{dictroot}{r}{q}
    \begin{dictentry}{RaQ}{n.}
        \dictdef{pain}
        \dictdef{strike, blow}
    \end{dictentry}
    \begin{dictentry}{RiiQ}{v.tr.}
        \dictdef{
            to attack \textit{sme.}, to strike \textit{sme.}, to hit \textit{sme.}, to try to injure \textit{sme.} with volition, to try to hurt \textit{sme.} deliberately
            \begin{quote}
                \famword{TBD}\\
                \textit{`Magnus's dog attacked me!'}
            \end{quote}
            }
    \end{dictentry}
    \begin{dictentry}{RiyaQ}{v.intr.}
        \dictdef*{
            \textit{(of a person)} to suffer, to be in pain, to be injured
            \begin{quote}
                \famword{TBD}\\
                \textit{`The sick old woman is in a lot of pain.'}
            \end{quote}
            }
    \end{dictentry}
    \begin{dictentry}{iRaaQ}{v.intr}
        \dictdef*{
            to cause harm through presence or inaction, to be harmful, to cause injury without volition
            \begin{quote}
                \famword{TBD}\\
                \textit{`A blow to the head really hurts.'}
            \end{quote}
        }
    \end{dictentry}
    \begin{dictentry}{RiRiyaQ}{v.tr.}
        \dictdef*{
            \textit{(of a body part, reflexive)} to hurt \textit{(sme.)}, to be injured, to cause pain
            \begin{quote}
                \famword{BuliT nas RiRiyaQami}\\
                \textit{`My head hurts.'}
            \end{quote}
        }
    \end{dictentry}
    \begin{dictentry}{aaRiQ}{v.tr.}
        \dictdef*{
            to hurt \textit{sme.}, to inflict pain on \textit{sme.} but without volition, to cause pain or injury to \textit{sme.} through inaction or mere presence
            \begin{quote}
                FanaS MarHi aaRiQibitu nas FanaSuru.\\
                \textit{`I am lactose-intolerant.'}\\
                (lit., \textit{`I am a person who milk hurts.'})
            \end{quote}
        }
    \end{dictentry}
    \begin{dictentry}{RaQa}{adj.}
        \dictdef*{painful, hurtful, injurous}
    \end{dictentry}
    \begin{dictentry}{RanaQ}{n.}
        \dictdef*{attacker, fighter, injurer}
    \end{dictentry}
    \begin{dictentry}{mRiQ}{n.}
        \dictdef*{weapon}
    \end{dictentry}
    \begin{dictentry}{RuQi}{adj., n.}
        \dictdef*{red, the color red}
    \end{dictentry}
    \begin{dictentry}{uRiQi}{adj.}
        \dictdef*{hurt, hurting, in pain, injured}
    \end{dictentry}
\dictsubtitle{Compounds \& Secondary Derivations}
    \begin{dictentry}{uRiQiana}{n.}
        \dictdef*{victim, injured person}
    \end{dictentry}
\end{dictroot}

\begin{dictroot}{r}{r}
    % strange, weird, bizarre
    \begin{dictentry}{RaRa}{adj.}
        \dictdef*{
            weird, strange, bizarre
        }
    \end{dictentry}
\end{dictroot}

\begin{dictroot}{r}{t}
    % cycle, year, wheel, whole
    \begin{dictentry}{RaT}{n.}
        \dictdef{
            cycle, circle, lap, revolution
        }
        \dictdef{
            year
        }
        \dictdef{
            \textit{(of a game)} round, game
        }
        \dictdef{
            conclusion
        }
    \end{dictentry}
    \begin{dictentry}{RiiT}{v.tr.}
        \dictdef*{
            to complete, finish \textit{smth.}
        }
    \end{dictentry}
    \begin{dictentry}{RiyaT}{v.intr.}
        \dictdef{
            to move in a circle, orbit around a central point
        }
        \dictdef{
            \textit{(of an event)} to conclude, finish, cycle
        }
    \end{dictentry}
    \begin{dictentry}{iRaaT}{v.intr.}
        \dictdef*{
            to rotate around one's own axis
        }
    \end{dictentry}
    \begin{dictentry}{RiRiyaT}{v.tr.}
        \dictdef{
            to have \textit{smth.} orbit one's own location, to be the central point around which something orbits
        }
        \dictdef{
            \textit{(fig.)} to be the center of attention, seek to have influence over everything
        }
    \end{dictentry}
    \begin{dictentry}{aaRiT}{v.tr.}
        \dictdef*{
            to spin \textit{smth.}
        }
    \end{dictentry}
    \begin{dictentry}{RaTa}{adj.}
        \dictdef*{
            whole, entire, complete
        }
    \end{dictentry}
    \begin{dictentry}{RanaT}{n.}
        \dictdef*{
            ???
        }
    \end{dictentry}
    \begin{dictentry}{RurTi}{n.}
        \dictdef*{
            wheel
        }
    \end{dictentry}
    \begin{dictentry}{iRuTa}{n.}
        \dictdef*{
            roundabout, traffic circle
        }
    \end{dictentry}
    \begin{dictentry}{mRiT}{n.}
        \dictdef*{
            turntable
        }
    \end{dictentry}
    \begin{dictentry}{inRiT}{n.}
        \dictdef{
            section, part of cycle or lap
        }
        \dictdef{
            \textit{(of a game)} a person's turn
        }
    \end{dictentry}
    \begin{dictentry}{RuliT}{n.}
        \dictdef*{
            uterine lining
        }
    \end{dictentry}
    \begin{dictentry}{RuTu}{n.}
        \dictdef*{
            roly poly, potato bug, woodlouse
        }
    \end{dictentry}
    \begin{dictentry}{RasiT}{n.}
        \dictdef*{
            axis, axle, shaft
        }
    \end{dictentry}
    \begin{dictentry}{RajuTa}{n.}
        \dictdef*{
            calendar, timetable
        }
    \end{dictentry}
\end{dictroot}

\begin{dictroot}{rt}{s}
    \begin{dictentry}{ReTaSa}{adj.}
        \dictdef{
            nude, bare, naked
        }
        \dictdef{
            just, merely, only
        }
    \end{dictentry}
\end{dictroot}

%%%%%%%%%%%
%    S
%%%%%%%%%%%
\section*{S}

\begin{dictroot}{s}{fr}
    % hot
    \begin{dictentry}{SaFR}{n.}
        \dictdef*{
            heat
        }
    \end{dictentry}
    \begin{dictentry}{SiiFR}{v.tr.}
        \dictdef*{
            to cook, bake, boil \textit{smth.}
        }
    \end{dictentry}
    \begin{dictentry}{SiyaFR}{v.intr.}
        \dictdef*{
            \textit{(mediopassive)} to cook, bake, boil
        }
    \end{dictentry}
    \begin{dictentry}{iSaaFR}{v.intr.}
        \dictdef*{
            to make food, prepare food
        }
    \end{dictentry}
    \begin{dictentry}{SiSiyaFR}{v.tr.}
        \dictdef*{
            \textit{(of a cooking apparatus)} to cook, bake, boil \textit{smth.}
        }
    \end{dictentry}
    \begin{dictentry}{aaSiFR}{v.tr.}
        \dictdef*{
            \textit{(of foodstuff)} to entice \textit{sme.} into eating or preparing food
        }
    \end{dictentry}
    \begin{dictentry}{SaFRa}{adj.}
        \dictdef{
            hot, warm
        }
        \dictdef{
            spicy
        }
    \end{dictentry}
    \begin{dictentry}{SanaFR}{n.}
        \dictdef*{
            cook, chef
        }
    \end{dictentry}
    \begin{dictentry}{SurFRi}{n.}
        \dictdef*{
            coal, embers
        }
    \end{dictentry}
    \begin{dictentry}{SarFRi}{n.}
        \dictdef{
            hot beverage, coffee, tea
        }
        \dictdef{
            \textit{(colloquial)} lava
        }
    \end{dictentry}
    \begin{dictentry}{iSuFRa}{n.}
        \dictdef*{
            desert
        }
    \end{dictentry}
    \begin{dictentry}{mSiFR}{n.}
        \dictdef*{
            radiator, space heater
        }
    \end{dictentry}
    \begin{dictentry}{inSiFR}{n.}
        \dictdef*{

        }
    \end{dictentry}
    \begin{dictentry}{SuliFR}{n.}
        \dictdef{
            body temperature
        }
        \dictdef{
            warm-bloodedness
        }
    \end{dictentry}
    \begin{dictentry}{SasiFR}{n.}
        \dictdef*{
            matchstick
        }
    \end{dictentry}
    \begin{dictentry}{SidiFR}{n.}
        \dictdef{
            sand
        }
    \end{dictentry}
    \begin{dictentry}{aSiFRu}{n.}
        \dictdef*{
            sauna
        }
    \end{dictentry}
    \begin{dictentry}{SuFRi}{n., adj.}
        \dictdef*{
            dark orange, red-orange
        }
    \end{dictentry}
    \begin{dictentry}{uSiFRi}{adj.}
        \dictdef*{
            feverish, having an elevated body temperature
        }
    \end{dictentry}
\end{dictroot}

\begin{dictroot}{s}{h}
    \begin{dictentry}{SaH}{n.}
        \dictdef*{a foul smell, a pungent odor}
    \end{dictentry}
    \begin{dictentry}{SiyaH}{v.tr.}
        \dictdef*{to stink, to smell bad, to be malodorous}
    \end{dictentry}
\end{dictroot}

\begin{dictroot}{s}{j}
    \begin{dictentry}{SaJ}{n.}
        \dictdef*{
            sleep, slumber, the state of being asleep
        }
    \end{dictentry}
    \begin{dictentry}{SiiY}{v.tr.}
        \dictdef*{
            to say good-bye to \textit{sme.}, to bid adieu
        }
    \end{dictentry}
    \begin{dictentry}{SiyaJ}{v.intr.}
        \dictdef*{
            to sleep, to be asleep, to fall asleep
        }
    \end{dictentry}
    \begin{dictentry}{iSaaJ}{v.intr.}
        \dictdef*{
            to go to bed, to relax with the intention of falling asleep    
        }
    \end{dictentry}
    \begin{dictentry}{SiSiyaJ}{v.tr.}
        \dictdef{
            to put to sleep, to knock out
        }
        \dictdef{
            \textit{(fig.)} to bore \textit{sme.}
        }
    \end{dictentry}
    \begin{dictentry}{aaSiY}{v.tr.}
        \dictdef{
            to put \textit{sme.} to bed, to tuck \textit{sme.} in
        }
        \dictdef{
            to coddle \textit{sme.}, patronize \textit{sme.}
        }
    \end{dictentry}
    \begin{dictentry}{SaJa}{adj.}
        \dictdef{
            tiring, sleep-inducing
        }
        \dictdef{
            evening's, in the evening \textit{(from \famword{SaJa-TaN})}
        }
        \dictdef{
            \textit{(fig.)} boring
        }
    \end{dictentry}
    \begin{dictentry}{SurYi}{n.}
        \dictdef*{
            pillow
        }
    \end{dictentry}
    \begin{dictentry}{SarYi}{n.}
        \dictdef*{
            a warm drink drunk before bed, nightcap
        }
    \end{dictentry}
    \begin{dictentry}{iSuJa}{n.}
        \dictdef{
            bed
        }
        \dictdef{
            bedroom
        }
    \end{dictentry}
    \begin{dictentry}{mSiY}{n.}
        \dictdef*{
            sedative, sleeping pill, sleep aid
        }
    \end{dictentry}
    \begin{dictentry}{inSiY}{n.}
        \dictdef*{
            nap, catnap
        }
    \end{dictentry}
    \begin{dictentry}{SuliY}{n.}
        \dictdef*{
            back, dorsum, the human back
        }
    \end{dictentry}
    \begin{dictentry}{SuJu}{n.}
        \dictdef*{
            bat
        }
    \end{dictentry}
    \begin{dictentry}{SajuJa}{n.}
        \dictdef{
            quilt, comforter, blanket, duvet
        }
        \dictdef{
            sleeping bag
        }
    \end{dictentry}
    \begin{dictentry}{SidiY}{n.}
        \dictdef*{
            rheum, sleep sand, eye booger, the crust or mucus that accumulates at the corner of one's eye during sleep
        }
    \end{dictentry}
    \begin{dictentry}{SimiYu}{n.}
        \dictdef*{
            crib, cradle, bassonet
        }
    \end{dictentry}
    \begin{dictentry}{SuuJ}{n.}
        \dictdef*{
            Dreamland, the world where people's souls go when they sleep
        }
    \end{dictentry}
    \begin{dictentry}{SuYi}{adj., n.}
        \dictdef*{
            the colors you see when you close your eyes
        }
    \end{dictentry}
    \begin{dictentry}{uSiYi}{adj.}
        \dictdef*{
            sleepy, tired, exhausted
        }
    \end{dictentry}
    \begin{dictentry}{SaJejalaj}{n.}
        \dictdef*{
            pajamas, sleep clothes
        }
    \end{dictentry}
\dictsubtitle{Compounds \& Secondary Derivations}
    \begin{dictentry}{SaJanara}{int.}
        \dictdef*{
            good night, sleep well, sweet dreams, goodbye (when the speaker does not expect to see the addressee again until at least after one of them has had a night's sleep)
        }
    \end{dictentry}
\end{dictroot}

\begin{dictroot}{s}{n}
    % clean, healthy
    \begin{dictentry}{SaN}{n.}
        \dictdef{
            health
        }
        \dictdef{
            hygiene
        }
    \end{dictentry}
    \begin{dictentry}{SiiN}{v.tr.}
        \dictdef*{
            to clean \textit{smth.} or \textit{sme.}
        }
    \end{dictentry}
    \begin{dictentry}{SiyaN}{v.intr.}
        \dictdef{
            to be rinsed off of something, to be cleaned off of something
        }
        \dictdef{
            \textit{(of an illness)} to be cured, alleviated
        }
    \end{dictentry}
    \begin{dictentry}{iSaaN}{v.intr.}
        \dictdef{
            to clean up, to wash, to tidy up
        }
        \dictdef{
            to clean oneself, to bathe, to wash up
        }
    \end{dictentry}
    \begin{dictentry}{SiSiyaN}{v.tr.}
        \dictdef{
            to wash \textit{smth.} off of an object
        }
        \dictdef{
            to cure \textit{sme.} of an illness
        }
    \end{dictentry}
    \begin{dictentry}{aaSiN}{v.tr.}
        \dictdef*{
            to clean up after \textit{sme.}, to take care of someone in the capacity of a home nurse or caretaker
        }
    \end{dictentry}
    \begin{dictentry}{SaNa}{adj.}
        \dictdef{
            clean
        }
        \dictdef{
            healthy
        }
    \end{dictentry}
    \begin{dictentry}{SanaN}{n.}
        \dictdef*{
            maid, caretaker, home nurse
        }
    \end{dictentry}
    \begin{dictentry}{SurNi}{n.}
        \dictdef*{
            sponge
        }
    \end{dictentry}
    \begin{dictentry}{SarNi}{n.}
        \dictdef{
            soap, detergent
        }
        \dictdef{
            liquid medicine
        }
    \end{dictentry}
    \begin{dictentry}{iSuNa}{n.}
        \dictdef*{
            bathroom
        }
    \end{dictentry}
    \begin{dictentry}{mSiN}{n.}
        \dictdef{
            syringe
        }
        \dictdef{
            vaccine
        }
    \end{dictentry}
    \begin{dictentry}{inSiN}{n.}
        \dictdef*{
            pick-me-up, remedy
        }
    \end{dictentry}
    \begin{dictentry}{SuliN}{n.}
        \dictdef*{
            spleen
        }
    \end{dictentry}
    \begin{dictentry}{SuNu}{n.}
        \dictdef*{
            ladybug, believed to grant good health to anyone in lands on
        }
    \end{dictentry}
    \begin{dictentry}{SasiN}{n.}
        \dictdef*{
            some kinda healthy plant idk
        }
    \end{dictentry}
    \begin{dictentry}{SajuNa}{n.}
        \dictdef*{
            washcloth
        }
    \end{dictentry}
    \begin{dictentry}{SidiN}{n.}
        \dictdef*{
            medicine
        }
    \end{dictentry}
    \begin{dictentry}{aSiNu}{n.}
        \dictdef{
            dishwasher
        }
        \dictdef{
            washing machine
        }
    \end{dictentry}
    \begin{dictentry}{SimiNu}{n.}
        \dictdef{
            wash basin, sink, bathtub
        }
    \end{dictentry}
    \begin{dictentry}{SuNi}{n., adj.}
        \dictdef*{
            healthy glow to one's skin, looking healthy
        }
    \end{dictentry}
    \begin{dictentry}{uSiNi}{adj.}
        \dictdef*{
            feeling healthy after a period of illness
        }
    \end{dictentry}
\dictsubtitle{Compounds \& Secondary Derivations}
    \begin{dictentry}{SaNaarjan}{n.}
        \dictdef*{
            doctor
        }
    \end{dictentry}
\end{dictroot}

\begin{dictroot}{s}{r}
    % identical, same
    \begin{dictentry}{SaR}{n.}
        \dictdef{
            identity
        }
        \dictdef{
            equal
        }
    \end{dictentry}
    \begin{dictentry}{SiiR}{v.tr.}
        \dictdef*{
            to equal, be the same as, 
        }
    \end{dictentry}
    \begin{dictentry}{SiyaR}{v.intr.}
        \dictdef*{
            to equal, be equal, be the same as
        }
    \end{dictentry}
    \begin{dictentry}{iSaaR}{v.intr.}
        \dictdef*{
            to equate oneself with, to identify as
        }
    \end{dictentry}
    \begin{dictentry}{SiSiyaR}{v.tr.}
        \dictdef*{
            to identify \textit{smth.} or \textit{sme.}, assign an identity to \textit{sme.}
        }
    \end{dictentry}
    \begin{dictentry}{aaSiR}{v.tr.}
        \dictdef*{
            to recruit \textit{sme.}, convert \textit{sme.}
        }
    \end{dictentry}
    \begin{dictentry}{SaRa}{adj.}
        \dictdef*{
            identical, same
        }
    \end{dictentry}
    \begin{dictentry}{SanaR}{n.}
        \dictdef{
            recruit, convert
        }
        \dictdef{
            private, most junior rank of military personnel
        }
        \dictdef{
            dictator, monarch, totalitarian leader of a state or kingdom
        }
    \end{dictentry}
    \begin{dictentry}{SurRi}{n.}
        \dictdef*{
            duplicate
        }
    \end{dictentry}
    \begin{dictentry}{iSuRa}{n.}
        \dictdef*{
            totalitarion regime, dictatorship, absolute monarchy
        }
    \end{dictentry}
    \begin{dictentry}{mSiR}{n.}
        \dictdef*{
            identity documentation
        }
    \end{dictentry}
    \begin{dictentry}{inSiR}{n.}
        \dictdef*{
            resemblance
        }
    \end{dictentry}
    \begin{dictentry}{SuliR}{n.}
        \dictdef{
            fingerprint
        }
        \dictdef{
            \textit{(compounded with other body parts)} unique natural pattern on one's bodypart
        }
    \end{dictentry}
    \begin{dictentry}{SuRu}{n.}
        \dictdef*{
            naturally camouflaged animal
        }
    \end{dictentry}
    \begin{dictentry}{SasiR}{n.}
        \dictdef{
            signpost
        }
        \dictdef{
            more specific i7u7a thing?
        }
    \end{dictentry}
    \begin{dictentry}{SajuRa}{n.}
        \dictdef*{
            paper copy, paper scan, fax
        }
    \end{dictentry}
    \begin{dictentry}{aSiRu}{n.}
        \dictdef*{
            category
        }
    \end{dictentry}
    \begin{dictentry}{SimiRu}{n.}
        \dictdef*{
            type, kind, sort
        }
    \end{dictentry}
    \begin{dictentry}{SuRi}{n., adj.}
        \dictdef*{
            camouflage, camouflaged
        }
    \end{dictentry}
    \begin{dictentry}{uSiRi}{adj.}
        \dictdef*{
            in agreement, of the same opinion, from the same perspective
        }
    \end{dictentry}
\end{dictroot}

\begin{dictroot}{s}{w}
    % grandparent-grandchild (close)
    \begin{dictentry}{SanaW}{n.}
        \dictdef{
            grandparent
        }
        \dictdef{
            grandchild
        }
    \end{dictentry}
    \begin{dictentry}{aSiWu}{n.}
        \dictdef*{
            extended family, family including grandparents, aunts/uncles, cousins, etc.
        }
    \end{dictentry}
    \dictsubtitle{Compounds \& Secondary Derivations}
    \begin{dictentry}{aSiWuana}{n.}
        \dictdef*{a family member, a relative, a member of one's extended family}
    \end{dictentry}
\end{dictroot}

\begin{dictroot}{sf}{r}
    %Swede, Sweden
    \begin{dictentry}{SFaR}{n.}
        \dictdef*{
            trouble, problem, hindrance, issue, bug
        }
    \end{dictentry}
    \begin{dictentry}{SFaRa}{adj.}
        \dictdef{
            troublesome, problematic
        }
        \dictdef{
            Swedish\\
            \textit{NB: though the sf--r meaning `Sweden,' is etymologically unrelated to the sf--r root meaning `trouble,' their homophony is often exploited for wordplay.}
        }
    \end{dictentry}
    \begin{dictentry}{SFurRi}{n.}
        \dictdef*{
            rutabaga, Swede, yellow turnip, Swedish turnip, the root of \textit{Brassica napus}
        }
    \end{dictentry}
    \begin{dictentry}{SFasiR}{n.}
        \dictdef*{
            rutabaga greens, the leaves of \textit{Brassica napus}
        }
    \end{dictentry}
    \begin{dictentry}{SFuuR}{n.}
        \dictdef*{
            the Swedish people\\
            \textit{from Old Norse \emph{svíar}}
        }
    \end{dictentry}
    \begin{dictentry}{SFaRia}{n.}
        \dictdef*{
            Sweden
        }
    \end{dictentry}
\end{dictroot}

\begin{dictroot}{sh}{n}
    \begin{dictentry}{SHaNa}{adj.}
        \dictdef*{
            English, of or related to England
        }
    \end{dictentry}
    \begin{dictentry}{SHuuN}{n.}
        \dictdef*{
            England, the English\\
            \textit{from Proto-Germanic \emph{\*sahsô} `Saxon' via Middle Irish \emph{Saxain}}
        }
    \end{dictentry}
\end{dictroot}

\begin{dictroot}{sk}{l}
    %learn, study
    \begin{dictentry}{SKaL}{n.}
        \dictdef*{
            learning, education, knowledge
        }
    \end{dictentry}
    \begin{dictentry}{SKiiL}{v.tr.}
        \dictdef*{
            to learn \textit{smth.}, to study \textit{smth.}
        }
    \end{dictentry}
    \begin{dictentry}{SKiyaL}{v.intr.}
        \dictdef*{
            \textit{(of a subject)} to be taught, be studied, to be researched
        }
    \end{dictentry}
    \begin{dictentry}{iSKaaL}{v.intr.}
        \dictdef*{
            to study, to perform research, to learn
        }
    \end{dictentry}
    \begin{dictentry}{SKiSKiyaL}{v.tr.}
        \dictdef*{
            to teach \textit{smth.}, to instruct about \textit{smth.}
        }
    \end{dictentry}
    \begin{dictentry}{aaSKiL}{v.tr.}
        \dictdef*{
            to teach \textit{sme.}, to instruct \textit{sme.}
        }
    \end{dictentry}
    \begin{dictentry}{SKaLa}{adj.}
        \dictdef{
            educational, of or relating to education
        }
        \dictdef{
            instructive, aiding study
        }
        \dictdef{
            \textit{(of a person)} studious
        }
    \end{dictentry}
    \begin{dictentry}{SKanaL}{n.}
        \dictdef*{
            student, pupil
        }
    \end{dictentry}
    \begin{dictentry}{iSKuLa}{n.}
        \dictdef*{
            school
        }
    \end{dictentry}
    \begin{dictentry}{mSKiL}{n.}
        \dictdef*{
            notebook
        }
    \end{dictentry}
    \begin{dictentry}{inSKiL}{n.}
        \dictdef*{
            class, course
        }
    \end{dictentry}
    \begin{dictentry}{SKuliL}{n.}
        \dictdef*{
            the mind, the intellect
        }
    \end{dictentry}
    \begin{dictentry}{SKuLu}{n.}
        \dictdef*{
            dolphin
        }
    \end{dictentry}
    \begin{dictentry}{SKasiL}{n.}
        \dictdef*{
            pointer
        }
    \end{dictentry}
    \begin{dictentry}{SKajuLa}{n.}
        \dictdef{
            blackboard, chalkboard
        }
        \dictdef{
            whiteboard, dry-eraseboard
        }
    \end{dictentry}
    \begin{dictentry}{SKidiL}{n.}
        \dictdef*{
            fact
        }
    \end{dictentry}
    \begin{dictentry}{aSKiLu}{n.}
        \dictdef{
            textbook
        }
        \dictdef{
            scientific paper
        }
    \end{dictentry}
    \begin{dictentry}{SKimiLu}{n.}
        \dictdef*{
            lecture, talk, presentation
        }
    \end{dictentry}
    \begin{dictentry}{uSKiLi}{adj.}
        \dictdef*{
            arrogant in the form of perceived superior intellect
        }
    \end{dictentry}
\dictsubtitle{Compounds \& Secondary Derivations}
    \begin{dictentry}{aaSKiLana}{n.}
        \dictdef*{
            teacher, lecturer
        }
    \end{dictentry}
\end{dictroot}

\begin{dictroot}{sk}{nd}
    % Scandinavian, Norse, North Germanic
    \begin{dictentry}{SKaNDa}{adj.}
        \dictdef*{
            Scandinavian, Nordic, North Germanic
        }
    \end{dictentry}
    \begin{dictentry}{SKanaND}{n.}
        \dictdef*{
            Scandinavian person, person from the Nordic countries, Northern European, North Germanic
        }
    \end{dictentry}
    \begin{dictentry}{iSKuNDa}{n.}
        \dictdef{
            Scandinavia, the Scandinavian peninsula, the Nordic countries
        }
    \end{dictentry}
    \begin{dictentry}{SKuuND}{n.}
        \dictdef*{
            Scandinavian people, people from Nordic countries, Northern Europeans, North Germanic peoples (as a monolith)
        }
    \end{dictentry}
\end{dictroot}

\begin{dictroot}{sl}{s}
    \begin{dictentry}{SLaS}{n.}
        \dictdef*{security, safety}
    \end{dictentry}
    \begin{dictentry}{SLiiS}{v.tr.}
        \dictdef*{to shut \emph{smth.} inside a container or behind a door}
    \end{dictentry}
    \begin{dictentry}{SLiyaS}{v.intr.}
        \dictdef{to be closed, to be locked}
        \dictdef{
            to be pickled, ferment in brine or vinegar
        }
    \end{dictentry}
    \begin{dictentry}{iSLaaS}{v.intr.}
        \dictdef{to lock oneself up, to shut oneself in, to close oneself away}
        \dictdef{to seclude oneself, to isolate onself, to cloister oneself}
    \end{dictentry}
    \begin{dictentry}{SLiSLiyaS}{v.tr.}
        \dictdef{to close, to shut, to lock}
        \dictdef{to pickle \textit{smth.}}
    \end{dictentry}
    \begin{dictentry}{aaSLiS}{v.tr.}
        \dictdef{to lock \textit{sme.} up, to shut \textit{sme.} in}
        \dictdef{to seclude \textit{sme.}, to isolate \textit{sme.}from others}
    \end{dictentry}
    \begin{dictentry}{SLaSa}{adj.}
        \dictdef*{secured, contained, safe}
    \end{dictentry}
    \begin{dictentry}{SLanaS}{n.}
        \dictdef{security guard}
        \dictdef{bank teller}
    \end{dictentry}
    \begin{dictentry}{SLurSi}{n.}
        \dictdef*{lock}
    \end{dictentry}
    \begin{dictentry}{SLarSi}{n.}
        \dictdef{
            brine, vinegar, liquid used in pickling 
        }
    \end{dictentry}
    \begin{dictentry}{iSLuSa}{n.}
        \dictdef{fortress}
        \dictdef{saferoom}
    \end{dictentry}
    \begin{dictentry}{mSLiS}{n.}
        \dictdef*{key}
    \end{dictentry}
    \begin{dictentry}{SLasiS}{n.}
        \dictdef*{zipper}
    \end{dictentry}
    \begin{dictentry}{SLajuSa}{n.}
        \dictdef{door, trap door, gate}
        \dictdef{tarp or other similar object used to close something off}
    \end{dictentry}
    \begin{dictentry}{SLidiS}{n.}
        \dictdef{
            salt
        }
    \end{dictentry}
    \begin{dictentry}{aSLiSu}{n.}
        \dictdef*{safe, lockbox, vault}
    \end{dictentry}
    \begin{dictentry}{SLimiSu}{n.}
        \dictdef*{a fenced-in area, an area contained by a fence}
    \end{dictentry}
    \begin{dictentry}{uSLiSi}{adj.}
        \dictdef{\textit{(of a person)} closed-off, repressed, reserved}
        \dictdef{confined, depressed, feeling trapped}
    \end{dictentry}
    \begin{dictentry}{SLuSaw}{n., adv.}
        \dictdef{
            north
        }
        \dictdef{
            northwards, to the north, in the north
        }
    \end{dictentry}
\end{dictroot}

\begin{dictroot}{sn}{m}
    \begin{dictentry}{SNaMa}{adj.}
        \dictdef*{
            tasty, delicious, good to eat, good-tasting
        }
    \end{dictentry}
\end{dictroot}

\begin{dictroot}{sw}{t}
    %Swiss, Switzerland
    \begin{dictentry}{iSWuTa}{n.}
        \dictdef*{
            Switzerland
        }
    \end{dictentry}
    \begin{dictentry}{SWuuT}{n.}
        \dictdef*{
            the Swiss people
        }
    \end{dictentry}
\end{dictroot}

%%%%%%%%%%%
%    T
%%%%%%%%%%%
\section*{T}

\begin{dictroot}{t}{fr}
    % parent-child (distance)
    \begin{dictentry}{TanaFR}{n.}
        \dictdef{
            parent considered less close due to not being biologically related or other reason (e.g., stepparent, parent-in-law, etc.)
        }
        \dictdef{
            child considered less close due to not being biologically related or other reason (e.g., stepchild, child-in-law, etc.)
        }
    \end{dictentry}
\end{dictroot}

\begin{dictroot}{t}{h}
    %Germany
    \begin{dictentry}{TaHa}{adj.}
        \dictdef*{
            German
        }
    \end{dictentry}
    \begin{dictentry}{TuuH}{n.}
        \dictdef{
            Germany
        }
        \dictdef{
            the German people
        }
    \end{dictentry}
\end{dictroot}

\begin{dictroot}{t}{k}
    % useful, to use, busy
    \begin{dictentry}{TiiK}{v.tr.}
        \dictdef*{
            to use \textit{smth.}, to benefit from \textit{smth.}
        }
    \end{dictentry}
    \begin{dictentry}{TiyaK}{v.intr.}
        \dictdef*{
            to be useful, to be of use, to come in handy, to benefit
        }
    \end{dictentry}
    \begin{dictentry}{iTaaK}{v.intr.}
        \dictdef*{
            to make oneself useful
        }
    \end{dictentry}
    \begin{dictentry}{TiTiyaK}{v.tr.}
        \dictdef{
            to set up, to put to use
        }
        \dictdef{
            to put into practice
        }
        \dictdef{
            to renovate, to repair, to refurbish
        }
    \end{dictentry}
    \begin{dictentry}{aaTiK}{v.tr.}
        \dictdef*{to manage \textit{sme.}, to boss \textit{sme.} around, to motive \textit{sme.}
        }
    \end{dictentry}
    \begin{dictentry}{TaKa}{adj.}
        \dictdef*{
            useful, helpful, beneficial
        }
    \end{dictentry}
    \begin{dictentry}{TanaK}{n.}
        \dictdef*{
            butler, maid, housekeeper
        }
    \end{dictentry}
    \begin{dictentry}{iTuKa}{n.}
        \dictdef*{
            job site, workplace
        }
    \end{dictentry}
    \begin{dictentry}{mTiK}{n.}
        \dictdef*{
            tool, instrument
        }
    \end{dictentry}
    \begin{dictentry}{TuKu}{n.}
        \dictdef*{
            beast of burden
        }
    \end{dictentry}
    \begin{dictentry}{uTiKi}{adj.}
        \dictdef*{busy, hard at work}
    \end{dictentry}
\end{dictroot}

\begin{dictroot}{t}{lw}
    \begin{dictentry}{\famword{TaLW}}{n.}
        \dictdef{
            the land, like, as an idea
        }
        \dictdef{
            job, labor, employment
        }
    \end{dictentry}
    \begin{dictentry}{TiiLW}{v.tr.}
        \dictdef{
            to dig \textit{smth.} up, to unearth
        }
        \dictdef{
            to till \textit{smth.}, to plow \textit{smwh.}
        }
        \dictdef{
            to discover \textit{smth.}, to unearth \textit{smth.}
        }
    \end{dictentry}
    \begin{dictentry}{TiyaLW}{v.intr.}
        \dictdef{
            to be unearthed, to be unburied
        }
        \dictdef{
            to lie in the ground, to be buried, to be underground
        }
        \dictdef{
            to be discovered, to be found out
        }
        \dictdef{
            to be agreed upon, to be mutually accepted
        }
    \end{dictentry}
    \begin{dictentry}{iTaaLW}{v.intr.}
        \dictdef{
            to till, plow a field
        }
        \dictdef{
            to dig
        }
        \dictdef{
            to work, especially manual labor
        }
        \dictdef{
            to come to an agreement
        }
    \end{dictentry}
    \begin{dictentry}{TiTiyaLW}{v.tr.}
        \dictdef{
            to bury \textit{smth.}
        }
        \dictdef{
            to come to an agreement on \textit{smth.}
        }
    \end{dictentry}
    \begin{dictentry}{aaTiLW}{v.tr.}
        \dictdef{
            to drive or stick \textit{smth.} into the ground
        }
        \dictdef{
            to hire, to employ \textit{sme.} 
        }
    \end{dictentry}
    \begin{dictentry}{TaLWa}{adj.}
        \dictdef{
            earthly, not of heavenly or extraterrestrial quality
        }
        \dictdef{
            regular, ordinary, every-day, humdrum
        }
    \end{dictentry}
    \begin{dictentry}{TanaLW}{n.}
        \dictdef{
            farmer
        }
        \dictdef{
            employee, worker
        }
        \dictdef{
            groundskeeper, gardener
        }
    \end{dictentry}
    \begin{dictentry}{TurLWi}{n.}
        \dictdef*{
            potato
        }
    \end{dictentry}
    \begin{dictentry}{TarLWi}{n.}
        \dictdef*{
            mud
        }
    \end{dictentry}
    \begin{dictentry}{iTuLWa}{n.}
        \dictdef*{
            piece/section of land
        }
    \end{dictentry}
    \begin{dictentry}{mTiLW}{n.}
        \dictdef*{
            spade, shovel
        }
    \end{dictentry}
    \begin{dictentry}{inTiLW}{n.}
        \dictdef*{
            island
        }
    \end{dictentry}
    \begin{dictentry}{TuLWu}{n.}
        \dictdef*{
            mole
        }
    \end{dictentry}
    \begin{dictentry}{TasiLW}{n.}
        \dictdef{
            hill, mound
        }
        \dictdef{
            worm
        }
    \end{dictentry}
    \begin{dictentry}{TajuLWa}{n.}
        \dictdef{
            the ground, soil
        }
        \dictdef{
            topsoil
        }
        \dictdef{
            field, cultivated land
        }
        \dictdef{
            forest floor
        }
    \end{dictentry}
    \begin{dictentry}{TidiLW}{n.}
        \dictdef*{
            actual dirt
        }
    \end{dictentry}
    \begin{dictentry}{aTiLWu}{n.}
        \dictdef*{
            grave, burial pit
        }
    \end{dictentry}
    \begin{dictentry}{TimiLWu}{n.}
        \dictdef*{
            valley
        }
    \end{dictentry}
    \begin{dictentry}{TuLWi}{n., adj.}
        \dictdef*{
            brown, the color brown
        }
    \end{dictentry}
    \begin{dictentry}{usTaLW}{n.}
        \dictdef*{
            jeans, denim pants, work pants
        }
    \end{dictentry}
    \begin{dictentry}{TuLWaw}{n., adv.}
        \dictdef{
            the right hand
        }
        \dictdef{
            to the right, on the right side
        }
    \end{dictentry}
\end{dictroot}

\begin{dictroot}{t}{t}
    % young, new
    \begin{dictentry}{TaT}{n.}
        \dictdef{
            news
        }
        \dictdef{
            youth
        }
    \end{dictentry}
    \begin{dictentry}{TiiT}{v.tr.}
        \dictdef*{
            to introduce, debut \textit{smth.} or \textit{sme.}
        }
    \end{dictentry}
    \begin{dictentry}{TiyaT}{v.intr.}
        \dictdef{
            to be introduced, debut, begin
        }
    \end{dictentry}
    \begin{dictentry}{iTaaT}{v.intr.}
        \dictdef{
            to introduce oneself
        }
        \dictdef{
            to involve oneself in an activity or event
        }
    \end{dictentry}
    \begin{dictentry}{TaTa}{adj.}
        \dictdef{
            new, recent
        }
        \dictdef{
            young
        }
    \end{dictentry}
    \begin{dictentry}{TanaT}{n.}
        \dictdef{
            youngster, youth, teenager, adolescent
        }
        \dictdef{
            noob, newbie, new guy
        }
    \end{dictentry}
    \begin{dictentry}{TurTi}{n.}
        \dictdef*{
            piece of information
        }
    \end{dictentry}
    \begin{dictentry}{iTuTa}{n.}
        \dictdef{
            frontier
        }
        \dictdef{
            \textit{(figuratively)} forefront
        }
    \end{dictentry}
    \begin{dictentry}{mTiT}{n.}
        \dictdef{
            technology, the accumulated tools of a culture
        }
        \dictdef{
            innovative, novel tool
        }
    \end{dictentry}
    \begin{dictentry}{inTiT}{n.}
        \dictdef*{
            gossip, hearsay
        }
    \end{dictentry}
    \begin{dictentry}{TuliT}{n.}
        \dictdef*{
            growth, cyst, tumor, cancer
        }
    \end{dictentry}
    \begin{dictentry}{TuTu}{n.}
        \dictdef*{
            some kind of vocal animal whose cry is symbolic or indicative of something
        }
    \end{dictentry}
    \begin{dictentry}{TasiT}{n.}
        \dictdef{
            conversation thread
        }
        \dictdef{
            timeline, newsfeed
        }
        \dictdef{
            chat channel
        }
    \end{dictentry}
    \begin{dictentry}{TajuTa}{n.}
        \dictdef*{
            newspaper
        }
    \end{dictentry}
    \begin{dictentry}{TidiT}{n.}
        \dictdef*{
            hint, small piece of information
        }
    \end{dictentry}
    \begin{dictentry}{aTiTu}{n.}
        \dictdef*{
            television or radio news broadcast
        }
    \end{dictentry}
    \begin{dictentry}{TuTi}{n., adj.}
        \dictdef*{
            black-and-white, monochrome
        }
    \end{dictentry}
    \begin{dictentry}{uTiTi}{adj.}
        \dictdef{
            inexperienced
        }
        \dictdef{
            overwhelmed due to inexperience
        }
    \end{dictentry}
    \begin{dictentry}{TaTjalaT}{n.}
        \dictdef{
            T-shirt
        }
    \end{dictentry}
    \begin{dictentry}{TuuT}{n.}
        \dictdef{
            culture and people group that is not ethnically Narelandic
        }
        \dictdef{
            \textit{(derogatory)} immigrants, refugees 
        }
    \end{dictentry}
\end{dictroot}

\begin{dictroot}{tr}{l}
    \begin{dictentry}{TRanaL}{n.}
        \dictdef*{troll, ogre}
    \end{dictentry}
\end{dictroot}

%%%%%%%%%%%
%    W
%%%%%%%%%%%
\section*{W}

\begin{dictroot}{w}{b}
    % to come, to come back
    \begin{dictentry}{WaB}{n.}
        \dictdef*{
            arrival, return, appearance
        }
    \end{dictentry}
    \begin{dictentry}{WiiB}{v.tr.}
        \dictdef*{
            to arrive \textit{smwh.}, attend \textit{smth.}
        }
    \end{dictentry}
    \begin{dictentry}{WiyaB}{v.intr.}
        \dictdef{
            \textit{(of an object)} to arrive, show up, appear
        }
        \dictdef{
            \textit{(of an object)} to return, come back to a place
        }
    \end{dictentry}
    \begin{dictentry}{iWaaB}{v.intr.}
        \dictdef{
            to arrive, show up, appear
        }
        \dictdef{
            to return, come back to a place
        }
    \end{dictentry}
    \begin{dictentry}{WiWiyaB}{v.tr.}
        \dictdef*{
            to deliver, bring, send, retrieve \textit{smth.}
        }
    \end{dictentry}
    \begin{dictentry}{aaWiB}{v.tr.}
        \dictdef{
            to guide, escort \textit{sme.}
        }
        \dictdef{
            to send, bring, retrieve \textit{sme.}
        }
    \end{dictentry}
    \begin{dictentry}{WaBa}{adj.}
        \dictdef*{
            recurring, returning, repeating
        }
    \end{dictentry}
    \begin{dictentry}{WanaB}{n.}
        \dictdef*{
            guest, regular
        }
    \end{dictentry}
    \begin{dictentry}{iWuBa}{n.}
        \dictdef{
            the arrivals section of a station, airport
        }
        \dictdef{
            foyer, entrance hall, lobby
        }
    \end{dictentry}
    \begin{dictentry}{WuBu}{n.}
        \dictdef*{
            migratory animal of some kind, gotta figure out
        }
    \end{dictentry}
    \begin{dictentry}{WasiB}{n.}
        \dictdef*{
            boomerang
        }
    \end{dictentry}
    \begin{dictentry}{WajuBa}{n.}
        \dictdef*{
            door mat, welcome mat
        }
    \end{dictentry}
    \begin{dictentry}{WidiB}{n.}
        \dictdef{
            \textit{(colloquial)} the first snow of the winter
        }
        \dictdef{
            \textit{(colloquial, poetic)} a period of snowfall during springtime, after the snow from winter has melted away
        }
    \end{dictentry}
    \begin{dictentry}{WuBi}{n.,adj.}
        \dictdef{
            reflection, mirror image
        }
        \dictdef{
            reflected, mirrored
        }
    \end{dictentry}
    \begin{dictentry}{uWiBi}{adj.}
        \dictdef{
            self-conscious, introspective
        }
    \end{dictentry}
\end{dictroot}

\begin{dictroot}{w}{h}
    % sex
    \begin{dictentry}{WaH}{n.}
        \dictdef*{
            sex
        }
    \end{dictentry}
    \begin{dictentry}{WiiH}{v.tr.}
        \dictdef*{
            to have sex with \textit{sme.}
        }
    \end{dictentry}
    \begin{dictentry}{WiyaH}{v.intr.}
        \dictdef*{
            to get off, experience sexual satisfaction, have an orgasm
        }
    \end{dictentry}
    \begin{dictentry}{iWaaH}{v.intr.}
        \dictdef*{
            to have sex
        }
    \end{dictentry}
    \begin{dictentry}{WiWiyaH}{v.tr.}
        \dictdef*{
            to get \textit{sme.} off, make \textit{sme.} orgasm
        }
    \end{dictentry}
    \begin{dictentry}{aaWiH}{v.tr.}
        \dictdef*{
            to seduce \textit{sme.}
        }
    \end{dictentry}
    \begin{dictentry}{WaHa}{adj.}
        \dictdef*{
            sexy, sexual
        }
    \end{dictentry}
    \begin{dictentry}{WanaH}{n.}
        \dictdef*{
            sexual partner
        }
    \end{dictentry}
    \begin{dictentry}{WarHi}{n.}
        \dictdef*{
            lubricant
        }
    \end{dictentry}
    \begin{dictentry}{mWiH}{n.}
        \dictdef*{
            sex toy
        }
    \end{dictentry}
    \begin{dictentry}{inWiH}{n.}
        \dictdef*{
            quickie, brief sexual encounter
        }
    \end{dictentry}
    \begin{dictentry}{WuliH}{n.}
        \dictdef*{
            genitals
        }
    \end{dictentry}
    \begin{dictentry}{WuHu}{n.}
        \dictdef*{
            masturbatory aid, fleshlight
        }
    \end{dictentry}
    \begin{dictentry}{WasiH}{n.}
        \dictdef*{
            penis
        }
    \end{dictentry}
    \begin{dictentry}{WidiH}{n.}
        \dictdef*{
            Viagra, Cialis, drug used to enhance sexual performance
        }
    \end{dictentry}
    \begin{dictentry}{aWiHu}{n.}
        \dictdef*{
            vagina
        }
    \end{dictentry}
    \begin{dictentry}{WimiHu}{n.}
        \dictdef*{
            condom
        }
    \end{dictentry}
    \begin{dictentry}{uWiHi}{adj.}
        \dictdef*{
            horny, sexually aroused
        }
    \end{dictentry}
\end{dictroot}

\begin{dictroot}{w}{l}
    % old, elderly
    % also to forget
    \begin{dictentry}{WaL}{n.}
        \dictdef{
            old age, seniority
        }
        \dictdef*{
            the distant past
        }
    \end{dictentry}
    \begin{dictentry}{WiiL}{v.tr.}
        \dictdef*{
            to forget \textit{smth.} from deep in one's memory, long ago
        }
    \end{dictentry}
    \begin{dictentry}{WiyaL}{v.intr.}
        \dictdef*{
            to get lost to time, be forgotten into legend
        }
    \end{dictentry}
    \begin{dictentry}{iWaaL}{v.intr.}
        \dictdef*{
            to forget, be forgetful
        }
    \end{dictentry}
    \begin{dictentry}{WiWiyaL}{v.tr.}
        \dictdef*{
            to surpress or censor \textit{smth.}, \textit{sme.}
        }
    \end{dictentry}
    \begin{dictentry}{aaWiL}{v.tr.}
        \dictdef*{
            to cause \textit{sme.} to feel the inevitable decay of age and senility
        }
    \end{dictentry}
    \begin{dictentry}{WaLa}{adj.}
        \dictdef{
            old, not recent
        }
        \dictdef{
            elderly
        }
    \end{dictentry}
    \begin{dictentry}{WanaL}{n.}
        \dictdef*{
            old person, senior citizen
        }
    \end{dictentry}
    \begin{dictentry}{iWuLa}{n.}
        \dictdef*{
            old times, the good old days
        }
    \end{dictentry}
    \begin{dictentry}{aWiLu}{n.}
        \dictdef*{
            retirement home
        }
    \end{dictentry}
    \begin{dictentry}{uWiLi}{adj.}
        \dictdef*{
            feeling old, decaying, fading
        }
    \end{dictentry}
\end{dictroot}

\begin{dictroot}{w}{w}
    \begin{dictentry}{WaW}{n.}
        \dictdef{
            sweetness, sugaryness
        }
        \dictdef{
            flirtation, flirtiness
        }
    \end{dictentry}
    \begin{dictentry}{WiiW}{v.tr.}
        \dictdef{
            to flirt with \textit{sme.}
        }
        \dictdef{
            to suck up to \textit{sme.}, to flatter \textit{sme.}
        }
    \end{dictentry}
    \begin{dictentry}{WiyaW}{v.intr.}
        \dictdef*{
            to be flattering or flirtatious
        }
        \dictdef*{
            \textit{(of an object, particularly clothes)} to be flattering, suit one's figure or style
        }
    \end{dictentry}
    \begin{dictentry}{iWaaW}{v.intr.}
        \dictdef*{
            to flirt, flatter
        }
    \end{dictentry}
    \begin{dictentry}{WiWiyaW}{v.tr.}
        \dictdef*{
            to interpret something as flattering
        }
    \end{dictentry}
    \begin{dictentry}{\famword{WaWa}}{adj.}
        \dictdef{
            sweet, sugary, saccharine
        }
        \dictdef{
            sweet, docile, lovely, adorable
        }
    \end{dictentry}
    \begin{dictentry}{WanaW}{n.}
        \dictdef*{
            dear, honey, sweetie, hon, sugar
        }
    \end{dictentry}
    \begin{dictentry}{\famword{WarWi}}{n.}
        \dictdef*{
            sweet drink, soft drink, sugary beverage
        }
    \end{dictentry}
    \begin{dictentry}{WuWu}{n.}
        \dictdef*{
            pet animal 
        }
    \end{dictentry}
    \begin{dictentry}{\famword{WidiW}}{n.}
        \dictdef*{
            sugar, powdered sweetener
        }
    \end{dictentry}
    \begin{dictentry}{uWiWi}{adj.}
        \dictdef*{
            flattered
        }
    \end{dictentry}
\dictsubtitle{Compounds \& Secondary Derivations}
    \begin{dictentry}{\famword{WarWibin}}{n.}
        \dictdef*{
            honey
        }
    \end{dictentry}
\end{dictroot}

\begin{dictroot}{wr}{p}
    %Europe
    %iWRuPa = continent
    %WRaPia = EU
    %WRuuP = white people?
    \begin{dictentry}{iWRuPa}{n.}
        \dictdef*{
            Europe
        }
    \end{dictentry}
    \begin{dictentry}{WRuuP}{n.}
        \dictdef*{
            \textit{(informal)} white people, Europeans
        }
    \end{dictentry}
    \begin{dictentry}{WRaPia}{n.}
        \dictdef*{
            \textit{(colloquial)} the European Union
        }
    \end{dictentry}
\end{dictroot}

\begin{dictroot}{ws}{n}
    % reason, happening
    \begin{dictentry}{WSaN}{n.}
        \dictdef{
            reason
        }
        \dictdef{
            cause
        }
    \end{dictentry}
    \begin{dictentry}{WeSiiN}{v.tr.}
        \dictdef*{
            to bring about, cause, effect
        }
    \end{dictentry}
    \begin{dictentry}{WeSiyaN}{v.intr.}
        \dictdef*{
            to happen, occur
        }
    \end{dictentry}
    \begin{dictentry}{iWSaaN}{v.tr.}
        \dictdef*{
            to do \textit{smth.}
        }
    \end{dictentry}
    \begin{dictentry}{aaWSiN}{v.tr.}
        \dictdef{
            to inspire, motivate, cause to act, coerce
        }
    \end{dictentry}
    \begin{dictentry}{WSaNa}{adj.}
        \dictdef*{
            current, relevant
        }
    \end{dictentry}
    \begin{dictentry}{WSanaN}{n.}
        \dictdef*{
            participant, actor
        }
    \end{dictentry}
    \begin{dictentry}{iWSuNa}{n.}
        \dictdef*{
            scene, place where something has happened or is happening
        }
    \end{dictentry}
    \begin{dictentry}{mWSiN}{n.}
        \dictdef*{
            method, manner
        }
    \end{dictentry}
    \begin{dictentry}{inWSiN}{n.}
        \dictdef*{
            excuse
        }
    \end{dictentry}
    \begin{dictentry}{uWSiNi}{adj.}
        \dictdef*{
            involved, invested
        }
    \end{dictentry}
\end{dictroot}

\end{multicols*}

\chapter{Rootless Words}
\begin{multicols*}{2}
\section{Verbs}
\begin{description}[leftmargin=*,labelwidth=*]
    \begin{dictentry}{dak}{mod.}
        \dictdef*{can, to be able to, to be allowed to}
    \end{dictentry}
    \begin{dictentry}{fidul}{asp.}
        \dictdef*{to be about to, to be going to}
    \end{dictentry}
    \begin{dictentry}{hak}{asp.}
        \dictdef*{to keep X-ing, to continue, to still X}
    \end{dictentry}
    \begin{dictentry}{hwii}{mod.}
        \dictdef*{not, no, don't, never}
    \end{dictentry}
    \begin{dictentry}{jaa}{asp.}
        \dictdef*{already, previously, by now}
    \end{dictentry}
    \begin{dictentry}{yin}{asp.}
        \dictdef*{just, recently, to have just done}
    \end{dictentry}
    \begin{dictentry}{kaj}{mod.}
        \dictdef*{to want, to want to}
    \end{dictentry}
    \begin{dictentry}{lit}{asp.}
        \dictdef*{to very, to do emphatically or to an extreme extent}
    \end{dictentry}
    \begin{dictentry}{naw}{asp.}
        \dictdef*{to stop, to cease, to quit}
    \end{dictentry}
    \begin{dictentry}{tuuq}{mod.}
        \dictdef*{to must, to have to}
    \end{dictentry}
    \begin{dictentry}{usnak}{mod.}
        \dictdef*{let it be so, \textit{(hortative)}}
    \end{dictentry}
    \begin{dictentry}{wadan}{asp.}
        \dictdef*{and now, but now, \textit{(stresses newness/recentness of the start of an eventuality)}}
    \end{dictentry}
\end{description}

\section{Postpositions}

\begin{description}[leftmargin=*,labelwidth=*]
    \begin{dictentry}{baj}{postp.}
        \dictdef*{using, by means of}
    \end{dictentry}
    \begin{dictentry}{bir}{postp.}
        \dictdef*{hanging from, suspended from}
    \end{dictentry}
    \begin{dictentry}{daw}{postp.}
        \dictdef*{to, towards}
    \end{dictentry}
    \begin{dictentry}{dis}{postp.}
        \dictdef*{because, because of, due to}
    \end{dictentry}
    \begin{dictentry}{fidul}{postp.}
        \dictdef*{before, in front of}
    \end{dictentry}
    \begin{dictentry}{fit}{postp.}
        \dictdef*{in, at, on}
    \end{dictentry}
    \begin{dictentry}{fun}{postp.}
        \dictdef*{from, out of}
    \end{dictentry}
    \begin{dictentry}{fus}{postp.}
        \dictdef*{at/in the home of}
    \end{dictentry}
    \begin{dictentry}{jan}{postp.}
        \dictdef*{under, beneath}
    \end{dictentry}
    \begin{dictentry}{yin}{postp.}
        \dictdef*{behind, after, in back of}
    \end{dictentry}
    \begin{dictentry}{lu}{postp.}
        \dictdef*{comprising, being from, being a subset of, being part of, \textit{(used for body parts and picking things out of a group)}}
    \end{dictentry}
    \begin{dictentry}{rijt}{postp.}
        \dictdef*{encircling, wrapped around, around}
    \end{dictentry}
    \begin{dictentry}{tui}{postp.}
        \dictdef*{on/covering the surface of}
    \end{dictentry}
    \begin{dictentry}{udan}{postp.}
        \dictdef*{if, depending on}
    \end{dictentry}
    \begin{dictentry}{uwa}{postp.}
        \dictdef*{over, above}
    \end{dictentry}
    \begin{dictentry}{wan}{postp.}
        \dictdef*{owned by, possessed by, 's, \textit{(indicates ownership or association)}}
    \end{dictentry}
\end{description}

\section{Attitudinals}

\begin{description}[leftmargin=*,labelwidth=*]
    \begin{dictentry}{x}{int.}
        \dictdef*{
            \textit{pronounced} \bripa{ǀ}\\
            \textit{expression of exasperation or mild disapproval}\\
            \textit{roughly equivalent to `smh'}
        }
    \end{dictentry}
    \begin{dictentry}{\textrm{X}}{int.}
        \dictdef*{
            \textit{pronounced} \bripa{ǁ} or \bripa{ǂ}\\
            \textit{expression of extreme annoyance, frustration, and disapproval}\\
            \textit{roughly equivalent to `ffs'}
        }
    \end{dictentry}
\end{description}

\section{Others}

\begin{description}[leftmargin=*, labelwidth=*]
    \begin{dictentry}{carde}{}
        \dictdef*{
            sir, milord, sire, boss, ma'am, milady
            }
    \end{dictentry}
\end{description}
\vspace*{\fill}
Number of Words Currently in Dictionary: \thedictwordcount
\end{multicols*}

\newpage

\section{Numbers}

\begin{multicols*}{2}
    \subsection{Cardinal Numbers}
    \begin{description}[align=parrightcent,leftmargin=!,labelwidth=4cm]
        \item[sar] one
        \item[\phantom{b}pas] two
        \item[\phantom{p}bar] three
        \item[qad] four
        \item[kaj] five
        \item[sesar] six
        \item[\phantom{b}pepas] seven
        \item[\phantom{p}bebar] eight
        \item[qeqad] nine
        \item[kekaj] ten
        \item[kekaj sar] eleven
        \item[kekaj pas] twelve
        \item[kekaj bar] thirteen
        \item[kekaj qad] fourteen
        \item[barkaj] fifteen
        \item[barkaj sesar] sixteen
        \item[barkaj pepas] seventeen
        \item[barkaj bebar] eightteen
        \item[barkaj qeqad] nineteen
        \item[qadkaj] twenty\phantom{b}
        \item[...]
        \item[kajkaj] twenty-five
        \item[sesarkaj] thirty
        \item[pepaskaj] thirty-five
        \item[bebarkaj] forty
        \item[qeqadkaj] forty-five
        \item[\phantom{j}dam] fifty
        \item[dam kaj] fifty-five
        \item[dam kekaj] sixty
        \item[dam barkaj] sixty-five
        \item[...]
        \item[dam qeqadkaj qeqad] ninety-nine
        \item[\phantom{j}dedam] one hundred
        \item[...]
        \item[\phantom{j}bardam] 150
        \item[qadedam] 200
        \item[kajdam] 250
        \item[\phantom{j}sesardam] 300
        \item[pepasdam] 350
        \item[\phantom{j}bebardam] 400
        \item[qeqadedam] 450
        \item[jal] 500
        \item[jal dam] 550
        \item[jal dedam] 600
        \item[...]
        \item[jal qeqadedam qeqadkaj qeqad] 999
        \item[suusan] 1000
        \item[...]
        \item[\phantom{b}pasesuusan] 2000
        \item[\phantom{j}barsuusan] 3000
        \item[qadsuusan] 4000
        \item[kajsuusan] 5000
        \item[sesarsuusan] 6000
        \item[\phantom{b}pepasesuusan] 7000
        \item[\phantom{j}bebarsuusan] 8000
        \item[qeqadsuusan] 9000
        \item[kekajsuusan] 10.000
        \item[kekajsarsuusan] 11.000
        \item[kekajpasesuusan] 12.000
        \item[...]
        \item[miljan] 1.000.000
        \item[miljad] 1.000.000.000 (10$^{9}$)
        \item[biljan] 1.000.000.000.000 (10$^{12}$)
        \item[biljad] 10$^{15}$
        \item[triljan] 10$^{18}$
        \item[triljad] 10$^{21}$
        \item[kfadriljan] 10$^{24}$
        \item[kfadriljad] 10$^{27}$
        \item[kfintiljan] 10$^{30}$
        \item[kfintiljad] 10$^{33}$
    \end{description}

    \subsection{Fractions}

    Fractions are formed by treating the entire number as if it were a root in the \rootpart{}a\rootpart{} pattern, with the final `a' in the cardinal number treated as the `a' in the pattern. This pseudo-root is then put into the in\rootpart{}i\rootpart{} pattern. Some example fractions:
    \begin{description}[align=parright,leftmargin=!,labelwidth=3.8cm]
        \item[inpis] one-half
        \item[inbir] one-third
        \item[inqid] one-quarter
        \item[inkij] one-fifth
        \item[insesir] one-sixth
        \item[inpepis] one-seventh
        \item[inbebir] one-eighth
        \item[inqeqid] one-ninth
        \item[inkekij] one-tenth
        \item[inkekajsir] one-eleventh
        \item[inkekajpis] one-twelfth
        \item[...]
        \item[injaldamsesarkajpis] 1 / 582
        \item[...]
        \item[indedim] one-hundredth (10$^{-2}$)
        \item[insuusin] one-thousandth (10$^{-3}$)
        \item[inkekajsuusin] one-tenthousandth (10$^{-4}$)
        \item[inmiljin] one-millionth (10$^{-6}$)
        \item[etc.]
    \end{description}
    If one wishes to describe a fraction with something other than one in the numerator, one simply precedes the fractional numeral for the denominator with the cardinal form of the numerator. So, for instance, 355/113 would be said aloud as \textit{pepasdam kaj indedamkekajbir}.
    \vfill\null
\columnbreak
\subsection{Ordinal Numbers}

Ordinals are not distinguishes from cardinal numbers in any explicit morphological way. In contexts where ordinal numerals would be used in other languages, \lang{} uses either cardinal numberals or other adjectives/adverbs to describe the relationship.

\end{multicols*}

\part{Example Texts \& Translations}

\setsecnumdepth{subsubsection}
\settocdepth{subsubsection}

\end{document}