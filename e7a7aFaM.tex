\documentclass[a4paper,10pt,twoside,openright]{memoir}
\usepackage{multicol, multirow, array}
\usepackage{fontspec}
\usepackage{anyfontsize}
\usepackage[pagecolor=none,dvipsnames]{xcolor}
\usepackage{ragged2e}
\usepackage{amsmath}
\usepackage{amssymb}
\usepackage[hidelinks]{hyperref}
\usepackage{url}
\usepackage[margin=0.8in]{geometry}
\usepackage{float, hhline}
\usepackage{booktabs}
\usepackage{textcomp}
\usepackage{expex}
\usepackage[calc,english]{datetime2}

%-----CONFIGURATION------
%------------------------

\setmainfont{Charis SIL}[CharacterVariant=43:1]
\restylefloat{table}

\setsecnumdepth{subsubsection}
\settocdepth{subsubsection}

\lingset{glstyle=nlevel,numoffset=3em,textoffset=1.5em,exskip=.75ex,belowglpreambleskip=.25ex,aboveglftskip=.25ex}

\DTMnewdatestyle{eurodate}{%
    \renewcommand{\DTMdisplaydate}[4]{%
        \number##3.\nobreakspace%           day
        \DTMmonthname{##2}\nobreakspace%    month
        \number##1%                         year
    }%
    \renewcommand{\DTMDisplaydate}{\DTMdisplaydate}%
}

\DTMsetdatestyle{eurodate}

\renewcommand{\arraystretch}{1.4}

%-------COMMANDS---------
%------------------------

\newcommand{\lang}{ɁaɁa\textsc{f}a\textsc{m}}
\newcommand{\longv}{ː}
\newcommand{\sqbrack}[1]{$\langle$#1$\rangle$}
\newcommand{\ttilde}{\raise.17ex\hbox{$\scriptstyle\sim$}}
\newcommand{\rootpart}{$\Theta$}
\newcommand{\bigglot}{Ɂ}
\newcommand{\lilglot}{ɂ}
\newcommand{\nm}{\symbol{"2205}}
\newcommand{\tiebar}{͡}

%-------TITLE PAGE-------
%------------------------

\title{{\fontsize{80}{80}\selectfont \lang} \\ \Huge \sffamily A Reference Grammar}
\author{Bethany E. Toma, Knut F. K. Ulstrup}
\date{\today}

%--------MAIN DOC--------
%------------------------

\begin{document}

\pagecolor{Melon}
\maketitle
\pagecolor{white}

\frontmatter

\chapter{Foreword}

\lang{} is a constructed language.

\newpage

\tableofcontents

\mainmatter

\part{Grammar}

\chapter{Phonology}
\section{Consonants}

\begin{table}[ht]
    \centering
    \begin{tabular}{rcccccc}
    \toprule
            & Labial & Alveolar & Palatal & Velar & Uvular & Glottal \\
    \midrule
    Fortis & pʰ \ttilde{} p\tiebar ɸ & tʰ \ttilde{} t\tiebar s &
    % \multirow{2}{*}{c \ttilde{} c\tiebar ç} & \multirow{2}{*}{k \ttilde{} k\tiebar x} & \multirow{2}{*}{q \ttilde{} q\tiebar χ} & \multirow{2}{*}{ʔ} \\
    c \ttilde{} c\tiebar ç & k \ttilde{} k\tiebar x & q \ttilde{} q\tiebar χ & \multirow{2}{*}{ʔ} \\
     Lenis & p \ttilde{} b & t \ttilde{} d & & & & \\
    Fricative & ɸ & s & \multicolumn{4}{c}{ç \enspace \ttilde{} \enspace x \enspace \ttilde{} \enspace χ \enspace \ttilde{} \enspace ħ \enspace \ttilde{} \enspace h} \\
    Approximant & & l & j & w & & \\
    Nasal & m & n & & & & \\
    Rhotic & & \multicolumn{4}{c}{ɾ \enspace \ttilde{} \enspace r \enspace \ttilde{} \enspace ɹ \enspace \ttilde{} \enspace ɽ \enspace \ttilde{} \enspace ɻ \enspace \ttilde{} \enspace ʀ \enspace \ttilde{} \enspace ʁ } & \\
    \bottomrule
    \end{tabular}
    \caption{Phonemic Consonant Inventory}
    \label{tab:consinv}
\end{table}

\section{Vowels}

\begin{table}[ht]
    \centering
    \begin{tabular}{rccc}
    \toprule
          & Front & Central & Back \\
    \midrule
    Close & i (i\longv{}) & & u (u\longv) \\
    Mid   & & ə & \\
    Open  & & a (a\longv) & \\
    \bottomrule
    \end{tabular}
    \caption{Phonemic Vowel Inventory}
    \label{tab:vowelinv}
\end{table}

\subsection{Epenthetic schwa}

\section{Morphophonemics}

\section{Orthography}

\lang{} has two recognized orthographic conventions, both based on the Latin alphabet. Both conventions use marked letterforms to indicate which part of a word are part of the underlying root and which are grammatical markers.

By and large, both orthographic conventions attempt to use the most intuitive representation of a given phoneme. There are very few differences between the conventions. Fortis and lenis stops are written using the typical voiceless and voiced symbols, respectively, in both systems. The labial fricative is written as \sqbrack{f} and the dorsal fricative as \sqbrack{h}. The palatal approximant is written using \sqbrack{j}, and the rhotic is, of course, written as \sqbrack{r}.

\subsection{Formal script}

The formal script makes use of small-caps letterforms to highlight roots. In addition, it uses the glottal stop character to indicate the glottal stop phoneme, using the capital glottal stop character \sqbrack{\bigglot} when the glottal stop is part of a root radical and the lowercase glottal stop character \sqbrack{\lilglot} otherwise.

\subsection{Informal script}

The informal script, also known as ``texting script", is the orthography used in the majority of day-to-day communication, as it has the advantage of being  It makes use of capital letters for roots, uses \sqbrack{7} for glottal stops, and allows for several shorthand abbreviations. 

\chapter{Morphology}
\section{Underlying roots}
\section{Derivational morphology}

The root system

\begin{table}[ht]
    \centering
    \begin{tabular}{llll}
    \textit{Pattern} & \textit{Meaning} & \textit{Example} & \\
    \multirow{2}{*}{{\rootpart}a{\rootpart}} & \multirow{2}{*}{Abstract noun}& \textsc{s}a\textsc{j} & sleep \emph{(cf. i\textsc{s}aa\textsc{j} `to sleep')}\\
    & & \textsc{k}a\textsc{l} & rainfall \emph{(cf. \textsc{k}ur\textsc{l}i `raindrop')}\\
    {\rootpart}ana{\rootpart} & Person of X, Agentive noun & \textsc{k}ana\textsc{j} & author \emph{(cf. \textsc{k}ii\textsc{j} `to write X')}\\
    {\rootpart}ar{\rootpart}i & Liquid noun & \textsc{q}ar\textsc{f}i & coffee \emph{(cf. i\textsc{q}aa\textsc{f} `to drink coffee')} \\
    {\rootpart}ur{\rootpart}i & Object noun & \textsc{n}ur\textsc{m}i & food \emph{(cf. \textsc{n}ii\textsc{m} `to eat X')}\\
    {\rootpart}idi{\rootpart} & Loose granular mass & \textsc{w}idi\textsc{w} & sugar \emph{(cf. \textsc{w}a\textsc{w}a `sweet')} \\
    {\rootpart}asi{\rootpart} & Long slender object & \textsc{b}asi\textsc{t} & hair \emph{(cf. \textsc{b}uli\textsc{t} `head')} \\
    {\rootpart}uli{\rootpart} & Associated body part & \textsc{b}uli\textsc{t} & head \emph{(cf. i\textsc{b}aa\textsc{t} `to understand')}\\
    m{\rootpart}i{\rootpart} & Instrument, tool & m\textsc{r}i\textsc{q} & weapon \emph{(cf. \textsc{r}a\textsc{q} `pain')} \\
    i{\rootpart}u{\rootpart}a & Place of X/with X attribute & i\textsc{h}u\textsc{t}a & night \emph{(cf. \textsc{h}a\textsc{t}a `dark')} \\
    {\rootpart}ii{\rootpart} & Transitive verb & \textsc{f}ii\textsc{s} & to give birth to \emph{(cf. \textsc{f}ana\textsc{s} `person')} \\
    i{\rootpart}aa{\rootpart} & Intransitive verb & i\bigglot aa\bigglot & to act stupidly \emph{(cf. e\bigglot a\bigglot a `dumb')} \\
    {\rootpart}a{\rootpart}a & Primary attribute & \textsc{s}a\textsc{fr}a & hot \emph{(cf. \textsc{s}a\textsc{f}e\textsc{r} `heat')} \\
    {\rootpart}u{\rootpart}u & Animal & \textsc{b}u\textsc{rk}u & dog \emph{(cf. \textsc{b}a\textsc{rk} `bark')} \\
    {\rootpart}uu{\rootpart} & Country & \textsc{f}uu\textsc{ns} & France \emph{(cf. \textsc{f}u\textsc{ns}u `frog')}
    
    \end{tabular}
    \caption{Primary derivation patterns}
    \label{tab:primedevs}
\end{table}

\subsection{Gender}

Certain lexical items may be inflected to convey the gender of its referent. On certain words, namely \emph{-ara} greetings, gender marking is obligatory.

\begin{table}[ht]
    \centering
    \begin{tabular}{>{\em}ll}
    -un & Feminine gender \\
    -aj & Masculine gender \\
    -uj & Explicitly non-binary \\
    -an & Gender-neutral, agender \\
    \end{tabular}
\end{table}



\section{Inflectional morphology}

\subsection{Verb finals}

\subsection{Evidential modality}

\section{Pronouns and determiners}

\begin{table}[ht]
    \centering
    \begin{tabular}{rll}
        & \textit{Nonplural} & \textit{Plural} \\
    \textit{Speaker-only} & nas & naswi \\
    \textit{Addressee-only} & mi & miwi \\
    \textit{Inclusive} & nemi & nemiwi \\
    \end{tabular}
    \caption{Discourse participant pronouns}
    \label{tab:firstandsecond}
\end{table}

\begin{table}[ht]
    \centering
    \begin{tabular}{>{\em}rll}
        & \textit{Determiner} & \textit{Pronoun}  \\
    Proximal & wa & wase \\
    Medial & par & parse \\
    Distal & bu & buse \\
    Interrogative & li & lise \\
    \end{tabular}
    \caption{Determiners and demonstrative pronouns}
    \label{tab:determiners}
\end{table}

\chapter{Syntax}

\section{Verb stacking}

\section{Comparative constructions}

from-comparative, marks standard (to which is compared)

\pex[interpartskip=3ex]
\a
\begingl
\textsc{p}u\textsc{m}u[rabbit]
\textsc{f}ana\textsc{s}[person]
fun[from]
\textsc{m}a\textsc{nt}a[big]
-uru[\textsc{cop}]
\glft `The rabbit was bigger than a person.'
\endgl
\a
\begingl
\textsc{t}a\textsc{n}[\textsc{top}/time]
nemi[\textsc{qual}/\textsc{du.in}]
buse[\textsc{std}/\textsc{dist:pn}]
fun[\textsc{mrk}/from]
\textsc{j}a\textsc{l}[/many\_things]
-ila[/have]
\glft `We have more time than them.'
\endgl
\xe

\section{Animacy hierarchy}

\begin{table}[ht]
    \centering
    \begin{tabular}{ll}
    0 & Natural Forces \\
    1 & Pronouns (1>2>3) \\
    2 & Speakers of \lang{} \\
    3 & Non-speakers of \lang{} \\
    4 & Higher-order animals (mammals, octopus, intelligent creatures) \\
    5 & \parbox[t]{7cm}{Body parts, tools, any inanimate object used for acting upon something} \\
    6 & Lower-order animals \\
    7 & Plants \\
    8 & Inanimate objects \\
    9 & Abstract concepts 
    \end{tabular}
    \caption{Animacy hierarchy in nominals}
    \label{tab:hierarchy}
\end{table}

\chapter{Semantics and pragmatics}
\section{Phatic expressions}

Phatic expressions in \lang{} are all in some way related to the nouns they are derived from, suggesting an emphasis on acknowledging the addressee's current or upcoming actions. The addressee may respond with the same expression back, even if it does not apply to the original speaker in any way, or respond in kind with a more suitable expression.

The obligatory gender marking is a means of expressing your gender identity in an unintrusive manner.\footnote{The real reason is that as Beth once ended a conversation with "sayonara", Knut made note of some coincidental similarities with the word \textsc{s}a\textsc{j} `sleep' and the affix -un to indicate feminine gender, with the -ara reanalyzed as a phatic/optative marker of sorts.}

\section{Idiomatic expressions}

\textsc{c}u\textsc{mp}u \textsc{c}u\textsc{mp}uuru = no shit, preaching to the choir

\part{Dictionary}

\setsecnumdepth{chapter}
\settocdepth{section}

\begin{multicols}{2}
%
\section{Rootless Words}
\begin{description}
    \item[test] This is an entry
\end{description}

\section{\textsc{b---t}}



\end{multicols}

\part{Example Texts \& Translations}

\setsecnumdepth{subsubsection}
\settocdepth{subsubsection}

\end{document}