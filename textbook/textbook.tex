\documentclass[a5paper,10pt,twoside,openright]{memoir}
\usepackage{multicol, multirow, array}
\usepackage{fontspec}
\usepackage{anyfontsize}
\usepackage[pagecolor=none,dvipsnames]{xcolor}
\usepackage{ragged2e}
\usepackage{amsmath}
\usepackage{amssymb}
\usepackage[hidelinks]{hyperref}
\usepackage{url}
\usepackage[margin=0.8in]{geometry}
\usepackage{float, hhline}
\usepackage{graphicx}
\usepackage{booktabs}
\usepackage{textcomp}
\usepackage{expex}
\usepackage{enumitem}
\usepackage[calc,english]{datetime2}
\usepackage{suffix}

%-----CONFIGURATION------
%------------------------

\setmainfont{Charis SIL}[CharacterVariant=43:1]
\restylefloat{table}

\setsecnumdepth{subsubsection}
\settocdepth{subsubsection}

\lingset{glstyle=nlevel,numoffset=3em,textoffset=1.5em,exskip=.75ex,belowglpreambleskip=.25ex,aboveglftskip=.25ex}

\DTMnewdatestyle{eurodate}{%
    \renewcommand{\DTMdisplaydate}[4]{%
        \number##3.\nobreakspace%           day
        \DTMmonthname{##2}\nobreakspace%    month
        \number##1%                         year
    }%
    \renewcommand{\DTMDisplaydate}{\DTMdisplaydate}%
}

\DTMsetdatestyle{eurodate}

\renewcommand{\arraystretch}{1.4}

\lingset{belowpreambleskip=2ex}

%-------COMMANDS---------
%------------------------

\newcommand{\lang}{ɁaɁa-\textsc{f}a\textsc{m}}
\newcommand{\langeng}{Narish}
\newcommand{\landfam}{iɁuɁa}
\newcommand{\landeng}{Nareland}
\newcommand{\longv}{ː}
\newcommand{\sqbrack}[1]{$\langle$#1$\rangle$}
\newcommand{\phipa}[1]{/#1/}
\newcommand{\bripa}[1]{[#1]}
\newcommand{\ttilde}{\raise.17ex\hbox{$\scriptstyle\sim$}}
\newcommand{\rootpart}{$\Theta$}
\newcommand{\glotstop}{ʔ}
\newcommand{\bigglot}{Ɂ}
\newcommand{\lilglot}{ɂ}
\newcommand{\nm}{\symbol{"2205}}
\newcommand{\tiebar}{͡}
\newcommand{\famwordold}[5]{#1\textsc{#2}#3\textsc{#4}#5}
\newcommand\famword[1]{{\addfontfeatures{Letters=UppercaseSmallCaps}#1}}
\newcommand{\famq}[1]{»#1«}
\newcommand{\glem}[1]{\underline{\smash{#1}}}

\newenvironment{conversation}
    {\begin{table}[ht]
        \begin{tabular}{p{0.962\textwidth}}
        \toprule
        \textbf{Conversation} \\
        \midrule
    }
    { 
        \bottomrule
        \end{tabular}
    \end{table}
    }

\newcommand{\diagline}[2]{\hspace{2em} \textbf{\uppercase{#1:}} {#2} \\}

%-------TITLE PAGE-------
%------------------------

\title{\fontsize{70}{70}\selectfont \itshape \famword{iFaaMak!} \\ \fontsize{25}{30} \sffamily A Beginner's Guide to \langeng}
\author{Bethany E. Toma, Knut F. K. Ulstrup}
\date{}%\today}

%--------MAIN DOC--------
%------------------------

\begin{document}

\begin{titlingpage}
    \maketitle
    \begin{figure}[ht]
        \includegraphics[width=\textwidth]{nareland3small.png}
    \end{figure}
\end{titlingpage}

\frontmatter

\begin{quote}
    \textbf{\famq{\famword{\Large NaJDa FaM baj iFaaM se BaBauru \\ JaLa FaM baj iFaaM se TaLWuru \\ da TaPa FaM baj iFaaM se mi FaRaariibi.}}}

    \textit{\small ``Speaking a single language is easy, \\ speaking many languages is a burden, \\ but speaking all languages liberates you.'' \\ -Old \langeng proverb}
\end{quote}

\newpage

\tableofcontents

\mainmatter

\chapter{Introduction to \langeng}

Whether you want to learn more about \langeng{} culture, move to \landeng{}, or just communicate with your \langeng{} friends, we hope this textbook will give you an approachable yet thorough introduction to the \langeng{} language! By the end of this book you'll be able to navigate almost anything from everyday situations to even nuanced philosophical 

\chapter{The Narish alphabet}

Narish uses the latin alphabet, slightly modified to better suit the language.

\begin{table}[ht]
    \centering
    \begin{tabular}{ccccl}
        \toprule
        \multicolumn{3}{l}{\footnotesize letter cases} & \footnotesize sound & \footnotesize example \\
        \midrule
        \lilglot & \bigglot & \bigglot & /ʔ/ & like the break in `uh-oh' \\
        a & A & \textsc{a} & /a/            & h\glem{a}rd \\
        b & B & \textsc{b} & /p{\ttilde}b/  & \glem{b}ottle \\
        c & C & \textsc{c} & /c{\ttilde}c{\tiebar}ç/ & \glem{ch}ance \\
        d & D & \textsc{d} & /t{\ttilde}d/  & \glem{d}onkey \\
        e & E & \textsc{e} & /ə/            & salm\glem{o}n \\
        f & F & \textsc{f} & /f/            & \glem{f}acet \\
        h & H & \textsc{h} & /h{\ttilde}χ/  & \glem{h}appy, lo\glem{ch} \\
        i & I & \textsc{i} & /i/            & f\glem{ee}l \\
        j & J & \textsc{j} & /j/            & \glem{y}awn \\
        k & K & \textsc{k} & /k{\ttilde}k{\tiebar}x/ & \glem{c}ar \\
        l & L & \textsc{l} & /l/            & be\glem{l}ieve \\
        m & M & \textsc{m} & /m/            & \glem{m}an \\
        n & N & \textsc{n} & /n/            & \glem{n}otion \\
        p & P & \textsc{p} & /pʰ{\ttilde}p{\tiebar}ɸ/& \glem{p}allet \\
        q & Q & \textsc{q} & /q/            & like /k/ further back \\ 
        r & R & \textsc{r} & /ɾ{\ttilde}ɹ/  & bu\glem{tt}er, \glem{r}eally \\
        s & S & \textsc{s} & /s/            & \glem{s}ing \\
        t & S & \textsc{t} & /tʰ{\ttilde}t{\tiebar}s/& \glem{t}ime \\
        u & U & \textsc{u} & /u/            & h\glem{u}l\glem{u} \\
        w & W & \textsc{w} & /w/            & \glem{w}ater \\
        y & Y & \textsc{y} & /j/            & \glem{y}awn \\
        \bottomrule
    \end{tabular}
\end{table}

\part{Beginner's grammar}

\chapter{Hello, I am...}

Of course, the first part of any conversation is greetings. It's important that you convey greetings and wishes properly, as their exact forms are determined by your gender expression.

\begin{conversation}
    \diagline{A}{\famword{FaSajara}.}
    \diagline{B}{\famword{FaSunara}.}
\end{conversation}

In the above example, person A identifies themselves as male by using the male form \emph{\famword{FaS\glem{aj}ara}}, and person B identifies themselves as female by means of the female form \emph{\famword{FaS\glem{un}ara}}. When you say hello, you're letting the other person know exactly how you identify, and they're also telling you back! There are in total four different forms, as listed below:

\begin{table}[ht]
    \centering
    \begin{tabular}{rc}
        \toprule
        Male & \famword{FaSajara} \\
        Female & \famword{FaSunara} \\
        Nonbinary & \famword{FaSujara} \\
        Unspecified & \famword{FaSanara} \\
        \bottomrule
    \end{tabular}
\end{table}

It's okay to switch between the forms if you should feel so inclined, and most \langeng{}ers don't mind. \landeng{} overall is very supportive of LGBTQ+ communities.

The unspecified form also works as an agender form. You'll often see companies use \emph{\famword{FaSanara}}, because they often speak on behalf of their employees whose identities can't be unanimously determined.

Just saying hello doesn't cut it a lot of the time, often you have to let people know your name as well. 

\begin{conversation}
    \diagline{a}{\famword{FaSajara, nas ha KNUTuru.} \emph{`Hello, I am Knut.'}}
    \diagline{b}{\famword{KNUT FaSunara, nas fu BETHANYuru.} \emph{`Hi Knut, I am Bethany.}}
\end{conversation}

In the above conversation, you see the suffix \emph{-uru}. This suffix does the same thing as `to be' in English, saying that something \emph{is} something. So, \emph{\famword{`KNUTuru'}} translates into `is Knut', `are Knut', or `am Knut' depending on the context.

The other thing you see is \emph{ha} and \emph{fu}. These prepositions are to show the gender of the person when saying their name. Note that they are \emph{only} used when talking about a person, not if you're calling someone's name or talking about the name as a concept. You can see B doesn't use \emph{ha} when saying A's name because she is directly addressing A.

\begin{table}[ht]
    \centering
    \begin{tabular}{rc}
        \toprule
        Male & ha \\
        Female & fu \\
        Nonbinary & \multirow{2}{*}{na} \\
        Unspecified &  \\
        \bottomrule
    \end{tabular}
\end{table}

The third preposition \emph{na} is used for both nonbinary and unspecified/agender people as there isn't a songle most common convention for distinguishing the two. However, people may have colloquial prepositions that do make that distinction, so don't be afraid to stray from what is taught in this book!

\chapter{Thanks}

Remember how \famword{FaSajara} is made of \famword{FaS} and some other bits? The -ara suffix constructs these phrases from nouns, together with one of the gender suffixes. The noun can be anything related to what you're currently doing, even saying things like "sorry, I forgot" and even "have a nice flight".

In this chapter's example conversations, A is serving food to B as a waiter or host.

\begin{conversation}
    \diagline{A}{\famword{NaMajara}! \emph{`enjoy your meal!'}}
    \diagline{B}{\famword{NaMunara}! \emph{`thank you!'}}
\end{conversation}

See how B responds with \famword{NaMunara}? The way you say thank you in response to someone's wellwishes is to repeat the expression right back. 

\begin{conversation}
    \diagline{A}{\famword{NurMiila}. \emph{Here's your meal.}}
    \diagline{B}{\famword{NaMunara}. \emph{Thank you.}}
    \diagline{A}{\famword{NaMajara}! \emph{You're welcome, enjoy!}}
\end{conversation}

In this conversation, B says the -ara expression first to say thank you, and A is the one to respond. The response here can be translated as "you're welcome", not "thank you" like in the first conversation. 

It doesn't matter which individual says \emph{\famword{NaMajara/NaMunara}} first. It's okay to not say anything if you're unsure, in most cases the other person will notice and pick up the helm.

\chapter{Yes, No, I am, Good}

Unlike English, \lang{} doesn't have specific words for "yes" and "no". Surprisingly, it's is not as universal as you'd think! But just because you don't have "yes" and "no" doesn't mean that there's no way to agree or disagree. Instead of saying yes, you simply repeat the verb in the question:

\begin{conversation}
    \diagline{a}{\famword{QarFi KiiLli?} \emph{`do you drink coffee?'}}
    \diagline{b}{\famword{KiiL.} \emph{`[yes,] I drink.'}}
    \diagline{b}{\famword{mi da PurNi NiiMli?} \emph{`but do you eat bread?'}}
    \diagline{a}{\famword{NiiM.} \emph{`[yes,] I eat.'}}
\end{conversation}

As you can also see, the way to make pretty much any statement into a question is to add \emph{-li} to the verb. 

\chapter{What? What's this? This is...}

\part{Intermediate Grammar}

\chapter{Forming words}

The peculiar writing style will seem odd to most, but it is this way for good reason. In many words, certain letters are more important than others as they point to a common overarching meaning, called the \emph{root}, and it's important that these letters stand out for readability.

In the words \famword{KarLi} \textit{`water'}, \famword{KaLa} \textit{`wet'}, \famword{iKaaL} \textit{`it's raining'}, and \famword{aKiLu} \textit{`bottle'}, the letters \textsc{k} and \textsc{l} are part of the root \emph{K-L}, which doesn't have a meaning of its own, but clearly ties together all these words related to water in some way. All of these words are comprised of two root consonants and a template called a \emph{pattern} that these consonants slot into. 

\end{document}